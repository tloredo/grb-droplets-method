%..............................................................................
\section{Comparison with prior Bayesian work}
\label{app:WMJ11}

Watson, Mortlock, and Jaffe \cite{WMJ11-BayesUHECR} (WMJ11) have
independently developed an approximate Bayesian approach for assessing
evidence for association of the PAO-08 data (with data for the 27 UHECRs in
periods 1 and 2) with nearby AGN.  They consider the best-fit CR directions
as points drawn from a Poisson intensity on the celestial sphere constructed
as a mixture of distributions located at AGN directions, with weights
reflecting the AGN distances (corresponding to a standard-candle
source intensity), and an isotropic background component.  The source
components have the form of a ``two-dimensional Gaussian on the sphere,''
with the probability density for a best-fit measured direction $\hat n$
arising from an AGN at $\hdrxn$ given by
\be
p(\hat n|\hdrxn)
  = \frac{1}{2\pi\left(1 - e^{-2/\sigma^2}\right)}
    \exp\left(-\frac{1 - \hat n\cdot\hdrxn}{\sigma^2}\right),
\label{sphere-gauss}
\ee
with $\sigma$ dubbed the ``smearing angle.'' This is a Fisher distribution
(as in equation~\ref{rho-def}), with concentration parameter $\kappa =
1/\sigma^2$.  They focus on a model with  $\sigma = 3^\circ$ ($\approx
0.052$~rad) [sic], intended to reflect a combination of $\approx 1^\circ$
measurement uncertainties and few-degree magnetic deflections.  They also
provide a few results for $\sigma = 6^\circ$ and $10^\circ$ [sic].\footnote{We
note that the parameter $\sigma$ is not an angle and does not have angular
units (degrees or radians).  We presume these dimensionally inconsistent
equations imply solving the nonlinear equation relating $\sigma$ (or $\kappa$)
to the stated angular scale; e.g., for a 68.3\% confidence or credible region
with angular radius $\theta$, in the small-angle limit $\sigma \approx
0.66\theta$ (e.g., $\sigma \approx 0.035$ for $3^\circ$ uncertainties;
see equation~(\ref{kappa-theta})).}
They calculate an approximate likelihood function by finely pixelizing the
celestial sphere, calculating the number of cosmic ray direction
measurements expected in each pixel, and multiplying Poisson counting
probabilities for the bins (with one count for bins containing a best-fit
cosmic ray direction, and zero counts for the remaining bins).

WMJ11 adopt a similar candidate host population as was used in the PAO-07 and
PAO-08 analyses (the nearby AGN in the 12th VCV compilation; WMJ11 consider
$\approx 900$ AGN within 100~Mpc) on the presumption that it is almost
complete for the nearest AGN.  They conclude that there is ``strong evidence
of a UHECR signal from the known VCV AGNs,'' such that at least some UHECRs
come from AGN in the VCV catalog (or from sources within a few degrees of the
AGN).  For a $3^\circ$ smearing angle, the marginal posterior density for the
fraction associated with AGN (our $f$ parameter) has a mode of 0.15, and 68\%
highest density credible interval of $[0.08, 0.25]$.  For $6^\circ$ and
$10^\circ$ smearing angles the estimated association fraction is larger, but
the marginal distribution is also somewhat broader.  They do not compare their
association model to an alternative, and thus do not compute Bayes factors.

Our analysis framework and our results differ in significant ways from those
of WMJ11 (focusing on our results for the data from Periods~1 and 2). 
Methodologically, our approach is based on explicit modeling of associations
(via marginal likelihood factors associating a particular cosmic ray with a
particular AGN or the background), rather than considering best-fit cosmic
ray directions to be samples from a point process.  As noted in
\S~\ref{sec:dtxn}, a factor in the likelihood function in our approach
is analogous to that underlying FMMs, but this
mixture-like factor arises as a consequence of some of our modeling
assumptions; it is not a starting point, and it does not hold for all
astrophysically interesting association models that our framework
accommodates.  For example, in \S~\ref{sec:summary} we describe a more
realistic family of magnetic deflection models (``radiant'' models with
exchangeable rather than IID deflections for cosmic rays comprising a
multiplet) whose likelihood function does not have the simple FMM form.  In
other common astronomical coincidence assessment problems, such as
establishing associations of sources detected in different electromagnetic
wavebands, only singlet associations are meaningful; such models have no FMM
representation but may be accommodated by our framework.  Further
discussion of this is in \cite{Loredo12-Coinc}.

Another point of departure in methodology is that we distinguish measurement
error from magnetic deflection.  In the buckshot model adopted here, both
the measurement error and deflection distributions are Fisher distributions
(note that the composition of these distributions is not a Fisher
distribution).  In radiant models, for example, the deflection distribution
is more complicated.  Explicit, separate treatment of these physically
distinct effects enables handling heteroskedastic measurement errors
for the UHECR directions (PAO-10 reports only a typical measurement
error, but future catalogs will hopefully report the heteroskedastic
uncertainties found in detailed air shower fits).  Heteroskedastic
uncertainties further thwart a simple FMM representation for the likelihood,
again emphasizing the need for a framework built on explicit modeling
of individual associations.

Our approach also provides explicit estimates of probabilities for possible
associations.  In our MCMC algorithm, these can be found by calculating
frequencies for different values of the $\lambda$ labels; we report
such results in \S~\ref{sec:results}.  WMJ11 report a weight for
candidate associations, but note it is not a rigorous probability.

WMJ11 analyze the data from Periods~1 and 2 jointly, without commenting on the
possible effects of tuning on the implications of the Period~1 data.

Turning to astrophysical differences, we adopt the G10
\cite{2010MNRAS.406..597G} volume-complete catalog of nearby AGN as a
candidate host catalog, rather than the 12th VCV catalog used by WMJ11 and
others.  Notably, 6 of the 17 AGN in the G10 catalog are not in the 12th VCV
catalog (one of them appears in the more recent 13th VCV catalog).  We chose
the G10 catalog both for its completeness, and because use of a small
catalog was convenient for an initial study, as it enabled more extensive
analyses of real and simulated data than would be possible with AGN from
VCV. Figure~\ref{fig:f1000} shows our marginal posterior density for $f$ for
$\kappa = 1000$ (corresponding to a deflection scale $\approx 2.7^\circ$)
for model $M_1$, for different subsets of the PAO-10 data. The mode based on
the combined data from Periods~1 and 2 is at $f \approx 0.1$, about a
two-thirds of the WMJ11 value of $\approx 0.15$, even though their catalog
contains more than 50 times as many potential counterparts.  Our common
assumption of a standard candle intensity distribution is probably the main
reason that the results are not more discrepant. In particular, although
WMJ11 include AGN at distances to 100~Mpc in their catalog, the standard
candle assumption forces the analysis to assign negligible detectable rates
to all but the closest few AGN.  The similarity of our estimates despite the
disparity between our AGN catalog sizes highlights how restrictive the
standard candle assumption is.

\begin{figure}
\centerline{\includegraphics[angle=-90,width=.9\textwidth]{margf_kappa1000_17AGNs.eps}}
\caption{Posterior distributions for $f$ for model $M_1$, conditioned on
$\kappa$ = 1000, for various subsets of the PAO-10 data.}
\label{fig:f1000}
\end{figure}

We explore a far greater range of magnetic deflection angular scales
than did WMJ11.  Their discussion of deflection scales implicitly
presumes UHECRs are light nuclei.  Recent cosmic ray data and
theoretical models motivate serious consideration of the possibility
that many or most UHECRs are heavy nuclei, as noted above.

Finally, we differ qualitatively in our conclusions about the strength of
evidence for association of UHECRs with nearby AGN, particularly after
examining period-to-period differences, and considering the impact of
period~3 data (unavailable to WMJ11).  We quantify the support for
association hypotheses by calculating Bayes factors explicitly comparing
association and null models, both conditional on $\kappa$ (in
Figure~\ref{fig:BFplot}) and marginalized over a broad $\kappa$
range (in Table~\ref{BFTable}).

WMJ11 do not calculate Bayes factors comparing their association and null
models.  Their claim of strong evidence for association appears to be based on
the small marginal posterior density for values of the association fraction
near zero.  But this fails to distinguish parameter estimation from
model assessment.  As long as the likelihood for models with $f=0$ does not
vanish, parameter estimation (with a continuous prior on $f$) simply does not
address how the $f=0$ hypothesis compares to alternatives.  To do this
requires assigning a finite prior probability for $f=0$, which we do here by
considering it as a separate model and calculating Bayes factors.  The Bayes
factor comparing nested models depends on the size of the parameter space of
the larger model in a way that accounts for ``fine tuning'' of the additional
model parameters:  the larger model will have parameter values producing
better fits than the smaller model, but if the values of the additional
parameters are close enough to the default values corresponding to the smaller
model, the marginal likelihood for the larger model will be {\em smaller} than
that for the smaller model (the well-known ``Ockham's razor'' behavior of
Bayes factors).  Focusing on low-dimensional marginal distributions, such as
the posterior density for $f$, can give an exaggerated impression of the
strength of evidence for the larger model because it suppresses the large
volume of parameter space associated with its additional parameters.  Here,
association models have not only the $f$ parameter, but also many latent label
parameters (i.e., many association hypotheses that cannot be ruled out a
priori).  Calculation of the Bayes factor takes all of this into account.
Using the Period~1 and Period~2 data available to WMJ11 does in fact produce
large Bayes factors favoring association.  But partitions of the data
excluding Period~1 data produce much smaller Bayes factors, even though the
$p(f)$ distributions found with these partitions assign very small density to
$f=0$. The Bayes factor calculations indicate that the complexity of
association models may not be justified by existing data.

Perhaps most importantly from an astrophysical perspective, we performed
more extensive checking of our models, calling into question the adequacy of
our shared isotropic background and standard candle assumptions.  We discuss
this further in \S~\ref{sec:summary}.

