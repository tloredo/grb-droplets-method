\section{Modeling the cosmic ray data}

% We adopt a Bayesian
% approach for directional coincidence assessment based on multilevel modeling,
% where upper levels in a model describe properties of potentially associated
% source populations and radiation propagation, and lower levels describe
% measurement errors and survey selection effects.

The basic statistical problem is to quantify evidence for associating some
number (possibly zero) of cosmic rays with each member of a candidate
source population.  The key observable is the cosmic ray direction; a set of
rays with directions near a putative host comprises a multiplet potentially
associated with that host.  This gives the problem the flavor of model-based
clustering (of points on the celestial sphere rather than in a Euclidean space),
but with some novel features:
\begin{itemize}
\item The model must account for measurement error in cosmic ray properties.

\item Observatories provide an incomplete and distorted sample of cosmic rays,
so the model must account for random truncation and nonuniform thinning.

\item The most realistic astrophysical models imply a joint distribution for
the properties of the cosmic rays assigned to a particular source that is
exchangeable rather than a product of independent distributions (as is the
case in standard clustering).

\item The number of cosmic rays is informative about the intensity scale of
the cosmic ray sources so the binomial point process model underlying
standard generative clustering approaches is not appropriate.
\end{itemize}

%1: Sources as a population
%2: Production per source
%3: Propagation
%4: Measurement

To account for these and other complexities, we model the data using a
hierarchical Bayesian framework with four levels:
\begin{enumerate}
\item {\em Source properties}:  At the top level we specify the properties
of the sources of cosmic rays.  This may include the choice of a candidate
source population of identified objects (e.g., a particular galaxy
population), and/or specification of the properties of a population of
unidentified sources.  For a given candidate source population, we must
specify source directions and cosmic ray intensities.  The simplest case is
a standard candle model, with each source having the same cosmic ray
intensity.  More generally, we may specify a (non-degenerate) distribution
of source intensities; this corresponds to specifying a ``luminosity
function'' in other astronomical contexts.  For a population of unidentified
sources, we must specify a directional distribution (isotropic in the
simplest case) as well as an intensity distribution.
\item {\em Cosmic ray production}:  We model the production of cosmic rays
from each source with a marked Poisson point process model for latent cosmic
ray properties.  The incident cosmic ray arrival times have a homogeneous
intensity measure in time, and the marks include the cosmic ray energies,
and latent categorical labels identifying the source of each ray.
\item {\em Cosmic ray propagation}:  Next we model magnetic deflection of
the rays, scattering their directions from the source directions.  
This requires introducing a latent arrival direction parameter for each ray.
Here we adopt a simple phenomenological model with a single parameter
specifying a typical scattering scale between the source and arrival
directions.  As the data become more abundant and detailed, the framework
can accommodate more complex models, e.g., with parameters explicitly
describing cosmic magnetic fields and cosmic ray composition.
\item {\em Detection and measurement}:  Last, we model detection and
measurement, accounting for truncation and thinning of the incident
cosmic ray flux, and measurement errors for directions and energies.
\end{enumerate}

Figure~\ref{fig:levels} schematically depicts the structure of our
framework, including identification of the various random variables
appearing in the calculations described below.  The variables will be
defined as they appear in the detailed development below; the figure serves
as visual reference to the notation.  The figure is not a graphical model
per se.  Rather, our models specify probability distributions over a space
of graphs, each graph corresponding to a possible set of associations of the
cosmic rays with particular sources.  This framework builds directly on an
earlier multilevel Bayesian model we developed to assess evidence that some
sources of gamma-ray bursts repeat \cite{LLW96}; this model, too, worked in
terms of probability distributions over candidate assignments.  See
\cite{Loredo12-Coinc} for a broad discussion of Bayesian methods for
assessing spatio-temporal coincidences in astronomical data.

\begin{figure}
\centerline{\includegraphics[clip=true,width=\textwidth]{CRCoincLevels.eps}}
\caption{Schematic depiction of the levels in our cosmic ray association
models, identifying random variables appearing in each level, including
parameters of interest (bold red labels), latent variables representing cosmic ray
properties that are not directly observable (slant type labels), and
observables (bold blue labels).}
\label{fig:levels}
\end{figure}

Our framework is designed to enable investigators to:
(1)~Ascertain which cosmic rays (if any) may be associated with specific
sources with high probability; (2)~Estimate luminosity function parameters
for populations of astrophysical sources; (3)~Estimate the proportion of all
detected cosmic rays generated by each population; (4)~Estimate parameters
describing the composition-dependent effects of cosmic magnetic fields;
(5)~Investigate whether cosmic rays from a single source are deflected
independently or share part of their deflection history (resulting in
correlated deflections).  Task (5) is not attempted here but will be
investigated in the future.

%..............................................................................
\subsection{Cosmic ray source properties}

% Note that the cosmic rays from a particular source arrive from
% different directions, so we cannot think of the flux $F_k$ in regard to
% a detector with unit area normal to the source direction; there is
% no one such direction.

We do not anticipate the UHECR flux passing through a volume element at the
Earth to vary in time over accessible time scales, so we model the
arrival rate into a small volume of space from any particular direction as a
homogeneous Poisson point process in time.  Let $F_k$ denote the UHECR flux
from source $k$.  $F_k$ is the expected number of UHECRs per unit time from
source $k$ that would enter a fully exposed spherical detector of unit
cross-sectional area.  A cosmic ray source model must specify the directions
and fluxes of candidate sources.  In our framework, a candidate source
catalog specifies source directions for a fixed number of potential sources,
$N_A$ ($N_A = 17$ for the G10 AGN catalog).  In addition, we presume some
cosmic rays may come from uncatalogued sources, so we introduce a background
component, labeled by $k=0$, considered to be a population of isotropically
distributed ``background'' sources.   We presume the background sources to
be numerous and to each have relatively low cosmic ray fluxes, so that at
most a single cosmic ray should be detected from any given background source
(i.e., we do not consider clustering of cosmic rays assigned to the
background).  In this limit, the background component may be described
by a single parameter, $F_0$, denoting the total flux from the entire
background population.

%\enote{In regard to the last comment:  I think we are implicitly presuming
%only a single CR from each background source.  This may be a slightly
%inconsistent simplification that should be noted.  E.g., for the radiant
%model, if a single unidentified background source produces a multiplet, in
%the radiant model this could lead to a ``ray'' of background sources with
%correlated directions but no apparent source.  For the buckshot model, there
%could be clusters with no apparent center.  This could be handled by
%introducing separate parameters for the total background flux, and the
%background source number density. Let's not go there for now!}

A model must specify a distribution for $\{F_k\} = \{F_0, \Fvec\}$; in
astronomical lingo, this corresponds to specifying a ``luminosity function''
for the background and source populations.  As a simple starting point, we
treat $F_0$ as a free parameter, and adopt a ``standard candle'' model
specifying the $N_A$ candidate host fluxes, $\Fvec$, via a single parameter
as follows.  We assume all sources emit isotropically with the same
intensity, $I$ (number of cosmic rays per unit time), so the flux from a
source (i.e., $F_k$ for $k>0$) can be written as $F_k = I/D_k^2$ (the
inverse-square law), with $D_k$ the (known) distance to source $k$ (there
could also be distance- and energy-dependent attenuation due to cosmic
ray-photon interactions, but the sources we consider here are close enough
that such attenuation should be negligible).  The total flux from the
sources is $F_A = \sum_{k>0} F_k$, and we adopt $F_A$ as the source
intensity parameter rather than $I$.  Thus $F_k = w_k F_A$, with the weights
$w_k$ given by
\be
w_k = \frac{1/D_k^2}{\sum_{j=1}^{N_A} 1/D_j^2},
\label{wt-def}
\ee
for $k=1$ to $N_A$.

%..............................................................................
\subsection{Top-Level Prior Specification}

We must specify a prior distribution for $F_0$ and $F_A$.
Earlier observations constrained the total UHECR flux.  In our association
model, the total flux is $F_T = F_0 + F_A$.  For the null model, there is
only one top-level parameter, the total flux from an isotropic distribution
of source directions.  So we adopt $F_T$ as a top-level parameter,
common to all models.  For association models, this motivates an alternative
parameterization that switches from $(F_0, F_A)$ to $(F_T, f)$, where $f =
F_A/(F_0 + F_A)$ is the fraction of the total flux attributed to the
candidate host population.
In this parameterization, we can specify a common total flux prior
for all models.  This is astrophysically sensible since we have
results from prior experiments to set a scale for the total flux.
It is also statistically desirable; Bayes factors tend to be robust
to specification of priors for parameters common to models being compared.

We adopt independent priors for the total flux and the associated fraction.
If their prior densities are $g(F_T)$ and $h(f)$, then the implied
joint prior density for $(F_0,F_A)$ is
\be
\pi(F_0, F_A) =
  \frac{g(F_0+F_A) h\left(\frac{F_A}{F_0+F_A}\right)}{F_0+F_A},
\label{FF-prior}
\ee
where the denominator is from the Jacobian of the transformation between
parameterizations.  In general, an independent prior for $F_T$ and $f$
corresponds to a dependent prior for $F_0$ and $F_A$.

For the calculations below, we adopt an exponential prior with scale $s$ for
$F_T$, and a beta prior for $f$ with shape parameters $(a,b)$, so
\be
g(F_T) = \frac{1}{s}e^{-F_T/s}
\quad\text{and}\quad
h(f) = \frac{1}{B(a,b)} f^{a-1}(1-f)^{b-1},
\label{exp-beta}
\ee
where $B(a,b)$ is the beta function.  We set the hyperparameters $(s,a,b)$
as follows.

We take $s = 0.01\times 4\pi$ km$^{-1}$ y$^{-1}$ for all models.  This scale
is compatible with flux estimates from AGASA and HiRes.  The likelihood
functions for $F_T$ from those experiments are formally different from
exponentials (they are more concentrated away from zero), but since this
prior is common to all models, and since the PAO data are very informative
about the total flux, our results are very robust to its detailed
specification.

For the beta prior for $f$, our default choice is $a=b=1$, which corresponds
to a uniform prior on $[0,1]$.  We also repeat some computations using $b=5$
to investigate the sensitivity of Bayes factors to this prior.  This case
skews the prior downward, increasing the probability that $f$ is close to 0.

%..............................................................................
\subsection{Cosmic ray mark distributions}

Given the fluxes, we model cosmic ray arrival times with a superposition of
homogeneous Poisson point processes from each component.  Besides its
arrival time, each cosmic ray has a label associated with it, identifying
its source component. Let $\tlabel$ be an integer-valued latent label for a
UHECR, specifying its source ($\tlabel = 0$ for the background, or $k \ge 1$
for AGN $k$).  Since a superposition of Poisson processes is a Poisson
process, we may consider the arrival times for the UHECRs arriving at Earth
to come from a total event rate process, and the labels to come from a
categorical mark distribution with probability mass function
\be
P(\lambda=k|F_0,\Fvec) = \frac{F_k}{\sum_{j=0}^{N_A} F_j}.
\label{label-pmf}
\ee
In the absence of magnetic deflection, the labels could be replaced by
source directions (with background source directions assigned
isotropically), and the process could be considered to be Poisson in time
with a directional mark distribution.  But magnetic deflection requires a
more complex setup.

Our full framework also assigns energies as marks for each cosmic ray, drawn
from a distribution describing the cosmic ray spectrum.  The energies would
be important in an analysis that seeks to infer the cutoff energy
distinguishing local UHECRs rays from lower-energy cosmic rays (i.e., the
GZK cutoff).  Although the PAO-10 catalog includes event energies, the PAO
team has already made an energy cut, and in the absence of lower-energy
data, we cannot usefully infer a cutoff.  Thus in the analysis presented
here, we ignore the energy mark distribution.  
(For models with more complex
luminosity functions and more distant sources, the energies would play a
role in accounting for the suppression of flux from distant sources due to
interactions with cosmic backgrounds.)

%..............................................................................
\subsection{Cosmic ray deflection}
\label{sec:dflxn}

After leaving a source, UHECRs will have their paths deflected as they
traverse galactic and intergalactic magnetic fields.  The Galactic field is
partially measured and is known to have both a turbulent component (varying
over length scales below $\sim 1$~kpc) and a regular component (coherent
over kpc scales and largely associated with spiral arms), with typical field
strengths $\sim 1~\mu$G.  The magnetic fields of other galaxies are at
best crudely measured and believed to be similar to the Galactic field. The
much smaller fields in intergalactic space are only weakly constrained (in
fact, cosmic rays might provide useful additional constraints); the typical
field strength is probably not larger than $\sim 10^{-9}$~G except within
galaxy clusters.

A number of investigators have modeled cosmic ray
propagation in the Galaxy, or in intergalactic space, using physical models
based on existing field measurements (recent examples include
\cite{HRM02-Lens,HMR02,D+05-CRDflxn,NM10-CRDflxn,AKP10-CRDflxn,J+10-CRSources}
; see \cite{Sigl12} for an overview).  Roughly speaking, there are two regimes
of deflection behavior, described here in the small-deflection limit
\cite{HMR02}.  As a cosmic ray with energy $E$ and atomic number $Z$ traverses
a distance $L$ spanning a regular magnetic (vector) field $\bm{B}$, it is
deflected by an angle
\be
\delta \approx
  6.4^\circ\; Z \left(\frac{E}{50~\mbox{EeV}}\right)^{-1}
  \left| \int_L \frac{d\bm{s}}{3~\mbox{kpc}} \times 
          \frac{\bm{B}}{2~\mu\mbox{G}} \right|,
\label{dflxn-reg}
\ee
where $\bm{s}$ (a vector) is an element of displacement along the
trajectory; the field and length scales are typical for the Galaxy.  If
instead it traverses a region with a turbulent structure, with the field
coherence length $\ell\ll L$, then the deflection will be stochastic; its
probability distribution has zero mean, and root-mean-square (RMS) angular
scale
\ba
\delta_{\rm rms}
  &\approx& 1.2^\circ\; Z \left(\frac{E}{50~\mbox{EeV}}\right)^{-1}
    \left(\frac{B_{\rm rms}}{4~\mu\mbox{G}}\right)
    \left(\frac{L}{3~\mbox{kpc}}\right)^{1/2}
    \left(\frac{\ell}{50~\mbox{pc}}\right)^{1/2}\\
  &\approx& 2.3^\circ\; Z \left(\frac{E}{50~\mbox{EeV}}\right)^{-1}
    \left(\frac{B_{\rm rms}}{1~\mbox{nG}}\right)
    \left(\frac{L}{10~\mbox{Mpc}}\right)^{1/2}
    \left(\frac{\ell}{1~\mbox{Mpc}}\right)^{1/2},\nonumber
\label{dflxn-turb}
\ea
where $B_{\rm rms}$ is the RMS field strength along the path, and quantities
are scaled to typical galactic and intergalactic scales on the first and
second lines, respectively.

For a detected cosmic ray, the energy is measured fairly accurately, but
other quantities appearing in the deflection formulae may be largely
unknown.  As noted above, there is significant uncertainty in the magnitudes
of cosmic magnetic fields, particularly for turbulent structures.  Turbulent
length scales are poorly known.  Finally, the composition (distribution of
atomic numbers) of UHECRs is not known.  Low energy cosmic rays are known to
be mainly protons and light nuclei, but the proportion of heavy nuclei (with
$Z$ up to 26, corresponding to iron nuclei, the most massive stable nuclei)
increases with energy up to about $10^{15}$~eV.  At higher energies,
inferring the cosmic ray composition is very challenging, requiring both
detailed measurement of air shower properties, and theoretical modeling
of the $Z$ dependence of hadronic interactions at energies far beyond
those probed by accelerators.  Measurements and modeling from HiRes
indicate light nuclei are predominant again at $\approx 1$~EeV and remain
so at least to $\approx 40$~EeV \cite{HiRes10-Final-arxiv}.
In contrast, recent PAO measurements indicate a transition from light
to heavy nuclei over the range $\approx 3$--30~EeV
\cite{PAO10-Composition,PAO12-Composition}.
(The discrepancy is not yet explained.)
For heavy nuclei, the deflection scales in both the regular and
turbulent deflection regimes can be large, $\sim 1$~rad. Some
investigators have suggested that many or most UHECRs may be heavy
nuclei originating from the nearest AGN, Cen~A, so strongly deflected
that they come from directions across the whole southern sky (e.g.,
\cite{B+09-CenA,GBdS10-CenA,BdS12-CenA}).

%\enote{All the PAO composition papers I found ignore the conflict with
%HiRes, except for a brief footnote in arxiv:1106.3048 (now in JCAP), a PAO
%paper on the Cen~A hypothesis.}


%It is difficult to measure the composition of UHECRs.  Lower-energy cosmic
%rays are mostly protons, but PAO sees suggestive evidence that there may be
%heavier nuclei among UHECRs \cite{PAO10-Composition,PAO12-Composition}.

In light of these uncertainties and the relative sparsity of UHECRs,
we use simple phenomenological models for magnetic deflection.  
In the simplest ``buckshot'' model, each cosmic ray from a particular source
experiences a deflection that is conditionally independent of the deflection
of other rays from that source, given a parameter, $\kappa$, describing the
distribution of deflections.  We have also devised a more
complex ``radiant'' model that allows cosmic rays assigned to the same
source to have correlated deflections, with the correlation representing
a partially shared deflection history.  For the analyses reported here, we
use the buckshot model; we describe the radiant model further in
\S~\ref{sec:summary}.

The buckshot deflection model adopts a Fisher distribution for the
deflection angles.  The model has a single parameter, $\kappa$, the
concentration parameter for the Fisher distribution.  The probability
density for observing a cosmic ray from direction $\tdrxn$ if it is assigned
to source $k$ with direction $\hdrxn_k$ is then
\be
\rho_k(\tdrxn|\kappa)
  = \frac{\kappa}{4\pi\sinh(\kappa)}\exp(\kappa\tdrxn\cdot\hdrxn_k).
\label{rho-def}
\ee
With this deflection distribution, when a cosmic ray is generated from an
isotropic background population, its deflected direction still has an
isotropic distribution.  Accordingly,
\be
\rho_0(\tdrxn|\kappa) =  \frac{1}{4\pi}.
\label{rho-iso}
\ee

The $\kappa$ parameter is convenient for computation, but an angular scale
is more convenient for interpretation.  The contour of the Fisher density
bounding a region containing probability $P$ is azimuthally symmetric
with angular radius $\theta$ satisfying
\be
\int_{\Omega}d \tdrxn \, \rho_k(\tdrxn|\kappa)= 
  \frac{1-e^{-\kappa[1-\cos(\theta)]}}{1-e^{-2\kappa}}=P,
\label{kappa-theta}
\ee
where $\Omega$ denotes the cone of solid angle subtended by the contour. In
plots showing $\kappa$-dependent results, we frequently provide an angular
scale axis, using (\ref{kappa-theta}) with $P=0.683$, in analogy to the
``$1\sigma$'' region of a normal distribution.\footnote{In the $\kappa\gg 1$
limit, the Fisher density becomes an uncorrelated bivariate normal with
respect to locally cartesian arc length coordinates about the mode on the
unit sphere.  The standard deviation of this normal distribution, $\theta$,
satisfies equation~(\ref{kappa-theta}) with $P \approx 0.683$; for $\kappa\gg
1$ this implies $\theta^2 \approx 2.30/\kappa$, or
$\theta \approx 86.9^\circ/\kappa^{1/2}$.}

% From full formula:
% def c1(k): return -log(1-(1-exp(-2*k))*.683)/k
% def ang(k):  return arccos(1-c1(k))*180/pi
% kappa  sig
% .5     97.5 deg
% 1      83.8
% 10     27.7
% 100    8.69
% 1000   2.75

Note that, astrophysically, $\kappa$ has a nontrivial interpretation.  If
all UHECRs are the same nuclear species (e.g., all protons), then $\kappa$
depends solely on the magnetic field history experienced by cosmic rays as
they propagate to Earth.  If UHECRs are of unknown or mixed chemical
composition, then $\kappa$ conflates magnetic field history and composition.
In a more complicated model, there could be a distribution for the values
of $\kappa$ assigned to UHECRs (accounting for different compositions and
magnetic field histories); the distribution could depend on source direction
(accounting for known magnetic field structure in the Galaxy and perhaps in
intergalactic space) and on source distance (related to the path length in
intergalactic space).

When estimating $\kappa$ or marginalizing over it, we adopt a log-flat prior
density for $\kappa\in[1,1000]$,
\be
p(\kappa) = \frac{1}{\log 1000}\,\frac{1}{\kappa}, \text{ for } 1\leq\kappa\leq1000.
\label{k-prior}
\ee
The lower limit corresponds to large angular deflection scales $\sim 1$~rad,
such as might be experienced by iron nuclei.  The upper limit corresponds
to small angular deflection scales $\sim 1^\circ$, such as might be
experienced by protons with $E\sim 100$~EeV.


%..............................................................................
\subsection{Cosmic ray detection and measurement}
\label{sec:dtxn}

Even though the arrival rate of UHECRs into a unit volume is constant in
time in our model, the expected number per unit time detected from a given
direction will vary as the rotation of the Earth changes the observatory's
projected area toward that direction, as noted above.  As a result, the
Poisson intensity function for detectable cosmic rays varies in
time for each source.

Recall that the likelihood function for an inhomogeneous Poisson
point process in time with rate (intensity function) $r(t)$ has the form
\be
\exp(-N_{\rm exp}) \prod_i r(t_i) \delta t,
\label{simple-ppp-like}
\ee
where the events are detected at times $t_i$ in detection intervals of size
$\delta t$, and $N_{\rm exp}$ is the total expected number in the observing
interval (the integral of the rate over the entire observing interval).
The likelihood function for the cosmic ray data has a similar form, but
with adjustments due to the mark distribution and measurement errors.

If the label and arrival direction for detected cosmic ray $i$ were known, the
factor in the likelihood function associated with that cosmic ray would be
$F_k \Aperp(\tdrxn_i, t_i) \delta t$, where $k=\lambda_i$.
In reality, both the label and the arrival direction are uncertain; the PAO
analysis pipeline produces a likelihood function for the direction to the
cosmic ray, $\ell_i(\tdrxn_i)$; see equation~(\ref{ell-def}).

Introducing the uncertain direction as a nuisance parameter, with a prior
denoted by $\rho_k(\tdrxn_i|\kappa)$, the likelihood factor for cosmic ray
$i$ when assigned to source $k$ may be calculated by marginalizing; it may
be written as $F_k f_{k,i}\delta t$, with
\be
f_{k,i}(\kappa) =
  \int d\omega_i \ell_i\left(\omega_i\right) \Aperp(\omega_i, t_i)
  \rho_k(\omega_i|\kappa).
\label{f-def}
\ee
The cosmic ray direction measurement uncertainty is relatively small ($\sim
1^\circ$) compared to the scale over which the area varies, so we can
approximate $f_{k,i}$ as
\ba
f_{k,i}(\kappa)
  \approx A_i\cos(\theta_i)
  \int  \ell_i(\omega_i)\rho_k(\omega_i|\kappa) d\omega_i,
\ea
where $\theta_i$ denotes the zenith angle of UHECR $i$ (reported by PAO-10)
and $A_i = A(t_i)$ is the area of the observatory at the arrival time of
UHECR $i$.  The integral can be computed analytically;
\ba
\int d\omega_i \ell_i(\omega_i) \rho_k(\omega_i|\kappa) =
\begin{cases}
\frac{\kappa_c\kappa}{4\pi\sinh(\kappa_c)\sinh(\kappa)}
  \frac{\sinh(|\kappa_c n_i+\kappa\hdrxn_k|)}{|\kappa_c n_i+\kappa\hdrxn_k|}
  & \mbox{if $k\geq 1$},\\
\frac{1}{4\pi} & \mbox{if $k=0$}.
\end{cases}
\label{f-approx}
\ea
The total event rate for cosmic rays with the properties (direction, energy,
and arrival time) of detected ray $i$ combines the contributions from each
potential source, i.e.,
$r(t_i) = \sum_k F_k f_{k,i}(\kappa)$.

To calculate $N_{\rm exp}$, we must account
for the observatory's exposure map.  The effective exposure given to cosmic
rays from source $k$ throughout the time of the survey depends, not just on
the direction to the source, but also on the deflection distribution,
$\rho_k$ (and thus on $\kappa$), since rays from that source will not arrive
precisely from the source direction.  The exposure factor for source $k$ is
\be
\epsilon_k(\kappa) =
  \int d\tdrxn \rho_k(\tdrxn|\kappa) \epsilon(\tdrxn).
\label{eps-def}
\ee
Note that $\epsilon_k$ has units of area $\times$ time, and for the
isotropic background component ($k=0$), $\epsilon_0(\kappa)$ is a constant
equal to the sky-averaged exposure (in the notation of
Supplementary Appendix~\ref{app:expo}, $\epsilon_0 = \alpha_T/4\pi$).
To find the total expected number of detected cosmic rays we sum over
sources: $N_{\rm exp} = \sum_{k \ge 0} F_k \epsilon_k(\kappa)$.

The prior probability mass function for the label of a {\em detected} cosmic
ray is not given by (\ref{label-pmf}); the terms must be weighted according to
the source exposures.  The result is
\be
P(\lambda_i=k|F_0,\Fvec,\kappa) =
  \frac{F_k\epsilon_k(\kappa)}{\sum_{j=0}^{N_A} F_j\epsilon_j(\kappa)}.
\label{label-eps-pmf}
\ee

We now have the ingredients needed to evaluate
equation~(\ref{simple-ppp-like}), generalized to include the cosmic ray
marks (directions and labels) and their uncertainties.  The resulting
likelihood function is
\be
\like(F_0,\Fvec,\kappa)
  = \exp\left(-\sum_k F_k\epsilon_k\right)
      \prod_i \left( \sum_k f_{k,i} F_k \right).
\label{like-poisson}
\ee
The product-of-sums factor resembles the likelihood for a finite mixture
model (FMM), if we identify the $f_{k,i}$ factors as the component
densities, and the $F_k$ factors as the mixing weights.  A common technique
for computing with mixture models is to rewrite the likelihood function as a
sum-of-products by introducing latent label parameters identifying which
component each datum may be assigned to (see, e.g., \cite{BG88-BayesFMM}).
Following this approach here, the likelihood function can be rewritten as a
sum over latent assignments of cosmic rays to sources,
\be
\like(F_0,\Fvec,\kappa)
  = \sum_{\lambda} \left(\prod_k F_k^{m_k(\lambda)}e^{-F_k\epsilon_k}\right)
    \prod_i f_{\lambda_i,i},
\label{like-assoc}
\ee
where $\lambda = \{\lambda_i\}$ and $\sum_\lambda$ denotes an
$N_C$-dimensional sum over all possible assignments of cosmic rays to
sources, and the multiplicity $m_k(\lambda)$ is the number of UHECRs
assigned to source $k$ according to $\lambda$.  We suppress the $\kappa$
dependence of $\epsilon_k(\kappa)$ and $f_{k,i}(\kappa)$ here and elsewhere
to simplify expressions.  Note that the $F_k$ dependence (for a given
$\lambda$) is of the same form as a gamma distribution.

Rewriting the previous expression with $(F_T,f)$ in place of $(F_0,F_A)$,
and using $F_k = w_kF_A = f w_k F_T$ (for $k\ge 1$), we can rewrite
$\like(F_0,\Fvec,\kappa)$ as
\ba
\label{eq:like}
\like(f,F_T,\kappa) 
  &=& \sum_\lambda (1-f)^{m_0(\lambda)} f^{N_C-m_0(\lambda)} F_T^{N_C}\\
  && \times e^{-F_T\left[(1-f)\epsilon_0+f\sum_{k\geq 1} w_k\epsilon_k \right]}
     \prod_{k\geq 1} w_k^{m_k(\lambda)} \prod_i f_{\lambda_i,i}.\nonumber
\ea

For computations it will be helpful to have the likelihood function
conditional on the label assignments,
\be
P(D|\lambda,F_0,\Fvec,\kappa)
  = \exp\left(-\sum_k F_k\epsilon_k\right)
    \left[ \sum_k F_k\epsilon_k \right]^{N_C}
    \prod_i \frac{f_{\lambda_i,i}}{\epsilon_{\lambda_i}}.
\label{eq:lik-lambda}
\ee
where $k$ runs over the host labels (from 0 to $N_A$), and $i$ runs over the
UHECR labels (from 1 to $N_C$).  We can recover the likelihood for $F_0$,
$\Fvec$, and $\kappa$ by multiplying by the prior for $\lambda$ from
equation~(\ref{label-eps-pmf}) and marginalizing, giving
equation~(\ref{like-assoc}).


%..............................................................................
\subsection{Estimating $\kappa$}

To estimate the deflection parameter, $\kappa$, we need the marginal
likelihood $\like_m(\kappa) = P(D|\kappa) = \int dF_T\int df
P(D,F_T,f|\kappa)$.  The integrand is the product of equation~(\ref{eq:like})
and the flux priors.  Using the exponential and beta priors described above,
we have that the marginal likelihood for $\kappa$ is
\ba
\like_m(\kappa)
  &=& \sum_\lambda \frac{\Gamma(N_C+1)\prod_{k\geq 1}w_k^{m_k(\lambda)}
      \prod_i f_{\lambda_i,i}}{s B(a,b)}\nonumber\\
  &&  \times\int_0^1
      \frac{f^{N_C-m_0(\lambda)+a-1}(1-f)^{m_0(\lambda)+b-1}}
           {\left[\frac{1}{s} + (1-f)\epsilon_o +
                f\sum_{k\geq 1}w_k\epsilon_k\right]^{N_C+1}}df.
\label{eq:marg}
\ea
Computing $\like_m(\kappa)$ requires summing over all possible values of
$\lambda$ which is intractable in practice.  In Supplementary
Appendix~\ref{sec:Chib}, we describe how to use Chib's method
\cite{MR1379473} to calculate this marginal likelihood.

%..............................................................................
\subsection{Model Comparison}

% $\like_m = \int d\kappa \pi(\kappa)\like_m(\kappa)$

We use Bayes factors (ratios of marginal likelihoods) to compare models both
conditioned on $\kappa$ (using marginal likelihood functions
$\like_m(\kappa)$), and after marginalizing over $\kappa$ (using the log-flat
prior of equation~(\ref{k-prior}), and numerical quadrature over
$\kappa$).

We consider three models.  The null model, $M_0$, assumes that all the
UHECRs come from the isotropic background source population; recall that it
has no $\kappa$ dependence (see equation~(\ref{rho-iso})).  Model $M_1$
allows the UHECRs to come from any of the 17 AGNs in the catalog or from the
isotropic background.  We also consider another model, $M_2$, in which the
UHECRs may come from the isotropic background or either of the two closest
AGNs, Cen~A (NGC~5128) and NGC~4945; this model is motivated in part by
recent suggestions that most UHECRs may be heavy nuclei from a single nearby
source, as cited above.  
(We also briefly explore a similarly-motivated fourth model that assigns all
UHECRs to Cen~A; as noted below, this model is tenable only for
$\kappa\approx 0$.)  In order to compare models $M_1$ and $M_2$ (conditioned
on $\kappa$) to the null model, we compute the Bayes factors:
\be
\mbox{BF}_{10}(\kappa) = \frac{\like_{m,1}(\kappa)}{\like_{m,0}},
\qquad
\mbox{ BF}_{20}(\kappa) = \frac{\like_{m,2}(\kappa)}{\like_{m,0}}
\label{BF10+20}
\ee
where $\like_{m,0}$ is the marginal likelihood for the null
model (similar equations hold for models
that marginalize over $\kappa$).  The value of $\like_{m,0}$
can be found from equation~(\ref{like-assoc}), noting that for the
null model, there is only one term in the sum over $\lambda$
(with all $\lambda_i = 0$, since the only allowed value of $k$ is $k=0$).
Marginalizing this term over $F_T$ (equal to $F_0$ in this case) gives
\be
\like_{m,0} =
  \frac{1}{s}
  \left(\frac{s}{s\epsilon_0+1}\right)^{N_C+1}
  \Gamma(N_C+1) \times \prod_i f_{0,i}.
\ee

%..............................................................................
\subsection{Computational techniques}
\label{subsec:compn}

The principal obstacle to computing with this framework is the combinatorial
explosion in the number of possible associations as the sizes of the
candidate source population and the cosmic ray sample grow.  For small
amounts of magnetic deflection, the vast majority of candidate associations
are improbable (they associate well-separated objects with each other). But
there is evidence that UHECRs may be massive (and thus highly charged)
nuclei, which would undergo significant deflection. To probe the full
variety of astrophysically interesting models requires techniques that can
handle both the small- and large-deflection regimes, for catalog sizes
corresponding to current and forthcoming catalogs from PAO.

For parameter estimation within a particular model, we have developed a
Markov chain Monte Carlo (MCMC) algorithm that draws samples of the
parameters $f$, $F_T$, and $\lambda$ from their joint posterior
distribution.  The algorithm takes advantage of two features of the models
described above.  First, by introducing latent labels, $\lambda$, we could
write the likelihood function in a sum-of-products form,
equation~(\ref{like-assoc}), with factors that depend on the fluxes
$F_k$ in the manner of a gamma distribution.  Second, the forms of the
likelihood and priors are conjugate for $F_T$ and the labels, so we can find
closed-form expressions for their conditional distributions.  These features
enable us to use Gibbs sampling techniques well known in mixture modeling
for sampling the $F_T$ and $\lambda$ parameters.  We handle the $f$
parameter using a random walk Metropolis algorithm, so our overall algorithm
is a Metropolis-within-Gibbs algorithm.  Supplementary
Appendix~\ref{app:compn} provides details on its implementation.

We treat the deflection parameter, $\kappa$, specially, considering
a logarithmically-spaced grid of values that we condition on.  We
did this so that we could explore the $\kappa$ dependence more
thoroughly than would be possible with posterior sampling of $\kappa$.
Of course, our Metropolis-within-Gibbs algorithm could be supplemented
with $\kappa$ proposals to enable sampling of the full posterior.

Finally, using Bayes factors to compare rival models requires computing
marginal likelihoods, which are not direct outputs of MCMC algorithms.
Using a simplified version of our model, and modest-sized simulated
datasets, we explored several approaches for marginal likelihood computation
in a regime where we could compute the correct result via direct summation
over all feasible associations.  We explored the harmonic mean estimator
(HME), Chib's method, and importance sampling algorithms.  The HME
performed poorly, often apparently converging to an incorrect result (such
behavior is not unexpected; see \cite{WS12-HMBad}).  Importance sampling
proved inefficient.  Chib's method was both accurate and efficient in these
trial calculations, and became our choice for the final implementation.
Supplementary Appendix~\ref{app:compn} provides details.
