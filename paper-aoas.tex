\documentclass[dvips,aoas,preprint]{imsart}
%\documentclass[dvips,aoas]{imsart}

\RequirePackage[OT1]{fontenc}
\RequirePackage{amsthm,amsmath}
\RequirePackage[numbers]{natbib}
\RequirePackage[colorlinks,citecolor=blue,urlcolor=blue]{hyperref}

\usepackage{amssymb,amsbsy}
\usepackage{mathtools}  % for \coloneqq
%\usepackage[dvips]{graphics}
\usepackage{graphicx}
\usepackage{rotating}  % for sidewaystable

% Packages that aren't standard for AOAS:
%\usepackage{subfig} 

% Watermarking:
%\usepackage{draftwatermark}
%\SetWatermarkText{DRAFT}
%\SetWatermarkLightness{0.9}
%\SetWatermarkScale{5}  % default is 1.2

% To typeset latex command names:
\newcommand\cmdname[1]{{\tt \textbackslash#1}}
\chardef\bslash=`\_

%\usepackage{amsthm,amsmath,natbib}
%\RequirePackage[colorlinks,citecolor=blue,urlcolor=blue]{hyperref}

% provide arXiv number if available:
%\arxiv{math.PR/0000000}

%===============================================================================
% put your definitions there:
\startlocaldefs

% Macro for an "editorial note"; uncomment the 2nd definition to
% produce a version without notes (put it in main doc if desired).
\newcommand\enote[1]{{$\bullet\bullet\bullet$}{\sl [#1]}{$\bullet\bullet\bullet$}}
%\renewcommand\enote[1]{\relax}

% Marginal note (just a redef of \marginpar or \marginparodd).
% Uncomment the 2nd definition to produce a version without notes (put it
% in main doc if desired).
% Could use \footnotesize for a bigger font.
% The \hspace enables hyphenation of the 1st word.
\setlength{\marginparwidth}{.85in}
\newcommand\mnote[1]{\-\marginpar[\raggedleft\scriptsize\hspace*{0pt}#1]%
{\raggedright\scriptsize\hspace*{0pt}#1}}
%\renewcommand\mnote[1]{\relax}

\newcommand{\txw}{\textwidth}

\renewcommand{\arraystretch}{1.2}


\newcommand{\bm}[1]{\mbox{\boldmath{$#1$}}}

\newcommand{\be}{\begin{equation}}
\newcommand{\ee}{\end{equation}}
\newcommand{\ba}{\begin{eqnarray}}
\newcommand{\ea}{\end{eqnarray}}
\newcommand{\barr}{\begin{array}}
\newcommand{\earr}{\end{array}}
\newcommand{\bc}{\begin{center}}
\newcommand{\ec}{\end{center}}

\newcommand{\like}{{\cal L}}
\newcommand{\pval}{$p$-value}
\newcommand{\Eth}{E_{\rm th}}
\newcommand{\Dmax}{D_{\rm max}}
\newcommand{\Nsrc}{N_S}

% Coincidence model macros; t* = target, h* = host:
\newcommand{\tdrxn}{\omega}  % target (CR) direction
\newcommand{\hdrxn}{\varpi}  % host (AGN) direction
\newcommand{\expo}{\epsilon}  % exposure
\newcommand{\tlabel}{\lambda}  % target label
\newcommand{\Fvec}{\bm{F}}  % host fluxes
\newcommand{\Aperp}{A_\perp}  % perp. area

\newcommand{\D}{\overline{\mbox{D}}}

\newcommand{\simless}{\mathbin{\lower 3pt\hbox
   {$\rlap{\raise 5pt\hbox{$\char'074$}}\mathchar"7218$}}}
\newcommand{\simgreat}{\mathbin{\lower 3pt\hbox
   {$\rlap{\raise 5pt\hbox{$\char'076$}}\mathchar"7218$}}}
\newcommand{\gta}{\simgreat}
\newcommand{\lta}{\simless}
\newcommand\msun{{\rm M}_\odot}
\endlocaldefs

%===============================================================================
\begin{document}

%===============================================================================
\begin{frontmatter}

% "Title of the paper"
\title{Multilevel Bayesian Framework for Modeling the Production,
Propagation and Detection of Ultra-High Energy Cosmic Rays}

% For footnote indicating draft status:
% END OF TITLE TEXT\protect\thanksref{DRAFT}}
%\thankstext{DRAFT}{This is an description of work in progress, prepared for
%participants in the 2011 workshop, {\em Case Studies in Bayesian Statistics and
%Machine Learning}.  {\bf PLEASE DO NOT CIRCULATE!}}

% indicate corresponding author with \corref{}
% \author{\fnms{John} \snm{Smith}\corref{}\ead[label=e1]{smith@foo.com}\thanksref{t1}}
% \thankstext{t1}{Thanks to somebody}
% \address{line 1\\ line 2\\ printead{e1}}
% \affiliation{Some University}

\author{\fnms{Kunlaya} \snm{Soiaporn}\corref{}\ead[label=e1]{ks354@cornell.edu}},
\address{\printead{e1}}
\author{\fnms{David} \snm{Chernoff}\ead[label=e3]{chernoff@astro.cornell.edu}},
\address{\printead{e3}}
\author{\fnms{Thomas} \snm{Loredo}\ead[label=e2]{loredo@astro.cornell.edu}},
\address{\printead{e2}}
\author{\fnms{David} \snm{Ruppert}\ead[label=e4]{dr24@cornell.edu}}
\address{\printead{e4}}
\and
\author{\fnms{Ira} \snm{Wasserman}\ead[label=e5]{ira@astro.cornell.edu}}
\address{\printead{e5}}
\affiliation{Cornell University}


\runauthor{Soiaporn, et al.}
\runtitle{Multilevel Models for Cosmic Rays}

\begin{abstract}
Cosmic rays are atomic nuclei with velocities close to the speed of light.
Ultra-high energy cosmic rays (UHECRs) are approximately ten million times
more energetic than the most extreme particles produced at the Large Hadron
Collider and likely originate from nearby extragalactic sources. Important
astrophysical questions include: what phenomenon accelerates particles to such
large energies, which astronomical objects host the accelerators, and what
sorts of nuclei end up being energized? We develop a multilevel Bayesian
framework for assessing evidence for association of UHECRs and candidate source
populations that has four levels:
(1) a source property description level, including specification of a
candidate source population and the distribution of cosmic ray
intensities among sources;
(2) a cosmic ray production level, using a marked Poisson point process
to model latent properties (arrival times and energies) of cosmic rays;
(3) a propagation level, modeling deflection of cosmic ray trajectories by
cosmic magnetic fields;
(4) a detection and measurement level, accounting for detector efficiency
and exposure, and measurement errors in arrival directions and energies.
Our framework can flexibly accommodate astrophysical components anywhere on
a spectrum from simplicity to great complexity and realism.
The number of possible associations is huge even for modest-sized datasets. 
We describe a Markov chain Monte Carlo algorithm for estimating model
parameters and comparing models by computing, via Chib's method, marginal
likelihoods and Bayes factors.

We demonstrate the framework by using simple model components to analyze
observations of 69 UHECRs by the Pierre Auger Observatory (PAO) during the
years 2004--2009, using a volume-complete catalog of 17 nearby active
galactic nuclei as a candidate host population.  The reported data are
incomplete; an early portion of the data (``Period~1,'' with 14 events) was
used to set an energy cut maximizing a measure of anisotropy in that data;
only data for cosmic rays with energies above that cut are reported.  Also,
measurement errors are approximately summarized.  These factors are
problematic for independent analyses of PAO data.  Within the context of
``standard candle'' source models (all sources with the same isotropic
cosmic ray emission rate), and considering only the untuned data subsequent
to Period~1, there is no significant evidence favoring association of UHECRs
with nearby active galactic nuclei (AGN) vs.\ attributing them to sources
drawn from an isotropic background distribution.  The highest-probability
associations are with the two nearest AGN, Centaurus~A and its neighbor,
NGC~4945.  If the association model is adopted, the fraction of UHECRs that
may be associated is likely nonzero but is constrained to be well below 50\%. 
Relatively modest magnetic deflection angular scales of $\approx 3^\circ$ to
$30^\circ$ are favored. Models that assign a large fraction of UHECRs to a
single nearby source (e.g., Centaurus~A) are ruled out unless very large
deflection scales are specified a priori, and even then they are disfavored. 
However, including the Period~1 data alters the conclusions significantly, and
a simulation study supports the idea that the Period~1 data are anomalous,
presumably due to the tuning.  Accurate and optimal analysis of future data
will likely require more complete disclosure of the data.
\end{abstract}


%\begin{keyword}[class=AMS]
%\kwd[Primary ]{}
%\kwd{}
%\kwd[; secondary ]{}
%\end{keyword}

\begin{keyword}
\kwd{Multilevel modeling}
\kwd{Hierarchical Bayes}
\kwd{Astrostatistics}
\kwd{Cosmic rays}
\kwd{Directional data}
\kwd{Coincidence assessment}
\kwd{Bayes factors}
\end{keyword}

\end{frontmatter}

%===============================================================================

%\enote{Editorial notes appear like this, using the \cmdname{enote} macro
%defined in paper.tex.  For issues not resolved but being deferred, leave the
%\cmdname{enote} in place, but uncomment the \cmdname{renewcommand}
%redefinition of \cmdname{enote} in paper.tex to hide them; we'll deal with
%them later.}


\section{Introduction}

Cosmic ray particles are naturally produced, positively charged atomic nuclei
arriving from outer space with velocities close to the speed of light.  
The origin of cosmic rays is not well-understood. The Lorentz force
experienced by a charged particle in a magnetic field alters its trajectory.
Simple estimates imply that cosmic rays with energy $E \lta 10^{15}$~eV 
have trajectories so strongly bent
by the Galactic magnetic field that they are largely trapped within the
Galaxy.\footnote{We follow the standard astronomical convention of using
``Galaxy'' and ``Galactic'' (capitalized) to refer to the Milky Way galaxy.}
The acceleration sites and the source populations are not definitively known
but probably include supernovae, pulsars, stars with strong winds, and
stellar-mass black holes.  For recent reviews, see \cite{Cronin99,Hillas06}.
More mysterious, however, are the highest energy cosmic rays.

By 1991, large arrays of cosmic ray detectors had seen a few events with
energies $\sim 100$~EeV (where 1~EeV $=10^{18}$~eV). In the 1990's the Akeno
Giant Air Shower Array (AGASA; \cite{1992APh.....1...27C}) and the High
Resolution Fly's Eye (HiRes; \cite{2002NIMPA.482..457B}) were built to
target these ultra-high energy cosmic rays (UHECRs); each detected a few
dozen cosmic rays with $E>10$~EeV.  For recent reviews, see
\cite{KO11-UHECRs,LS11-UHECRs}.
Detectable UHECRs likely emanate from relatively nearby extragalactic
sources. On the one hand their trajectories are only weakly deflected by
galactic magnetic fields so they are unconfined to the galaxy from which
they originate.  On the other hand, they are unlikely to reach us from
distant (and thus isotropically distributed) cosmological sources.  Cosmic
rays with energies above the Greisen-Zatsepin-Kuzmin (GZK) limit of $\sim
50$~EeV  should scatter off of cosmic microwave background photons, losing
some of their energy to pion production with each interaction
\cite{G66-GZK,ZK66-GZK}.  Thus the universe is not transparent to UHECRs;
they are not expected to travel more than about 50 to 100 megaparsecs (Mpc)
before their energies fall below the GZK limit. Notably, over this distance
scale there is significant anisotropy in the distribution of matter that
should be reflected in the arrival directions of UHECRs. Astronomers hope
that continued study of the directions and energies of UHECRs will address
the fundamental questions of the field: What phenomenon accelerates
particles to such large energies?  Which astronomical objects host the
accelerators?  What sorts of nuclei end up being energized?  In addition,
UHECRs probe galactic and intergalactic magnetic fields.

The flux of UHECRs is very small, approximately 1 per square kilometer per
century for energies $E\gta 50$ EeV.  Large detectors are needed to find
these elusive objects; the largest and most sensitive
detector to date is the Pierre Auger Observatory (PAO) \cite{PAO04-Proto}
in Argentina.  The
observatory uses air fluorescence telescopes and water
Cerenkov surface detectors to observe the air shower generated when a cosmic
ray interacts with nuclei in the upper atmosphere over the
observatory.  The surface detectors (SDs) operate continuously, detecting
energetic subatomic particles produced in the air shower and reaching the
ground.  The fluorescence detectors (FDs) image light from the air shower and
supplement the surface detector data for events detected on clear, dark
nights.\footnote{The FD on-time is about 13\% \cite{PAO10-GZK}, but analysis
can reveal complications preventing use of the data---e.g., obscuration due to
light cloud cover, or showers with significant development underground---so
fewer than 13\% of events have usable FD data.  These few so-called {\em
hybrid} events are important for calibrating energy measurements and provide
information about cosmic ray composition vs.\ energy.}
PAO began taking data in 2004 during construction; by 
June 2008 the PAO array comprised $\approx 1600$ SDs
covering $\approx 3000$ km$^2$, surrounded by four
fluorescence telescope stations (with six telescopes in each station)
observing the atmosphere over the array.

By 31 August 2007, PAO had detected 81 UHECRs with $E > 40$~EeV
(see \cite{PAO07-Aniso}, hereafter PAO-07), finding clear evidence of
an energy cutoff resembling the predicted GZK cutoff, i.e., a sharp drop in
the energy spectrum above $\approx 100$~EeV and a discernable pile-up of
events at energies below that \cite{PAO10-GZK}. This supports the idea that
the UHECRs must originate in the nearby universe.

The PAO team searched for correlations between the cosmic ray arrival
directions and the directions to nearby active galactic nuclei (AGN)
(initial results were reported in PAO-07; further details and a catalog of
the events are in \cite{PAO08-AGN}, hereafter PAO-08). AGN are unusually
bright cores of galaxies; there is strong (but indirect) evidence that they
contain rapidly mass-accreting supermassive black holes that eject some
material in energetic, jet-like outflows.  AGN are theoretically favored
sites for producing UHECRs; electromagnetic observations indicate particles
are accelerated to high energies near AGN.  The PAO team's analysis was
based on a significance test that counted the number of UHECRs with best-fit
directions within a critical angle, $\psi$, of an AGN in a catalog of local
AGN (more details about the catalog appear below); the number was compared
with what would be expected from an isotropic UHECR directional distribution
using a \pval.  A simple sequential approach was adopted.  The earliest half
of the data was used to tune three parameters defining the test statistic by
minimizing the \pval.  The parameters were:  $\psi$; a maximum distance,
$\Dmax$, for possible hosts; and a minimum energy, $\Eth$, for UHECRs
considered to be associated with AGN.  With these parameters tuned
($\Eth=56$~EeV, $\psi=3.1^\circ$, $\Dmax=75$~Mpc), the test was applied to
the later half of the data; 13 UHECRs in that period had $E>\Eth$.  The
resulting \pval\ of $1.7\times 10^{-3}$ was taken as indicating the data
reject the hypothesis of isotropic arrival directions ``with at least a 99\%
confidence level.''  The PAO team was careful to note that this result did
not necessarily imply that UHECRs were associated with the cataloged AGN,
but rather that they were likely to be associated with some nearby
extragalactic population with similar anisotropy.

Along with these results, the PAO team published a catalog of energy and
direction estimates for the 27 UHECRs satisfying the $E>\Eth$ criterion,
including both the earliest 14 events used to define $\Eth$, and the 13
subsequent events used to obtain the reported \pval\ (the PAO data are
proprietary; measurements of the other 54 events used in the analysis were
not published).  Their statistical result
spurred subsequent analyses of these early published PAO UHECR
arrival directions, adopting different methods and aiming to make more
specific claims about the hosts of the UHECRs.  Roughly speaking, these
analyses found similarly suggestive evidence for anisotropy, but no
conclusive evidence for any specific association hypothesis.

% If needed, cite studies cited by WMJ11.

In late 2010, the PAO team published a revised catalog, including new data
collected through 2009 (\cite{PAO10-AnisoUpdate}; hereafter PAO-10).  An
improved analysis pipeline revised the energies of earlier events downward
by 1~EeV; accordingly, the team adopted $\Eth = 55$~EeV on the new energy
scale.  The new catalog includes measurements of 42 additional UHECRs (with
$E>\Eth$) detected from 1~September 2009 through 31~December 2010.  A repeat
of the previous analysis (adding the new events but again excluding the
early tuning events) produced a larger \pval\ of $3\times 10^{-3}$, i.e.,
{\em weaker} evidence against the isotropic hypothesis.  The team performed
a number of other analyses (including considering new candidate host
populations).  Despite the growth of the post-tuning sample size from 14 to
55, they found evidence for anisotropy weakened.  Time-resolved measures of
anisotropy provided puzzling indications that later data might have
different directional properties than early data, although the sample size
is too small to demonstrate this conclusively.

Here we describe a new framework for modeling UHECR data based on Bayesian
multilevel modeling of cosmic ray emission, propagation, and detection.
A virtue of this approach is that physical and experimental
processes have explicit representations in the framework, facilitating
exploration of various scientific hypotheses, and physical interpretation of
the results.  This is in contrast to
hypothesis testing approaches where elements such as angular and energy
thresholds only implicitly represent underlying physics, and potentially
conflate astrophysical and experimental effects (e.g., magnetic scattering
of trajectories, and measurement errors in direction).
Our framework can
handle a priori uncertainty in model parameters via marginalization. 
Marginalization also accounts for the uncertainty in such parameters via
weighted averaging, rather than fixing them at precise, tuned values.  This
eliminates the need to tune energy, angle, and distance scales with a subset
of the data that must then be excluded from a final analysis.  Such parameters
are allowed to adapt to the data, but the ``Ockham's razor'' effect associated
with marginalization penalizes models for fine-tuned degrees of freedom,
thereby accounting for the adaptation.

In this paper we describe our general framework, computational algorithms
for its implementation, and results from analyses based on a few
representative models.  Our models are somewhat simplistic astrophysically,
although similar to models adopted in previous studies.  We do not aim to
reach final conclusions about the sources of UHECRs; the focus here is on
developing new methodology and demonstrating the capabilities of the
approach in the context of simple models.

An important finding is that {\em thorough and accurate independent analysis
of the PAO data likely requires more data than has so far been publicly
released} by the PAO collaboration.  In particular, although our Bayesian
approach eliminates the need for tuning, in the absence of publicly available
``untuned'' data (i.e., measurements of lower-energy cosmic rays), we cannot
completely eliminate the effects of tuning from analyses of the
published data (Bayesian or otherwise).  Additionally, a Bayesian analysis can
(and should) use event-by-event (i.e., heteroskedastic) measurement
uncertainties, but these are not publicly available.  Finally, astrophysically
plausible conclusions about the sources of UHECRs will require models more
sophisticated than those we explore here (and those explored in other recent
studies).


\section{Description of cosmic ray and candidate host data}

%..............................................................................
\subsection{Cosmic ray data}
\label{sec:data}

The reported PAO measurements depend not only on the intrinsic particle
population but also on many experimental and algorithmic choices in the
detection and analysis chain, many of them associated with the need to
distinguish between events of interest and background events from uninteresting
but uncontrollable sources (e.g., natural radioactivity).  UHECRs can
impinge on the observatory at any time, from any direction and with any
energy.
However, virtually no background sources
produce events with properties mimicking those of very high energy cosmic rays
arriving from directions well above the horizon.
Cosmic rays with $E>3$~EeV arriving from any direction lying within a large
window on the sky create air showers detected with nearly 100\% efficiency
(no false positives, no false dismissals).  The SDs and FDs measure
the spatio-temporal development of the air shower which allows
the energy and arrival direction to be measured.  The uncertainties depend
upon how many counters of each type are triggered plus the systematic and
statistical uncertainties implicit in modeling  the development of the
air shower. The PAO team reports energy and
arrival direction estimates for each cosmic ray falling within the geometric
bounds of its zone of secure detection.\footnote{The directional criterion
adopted for the PAO catalogs is that an event is reported if it best-fit
arrival direction is within $60^\circ$ of the observatory's zenith, the
local normal to Earth's surface at the time of the event.}

We consider the $N_C = 69$ UHECRs with energies $E \geq \Eth = 55$~EeV
cataloged in PAO-10, which reports measurements of all UHECRs seen by PAO
through 31 December 2009 with $E \geq \Eth$, based on analysis of the
surface detector data only.  Although our framework does not tune an event
selection criterion, for interpreting the results it is important to
remember that the $\Eth=55$~EeV threshold value was set to maximize a
signature of anisotropy in an early subset of the data.
The tuning data included the 14 earliest
reported events, detected from 1~January 2004 to 26~May 2006 (inclusive;
Period~1), as well as numerous unreported events with $E<\Eth$.  The first
published catalog in PAO-08 included 13 subsequent UHECRs observed through 31
August 2007 (Period~2).  The PAO-10 catalog includes 42 additional UHECRs
observed through 31 December 2009 (Period~3). Table~1 in PAO-10 provides
information about the three periods, including the sky exposure for each
period, which is not simply proportional to duration (the observatory grew in
size considerably through 2008).  Data for cosmic rays with $E < \Eth$ are not
publicly available.\footnote{The PAO web site hosts public data for 1\% of
lower-energy cosmic rays, but the sample is not statistically characterized
and UHECRs are not included.}

The direction estimate for a particular cosmic ray is the result of a
complicated analysis of time series data from the array of PAO surface
detectors.\footnote{The SD data may be supplemented by data from the
fluorescence detectors for hybrid events observed under favorable
conditions, but there are very few such events at ultra-high energies,
and PAO-10 reports analysis of SD data only.}
Roughly speaking, the direction is inferred by triangulation. 
The analysis produces a likelihood function for the cosmic ray arrival
direction, $\omega$ (a unit vector on the celestial sphere).  The shapes of
the likelihood contours are not simple, but they are roughly azimuthally
symmetric about the best-fit direction.  The PAO-10 catalog summarizes the
likelihood function with a best-fit direction, and a typical directional
uncertainty of $\approx 0.9^\circ$ corresponding to the angular radius of an
azimuthally symmetric 68.3\% confidence region.  We use these summaries to
approximate the likelihood functions with a Fisher distribution with mode at
the best-fit direction for each cosmic ray, and with concentration parameter
$\kappa_c = 9323$, corresponding to a 68.3\% confidence region with an
angular radius of 0.9$^\circ$.  Let $d_i$ denote the data associated with
cosmic ray $i$, and $\omega_i$ denote its actual arrival direction.  The
likelihood function for the direction is
\ba
\ell(\omega_i)
  \coloneqq P(d_i|\omega_i)
   \approx \frac{\kappa_c}{4\pi\sinh(\kappa_c)} \exp(\kappa_c n_i\cdot\omega_i),
\label{ell-def}
\ea
where $n_i$ denotes the best-fit direction for cosmic ray $i$, and we have
scaled the likelihood function so its integral over $\omega_i$ is unity,
merely as a convenient convention.  Bonifazi et al.\
\cite{B+PAO09-DrxnUncert} provide more information about the PAO direction
measurement capability.  Note that the expected angular scale of magnetic
deflection is larger than the PAO directional uncertainties, significantly
so if UHECRs are heavy nuclei (see \S~\ref{sec:dflxn}).

Similarly, the analysis pipeline produces energy estimates for each event.
These estimates have significant random and systematic uncertainties
\cite{PAO08-GZK,PAO10-GZK}.
The models we study here do not make use of the reported energies and are
unaffected by these uncertainties.  But our framework readily generalizes to
account for energy dependence.  In principle it is straightforward to account
for the random uncertainties, but a consistent treatment requires data for
events below any imposed threshold:  the true energies of events with best-fit
energies below threshold could be above threshold (and vice versa for those
with best-fit energies above threshold); accounting for this requires data to
energies below astrophysically important thresholds.  The systematic
uncertainties become important for joint analyses of PAO data with data from
other experiments, and for linking results of spectral analyses to particle
physics theory.


%..............................................................................
\subsection{Candidate source catalog}

As candidate sources for the PAO UHECRs, the analysis reported in PAO-07 and
PAO-08, and several subsequent analyses, considered 694 
AGN within $\approx 75$~Mpc from the 12th catalog 
assembled by V\'eron-Cetty and V\'eron \cite{VCV-12thAGNCat} (VCV). 
This catalog includes data on all AGN and quasars (AGN with star-like
images) with published spectroscopic redshifts; it includes observations
from numerous investigators using diverse equipment and AGN selection
methods, and does not represent a statistically well-characterized sample of
AGN.\footnote{VCV say of the catalog, ``This catalogue should not be used
for any statistical analysis as it is not complete in any sense, except that
it is, we hope, a complete survey of the literature.''}
Subsequent analyses in PAO-10, and a few other analyses, used more recent
catalogs of active galaxies or normal galaxies, including flux-limited
catalogs (i.e., well-characterized catalogs that contain all bright sources
within a specified volume, but dimmer sources only in progressively smaller
volumes).

%As noted above, the Greisen-Zatsepin-Kuzmin (GZK) limit implies that cosmic
%rays with energies $\gtrsim$ 50~Eev should interact with cosmic microwave
%background photons and should almost never reach the earth from distance
%longer than 50~Mpc.

For the representative analyses reported here, we consider the 17 AGN
cataloged by Goulding et al.\cite{2010MNRAS.406..597G} (2010; hereafter G10)
as candidate sources.  This is a well-characterized {\em volume}-limited
sample; it includes all infrared-bright AGN within 15~Mpc.  
For each AGN in the calalog, we take its position on the sky, $\varpi_k$
($k=1$ to $\Nsrc$), and its distance, $D_k$, to be known precisely (galaxy
directions have negligible uncertainties compared to cosmic ray directions).
Notably, this catalog includes Centaurus~A (Cen~A), the nearest AGN
($D\approx 4.0$~Mpc), an unusually active and morphologically peculiar AGN.
Theorists have hypothesized Cen~A to be a source of many or even most UHECRs
if UHECRs are heavy nuclei, which would be deflected through large angles;
see \cite{B+09-CenA,GBdS10-CenA,BdS12-CenA}.  The small size of this catalog
facilitates thorough exploration of our methodology: Markov chain Monte
Carlo algorithms can be validated against more straightforward algorithms
that could not be deployed on large catalogs, and simulation studies are
feasible that would be too computationally expensive with large catalogs.
Also, for simple ``standard candle'' models (adopted here and in other
studies), that assign all sources the same cosmic ray intensity, little is
gained by considering large catalogs, because assigning detectable cosmic
ray intensities to distant sources would imply cosmic ray fluxes from nearby
sources too large to be compatible with the data.

We also include an isotropic background component as a ``zeroth" source.
This allows a model to assign some UHECRs to sources not included in the AGN
catalog (either galaxies not cataloged, or other, unobserved sources). In
addition, we consider an isotropic source distribution for {\em all} cosmic
rays (i.e., a model with only the zeroth source) as a ``null'' model for
comparison with models that associate some cosmic rays with AGN or other
discrete sources.  An isotropic distribution is convenient for calculations
and has been adopted as a null hypothesis in several previous studies. 
Historically, before PAO's convincing observation of a GZK-like cutoff in
the UHECR energy spectrum, the isotropic distribution was meant to represent
a distant cosmological origin for UHECRs.  Accepting the null would indicate
that the GZK prediction was incorrect, and that changes in fundamental
physics would be required to explain UHECRs.  In light of PAO's compelling
observation of a GZK-like cutoff (with its implied $\sim 100$~Mpc distance
scale), interpreting an isotropic null or background component is
problematical if there are many light nuclei among the UHECRs.  We adopt it
here both for convenience and due to precedent.  We discuss this further
below.

%..............................................................................
\subsection{Sky map}

Figure~\ref{fig:skymap} shows a sky map displaying the
directions to both the UHECRs seen by PAO, and the AGN in the G10 catalog.
The directions are shown in an equal-area Hammer-Aitoff projection in
Galactic coordinates; the Galactic plane is the equator (Galactic latitude
$b=0^\circ$), and the vertically-oriented grid lines are meridians of
constant Galactic longitude, $l$.  The star indicates the south celestial
pole (SCP), the direction directly above Earth's south pole (effects like
precession and nutation of the Earth's axis are negligible for this
application and we ignore them in this description).  The thick gray line
bounds the PAO field of view.  The UHECR and AGN directions are displayed as
``tissots,'' projections of circular patches centered on the reported
directions.  The small tissots show the UHECR directions; the tissot size is
$2^\circ$, corresponding to $\approx 2$ standard deviation errors, and the
tissot color indicates energy.  The large green tissots indicate AGN
directions; the tissot size is $5^\circ$, corresponding to a plausible scale
for magnetic deflection of UHE protons in the Galactic magnetic
field.\footnote{If UHECRs are comprised of heavier, more positively charged
nuclei, they could suffer much larger deflections; see \S~\ref{sec:dflxn}.}
The tissots are rendered with transparency; the two darker tissots near the
Galactic north pole indicate pairs of AGN with nearly coincident directions.
Two of the AGN tissots are outlined in solid black; these correspond to the
two nearest AGN, Centaurus~A (Cen~A, also known as NGC~5128) and NGC~4945,
neighboring AGN at distances of 4.0 and 3.9~Mpc (as reported in G10).  Five
others are outlined in dashed black; these have distances ranging from 6.6
to 10.0~Mpc. The remaining 10 AGN have distances from 11.5 to 15.0~Mpc. 
Four of the AGN are outside the PAO field of view, but depending on the
scale of magnetic deflection, they could be sources of observable cosmic
rays.


\begin{figure}
\begin{centering}
\includegraphics[width=\textwidth]{CR+LocalAGN-all.eps}
\end{centering}
\caption{Sky map showing directions to 69 UHECRs detected by PAO, and
to 17 nearby AGN from the catalog of Goulding et al..  Directions are
shown in an equal-area Hammer-Aitoff projection in Galactic coordinates.
Thick gray line indicates the boundary of the PAO field of view.
Small tissots show UHECR directions; tissot radius is $2^\circ$ corresponding
to $\approx 2$ standard deviation errors; tissot color indicates energy.
Large green tissots indicate AGN directions; tissot radius is $5^\circ$.
Thin curves are geodesics connecting each UHECR to its nearest AGN.}
\label{fig:skymap}
\end{figure}

Figure~\ref{fig:skymap} shows the measured directions for the 69 UHECRs.
The thin curves (teal) show
geodesics connecting each UHECR to its nearest AGN.
There is a noticable concentration of cosmic ray directions near the
directions of Cen~A and NGC~4945; a few other AGN also have conspicuously
close cosmic rays.  We have also examined similar maps for the subsets
of the UHECRs in the three periods.
The concentration in the vicinity of the two closest AGN
is also evident in the maps for periods~1 and 2.
Curiously, except for a single UHECR about $6^\circ$
from NGC~4945, no such concentration is evident in the map for Period~3,
despite it having about three times the number of UHECRs found in earlier
periods.  This is a presage of results from our quantitative analysis that
suggest the data may not be consistent with simple models for the cosmic ray
directions, with or without AGN associations.

%..............................................................................
\subsection{PAO exposure}
\label{sec:expo}

PAO is not equally sensitive to cosmic rays coming from all directions.
Quantitative assessment of evidence for associations or other anisotropy must
account for the observatory's direction-dependent exposure.

Let $F$ be the cosmic ray flux at Earth from a source at a given direction,
$\tdrxn$, i.e., the expected number of cosmic rays per unit time per unit
area normal to $\tdrxn$.   Then the expected number of rays detected in a
short time interval $dt$ is $F \Aperp(t,\tdrxn) dt$, where $\Aperp(t,\tdrxn)$ is
the projected area of the observatory toward $\tdrxn$ at time $t$. The total
expected number of cosmic rays is given by integrating over $t$; it can be
written as $F\expo(\tdrxn)$, with the {\em exposure map} $\expo(\tdrxn)$
defined by
\begin{equation}
\expo(\tdrxn) \coloneqq \int_T  \Aperp(\tdrxn,t) dt;
\label{expo-def-main}
\end{equation}
the integral is over the time intervals when the observatory was operating,
denoted collectively by $T$.
The \ref{supp} \cite{S+12-UHECR-Supp} describes calculation of
$\expo(\tdrxn)$; the thick gray curves shown in the sky maps mark the
boundary of the region of nonzero exposure.
  % omits extra skymaps

\section{Modeling the cosmic ray data}

% We adopt a Bayesian
% approach for directional coincidence assessment based on multilevel modeling,
% where upper levels in a model describe properties of potentially associated
% source populations and radiation propagation, and lower levels describe
% measurement errors and survey selection effects.

The basic statistical problem is to quantify evidence for associating some
number (possibly zero) of cosmic rays with each member of a candidate
source population.  The key observable is the cosmic ray direction; a set of
rays with directions near a putative host comprises a multiplet potentially
associated with that host.  This gives the problem the flavor of model-based
clustering (of points on the celestial sphere rather than in a Euclidean space),
but with some novel features:
\begin{itemize}
\item The model must account for measurement error in cosmic ray properties.

\item Observatories provide an incomplete and distorted sample of cosmic rays,
so the model must account for random truncation and nonuniform thinning.

\item The most realistic astrophysical models imply a joint distribution for
the properties of the cosmic rays assigned to a particular source that is
exchangeable rather than a product of independent distributions (as is the
case in standard clustering).

\item The number of cosmic rays is informative about the intensity scale of
the cosmic ray sources so the binomial point process model underlying
standard generative clustering approaches is not appropriate.
\end{itemize}

%1: Sources as a population
%2: Production per source
%3: Propagation
%4: Measurement

To account for these and other complexities, we model the data using a
hierarchical Bayesian framework with four levels:
\begin{enumerate}
\item {\em Source properties}:  At the top level we specify the properties
of the sources of cosmic rays.  This may include the choice of a candidate
source population of identified objects (e.g., a particular galaxy
population), and/or specification of the properties of a population of
unidentified sources.  For a given candidate source population, we must
specify source directions and cosmic ray intensities.  The simplest case is
a standard candle model, with each source having the same cosmic ray
intensity.  More generally, we may specify a (non-degenerate) distribution
of source intensities; this corresponds to specifying a ``luminosity
function'' in other astronomical contexts.  For a population of unidentified
sources, we must specify a directional distribution (isotropic in the
simplest case) as well as an intensity distribution.
\item {\em Cosmic ray production}:  We model the production of cosmic rays
from each source with a marked Poisson point process model for latent cosmic
ray properties.  The incident cosmic ray arrival times have a homogeneous
intensity measure in time, and the marks include the cosmic ray energies,
and latent categorical labels identifying the source of each ray.
\item {\em Cosmic ray propagation}:  Next we model magnetic deflection of
the rays, scattering their directions from the source directions.  
This requires introducing a latent arrival direction parameter for each ray.
Here we adopt a simple phenomenological model with a single parameter
specifying a typical scattering scale between the source and arrival
directions.  As the data become more abundant and detailed, the framework
can accommodate more complex models, e.g., with parameters explicitly
describing cosmic magnetic fields and cosmic ray composition.
\item {\em Detection and measurement}:  Last, we model detection and
measurement, accounting for truncation and thinning of the incident
cosmic ray flux, and measurement errors for directions and energies.
\end{enumerate}

Figure~\ref{fig:levels} schematically depicts the structure of our
framework, including identification of the various random variables
appearing in the calculations described below.  The variables will be
defined as they appear in the detailed development below; the figure serves
as visual reference to the notation.  The figure is not a graphical model
per se.  Rather, our models specify probability distributions over a space
of graphs, each graph corresponding to a possible set of associations of the
cosmic rays with particular sources.  This framework builds directly on an
earlier multilevel Bayesian model we developed to assess evidence that some
sources of gamma-ray bursts repeat \cite{LLW96}; this model, too, worked in
terms of probability distributions over candidate assignments.  See
\cite{Loredo12-Coinc} for a broad discussion of Bayesian methods for
assessing spatio-temporal coincidences in astronomical data.

\begin{figure}
\centerline{\includegraphics[clip=true,width=\textwidth]{CRCoincLevels.eps}}
\caption{Schematic depiction of the levels in our cosmic ray association
models, identifying random variables appearing in each level, including
parameters of interest (bold red labels), latent variables representing cosmic ray
properties that are not directly observable (slant type labels), and
observables (bold blue labels).}
\label{fig:levels}
\end{figure}

Our framework is designed to enable investigators to:
(1)~Ascertain which cosmic rays (if any) may be associated with specific
sources with high probability; (2)~Estimate luminosity function parameters
for populations of astrophysical sources; (3)~Estimate the proportion of all
detected cosmic rays generated by each population; (4)~Estimate parameters
describing the composition-dependent effects of cosmic magnetic fields;
(5)~Investigate whether cosmic rays from a single source are deflected
independently or share part of their deflection history (resulting in
correlated deflections).  Task (5) is not attempted here but will be
investigated in the future.

%..............................................................................
\subsection{Cosmic ray source properties}

% Note that the cosmic rays from a particular source arrive from
% different directions, so we cannot think of the flux $F_k$ in regard to
% a detector with unit area normal to the source direction; there is
% no one such direction.

We do not anticipate the UHECR flux passing through a volume element at the
Earth to vary in time over accessible time scales, so we model the
arrival rate into a small volume of space from any particular direction as a
homogeneous Poisson point process in time.  Let $F_k$ denote the UHECR flux
from source $k$.  $F_k$ is the expected number of UHECRs per unit time from
source $k$ that would enter a fully exposed spherical detector of unit
cross-sectional area.  A cosmic ray source model must specify the directions
and fluxes of candidate sources.  In our framework, a candidate source
catalog specifies source directions for a fixed number of potential sources,
$N_A$ ($N_A = 17$ for the G10 AGN catalog).  In addition, we presume some
cosmic rays may come from uncatalogued sources, so we introduce a background
component, labeled by $k=0$, considered to be a population of isotropically
distributed ``background'' sources.   We presume the background sources to
be numerous and to each have relatively low cosmic ray fluxes, so that at
most a single cosmic ray should be detected from any given background source
(i.e., we do not consider clustering of cosmic rays assigned to the
background).  In this limit, the background component may be described
by a single parameter, $F_0$, denoting the total flux from the entire
background population.

%\enote{In regard to the last comment:  I think we are implicitly presuming
%only a single CR from each background source.  This may be a slightly
%inconsistent simplification that should be noted.  E.g., for the radiant
%model, if a single unidentified background source produces a multiplet, in
%the radiant model this could lead to a ``ray'' of background sources with
%correlated directions but no apparent source.  For the buckshot model, there
%could be clusters with no apparent center.  This could be handled by
%introducing separate parameters for the total background flux, and the
%background source number density. Let's not go there for now!}

A model must specify a distribution for $\{F_k\} = \{F_0, \Fvec\}$; in
astronomical lingo, this corresponds to specifying a ``luminosity function''
for the background and source populations.  As a simple starting point, we
treat $F_0$ as a free parameter, and adopt a ``standard candle'' model
specifying the $N_A$ candidate host fluxes, $\Fvec$, via a single parameter
as follows.  We assume all sources emit isotropically with the same
intensity, $I$ (number of cosmic rays per unit time), so the flux from a
source (i.e., $F_k$ for $k>0$) can be written as $F_k = I/D_k^2$ (the
inverse-square law), with $D_k$ the (known) distance to source $k$ (there
could also be distance- and energy-dependent attenuation due to cosmic
ray-photon interactions, but the sources we consider here are close enough
that such attenuation should be negligible).  The total flux from the
sources is $F_A = \sum_{k>0} F_k$, and we adopt $F_A$ as the source
intensity parameter rather than $I$.  Thus $F_k = w_k F_A$, with the weights
$w_k$ given by
\be
w_k = \frac{1/D_k^2}{\sum_{j=1}^{N_A} 1/D_j^2},
\label{wt-def}
\ee
for $k=1$ to $N_A$.

%..............................................................................
\subsection{Top-Level Prior Specification}

We must specify a prior distribution for $F_0$ and $F_A$.
Earlier observations constrained the total UHECR flux.  In our association
model, the total flux is $F_T = F_0 + F_A$.  For the null model, there is
only one top-level parameter, the total flux from an isotropic distribution
of source directions.  So we adopt $F_T$ as a top-level parameter,
common to all models.  For association models, this motivates an alternative
parameterization that switches from $(F_0, F_A)$ to $(F_T, f)$, where $f =
F_A/(F_0 + F_A)$ is the fraction of the total flux attributed to the
candidate host population.
In this parameterization, we can specify a common total flux prior
for all models.  This is astrophysically sensible since we have
results from prior experiments to set a scale for the total flux.
It is also statistically desirable; Bayes factors tend to be robust
to specification of priors for parameters common to models being compared.

We adopt independent priors for the total flux and the associated fraction.
If their prior densities are $g(F_T)$ and $h(f)$, then the implied
joint prior density for $(F_0,F_A)$ is
\be
\pi(F_0, F_A) =
  \frac{g(F_0+F_A) h\left(\frac{F_A}{F_0+F_A}\right)}{F_0+F_A},
\label{FF-prior}
\ee
where the denominator is from the Jacobian of the transformation between
parameterizations.  In general, an independent prior for $F_T$ and $f$
corresponds to a dependent prior for $F_0$ and $F_A$.

For the calculations below, we adopt an exponential prior with scale $s$ for
$F_T$, and a beta prior for $f$ with shape parameters $(a,b)$, so
\be
g(F_T) = \frac{1}{s}e^{-F_T/s}
\quad\text{and}\quad
h(f) = \frac{1}{B(a,b)} f^{a-1}(1-f)^{b-1},
\label{exp-beta}
\ee
where $B(a,b)$ is the beta function.  We set the hyperparameters $(s,a,b)$
as follows.

We take $s = 0.01\times 4\pi$ km$^{-1}$ y$^{-1}$ for all models.  This scale
is compatible with flux estimates from AGASA and HiRes.  The likelihood
functions for $F_T$ from those experiments are formally different from
exponentials (they are more concentrated away from zero), but since this
prior is common to all models, and since the PAO data are very informative
about the total flux, our results are very robust to its detailed
specification.

For the beta prior for $f$, our default choice is $a=b=1$, which corresponds
to a uniform prior on $[0,1]$.  We also repeat some computations using $b=5$
to investigate the sensitivity of Bayes factors to this prior.  This case
skews the prior downward, increasing the probability that $f$ is close to 0.

%..............................................................................
\subsection{Cosmic ray mark distributions}

Given the fluxes, we model cosmic ray arrival times with a superposition of
homogeneous Poisson point processes from each component.  Besides its
arrival time, each cosmic ray has a label associated with it, identifying
its source component. Let $\tlabel$ be an integer-valued latent label for a
UHECR, specifying its source ($\tlabel = 0$ for the background, or $k \ge 1$
for AGN $k$).  Since a superposition of Poisson processes is a Poisson
process, we may consider the arrival times for the UHECRs arriving at Earth
to come from a total event rate process, and the labels to come from a
categorical mark distribution with probability mass function
\be
P(\lambda=k|F_0,\Fvec) = \frac{F_k}{\sum_{j=0}^{N_A} F_j}.
\label{label-pmf}
\ee
In the absence of magnetic deflection, the labels could be replaced by
source directions (with background source directions assigned
isotropically), and the process could be considered to be Poisson in time
with a directional mark distribution.  But magnetic deflection requires a
more complex setup.

Our full framework also assigns energies as marks for each cosmic ray, drawn
from a distribution describing the cosmic ray spectrum.  The energies would
be important in an analysis that seeks to infer the cutoff energy
distinguishing local UHECRs rays from lower-energy cosmic rays (i.e., the
GZK cutoff).  Although the PAO-10 catalog includes event energies, the PAO
team has already made an energy cut, and in the absence of lower-energy
data, we cannot usefully infer a cutoff.  Thus in the analysis presented
here, we ignore the energy mark distribution.  
(For models with more complex
luminosity functions and more distant sources, the energies would play a
role in accounting for the suppression of flux from distant sources due to
interactions with cosmic backgrounds.)

%..............................................................................
\subsection{Cosmic ray deflection}
\label{sec:dflxn}

After leaving a source, UHECRs will have their paths deflected as they
traverse galactic and intergalactic magnetic fields.  The Galactic field is
partially measured and is known to have both a turbulent component (varying
over length scales below $\sim 1$~kpc) and a regular component (coherent
over kpc scales and largely associated with spiral arms), with typical field
strengths $\sim 1~\mu$G.  The magnetic fields of other galaxies are at
best crudely measured and believed to be similar to the Galactic field. The
much smaller fields in intergalactic space are only weakly constrained (in
fact, cosmic rays might provide useful additional constraints); the typical
field strength is probably not larger than $\sim 10^{-9}$~G except within
galaxy clusters.

A number of investigators have modeled cosmic ray
propagation in the Galaxy, or in intergalactic space, using physical models
based on existing field measurements (recent examples include
\cite{HRM02-Lens,HMR02,D+05-CRDflxn,NM10-CRDflxn,AKP10-CRDflxn,J+10-CRSources}
; see \cite{Sigl12} for an overview).  Roughly speaking, there are two regimes
of deflection behavior, described here in the small-deflection limit
\cite{HMR02}.  As a cosmic ray with energy $E$ and atomic number $Z$ traverses
a distance $L$ spanning a regular magnetic (vector) field $\bm{B}$, it is
deflected by an angle
\be
\delta \approx
  6.4^\circ\; Z \left(\frac{E}{50~\mbox{EeV}}\right)^{-1}
  \left| \int_L \frac{d\bm{s}}{3~\mbox{kpc}} \times 
          \frac{\bm{B}}{2~\mu\mbox{G}} \right|,
\label{dflxn-reg}
\ee
where $\bm{s}$ (a vector) is an element of displacement along the
trajectory; the field and length scales are typical for the Galaxy.  If
instead it traverses a region with a turbulent structure, with the field
coherence length $\ell\ll L$, then the deflection will be stochastic; its
probability distribution has zero mean, and root-mean-square (RMS) angular
scale
\ba
\delta_{\rm rms}
  &\approx& 1.2^\circ\; Z \left(\frac{E}{50~\mbox{EeV}}\right)^{-1}
    \left(\frac{B_{\rm rms}}{4~\mu\mbox{G}}\right)
    \left(\frac{L}{3~\mbox{kpc}}\right)^{1/2}
    \left(\frac{\ell}{50~\mbox{pc}}\right)^{1/2}\\
  &\approx& 2.3^\circ\; Z \left(\frac{E}{50~\mbox{EeV}}\right)^{-1}
    \left(\frac{B_{\rm rms}}{1~\mbox{nG}}\right)
    \left(\frac{L}{10~\mbox{Mpc}}\right)^{1/2}
    \left(\frac{\ell}{1~\mbox{Mpc}}\right)^{1/2},\nonumber
\label{dflxn-turb}
\ea
where $B_{\rm rms}$ is the RMS field strength along the path, and quantities
are scaled to typical galactic and intergalactic scales on the first and
second lines, respectively.

For a detected cosmic ray, the energy is measured fairly accurately, but
other quantities appearing in the deflection formulae may be largely
unknown.  As noted above, there is significant uncertainty in the magnitudes
of cosmic magnetic fields, particularly for turbulent structures.  Turbulent
length scales are poorly known.  Finally, the composition (distribution of
atomic numbers) of UHECRs is not known.  Low energy cosmic rays are known to
be mainly protons and light nuclei, but the proportion of heavy nuclei (with
$Z$ up to 26, corresponding to iron nuclei, the most massive stable nuclei)
increases with energy up to about $10^{15}$~eV.  At higher energies,
inferring the cosmic ray composition is very challenging, requiring both
detailed measurement of air shower properties, and theoretical modeling
of the $Z$ dependence of hadronic interactions at energies far beyond
those probed by accelerators.  Measurements and modeling from HiRes
indicate light nuclei are predominant again at $\approx 1$~EeV and remain
so at least to $\approx 40$~EeV \cite{HiRes10-Final-arxiv}.
In contrast, recent PAO measurements indicate a transition from light
to heavy nuclei over the range $\approx 3$--30~EeV
\cite{PAO10-Composition,PAO12-Composition}.
(The discrepancy is not yet explained.)
For heavy nuclei, the deflection scales in both the regular and
turbulent deflection regimes can be large, $\sim 1$~rad. Some
investigators have suggested that many or most UHECRs may be heavy
nuclei originating from the nearest AGN, Cen~A, so strongly deflected
that they come from directions across the whole southern sky (e.g.,
\cite{B+09-CenA,GBdS10-CenA,BdS12-CenA}).

%\enote{All the PAO composition papers I found ignore the conflict with
%HiRes, except for a brief footnote in arxiv:1106.3048 (now in JCAP), a PAO
%paper on the Cen~A hypothesis.}


%It is difficult to measure the composition of UHECRs.  Lower-energy cosmic
%rays are mostly protons, but PAO sees suggestive evidence that there may be
%heavier nuclei among UHECRs \cite{PAO10-Composition,PAO12-Composition}.

In light of these uncertainties and the relative sparsity of UHECRs,
we use simple phenomenological models for magnetic deflection.  
In the simplest ``buckshot'' model, each cosmic ray from a particular source
experiences a deflection that is conditionally independent of the deflection
of other rays from that source, given a parameter, $\kappa$, describing the
distribution of deflections.  We have also devised a more
complex ``radiant'' model that allows cosmic rays assigned to the same
source to have correlated deflections, with the correlation representing
a partially shared deflection history.  For the analyses reported here, we
use the buckshot model; we describe the radiant model further in
\S~\ref{sec:summary}.

The buckshot deflection model adopts a Fisher distribution for the
deflection angles.  The model has a single parameter, $\kappa$, the
concentration parameter for the Fisher distribution.  The probability
density for observing a cosmic ray from direction $\tdrxn$ if it is assigned
to source $k$ with direction $\hdrxn_k$ is then
\be
\rho_k(\tdrxn|\kappa)
  = \frac{\kappa}{4\pi\sinh(\kappa)}\exp(\kappa\tdrxn\cdot\hdrxn_k).
\label{rho-def}
\ee
With this deflection distribution, when a cosmic ray is generated from an
isotropic background population, its deflected direction still has an
isotropic distribution.  Accordingly,
\be
\rho_0(\tdrxn|\kappa) =  \frac{1}{4\pi}.
\label{rho-iso}
\ee

The $\kappa$ parameter is convenient for computation, but an angular scale
is more convenient for interpretation.  The contour of the Fisher density
bounding a region containing probability $P$ is azimuthally symmetric
with angular radius $\theta$ satisfying
\be
\int_{\Omega}d \tdrxn \, \rho_k(\tdrxn|\kappa)= 
  \frac{1-e^{-\kappa[1-\cos(\theta)]}}{1-e^{-2\kappa}}=P,
\label{kappa-theta}
\ee
where $\Omega$ denotes the cone of solid angle subtended by the contour. In
plots showing $\kappa$-dependent results, we frequently provide an angular
scale axis, using (\ref{kappa-theta}) with $P=0.683$, in analogy to the
``$1\sigma$'' region of a normal distribution.\footnote{In the $\kappa\gg 1$
limit, the Fisher density becomes an uncorrelated bivariate normal with
respect to locally cartesian arc length coordinates about the mode on the
unit sphere.  The standard deviation of this normal distribution, $\theta$,
satisfies equation~(\ref{kappa-theta}) with $P \approx 0.683$; for $\kappa\gg
1$ this implies $\theta^2 \approx 2.30/\kappa$, or
$\theta \approx 86.9^\circ/\kappa^{1/2}$.}

% From full formula:
% def c1(k): return -log(1-(1-exp(-2*k))*.683)/k
% def ang(k):  return arccos(1-c1(k))*180/pi
% kappa  sig
% .5     97.5 deg
% 1      83.8
% 10     27.7
% 100    8.69
% 1000   2.75

Note that, astrophysically, $\kappa$ has a nontrivial interpretation.  If
all UHECRs are the same nuclear species (e.g., all protons), then $\kappa$
depends solely on the magnetic field history experienced by cosmic rays as
they propagate to Earth.  If UHECRs are of unknown or mixed chemical
composition, then $\kappa$ conflates magnetic field history and composition.
In a more complicated model, there could be a distribution for the values
of $\kappa$ assigned to UHECRs (accounting for different compositions and
magnetic field histories); the distribution could depend on source direction
(accounting for known magnetic field structure in the Galaxy and perhaps in
intergalactic space) and on source distance (related to the path length in
intergalactic space).

When estimating $\kappa$ or marginalizing over it, we adopt a log-flat prior
density for $\kappa\in[1,1000]$,
\be
p(\kappa) = \frac{1}{\log 1000}\,\frac{1}{\kappa}, \text{ for } 1\leq\kappa\leq1000.
\label{k-prior}
\ee
The lower limit corresponds to large angular deflection scales $\sim 1$~rad,
such as might be experienced by iron nuclei.  The upper limit corresponds
to small angular deflection scales $\sim 1^\circ$, such as might be
experienced by protons with $E\sim 100$~EeV.


%..............................................................................
\subsection{Cosmic ray detection and measurement}
\label{sec:dtxn}

Even though the arrival rate of UHECRs into a unit volume is constant in
time in our model, the expected number per unit time detected from a given
direction will vary as the rotation of the Earth changes the observatory's
projected area toward that direction, as noted above.  As a result, the
Poisson intensity function for detectable cosmic rays varies in
time for each source.

Recall that the likelihood function for an inhomogeneous Poisson
point process in time with rate (intensity function) $r(t)$ has the form
\be
\exp(-N_{\rm exp}) \prod_i r(t_i) \delta t,
\label{simple-ppp-like}
\ee
where the events are detected at times $t_i$ in detection intervals of size
$\delta t$, and $N_{\rm exp}$ is the total expected number in the observing
interval (the integral of the rate over the entire observing interval).
The likelihood function for the cosmic ray data has a similar form, but
with adjustments due to the mark distribution and measurement errors.

If the label and arrival direction for detected cosmic ray $i$ were known, the
factor in the likelihood function associated with that cosmic ray would be
$F_k \Aperp(\tdrxn_i, t_i) \delta t$, where $k=\lambda_i$.
In reality, both the label and the arrival direction are uncertain; the PAO
analysis pipeline produces a likelihood function for the direction to the
cosmic ray, $\ell_i(\tdrxn_i)$; see equation~(\ref{ell-def}).

Introducing the uncertain direction as a nuisance parameter, with a prior
denoted by $\rho_k(\tdrxn_i|\kappa)$, the likelihood factor for cosmic ray
$i$ when assigned to source $k$ may be calculated by marginalizing; it may
be written as $F_k f_{k,i}\delta t$, with
\be
f_{k,i}(\kappa) =
  \int d\omega_i \ell_i\left(\omega_i\right) \Aperp(\omega_i, t_i)
  \rho_k(\omega_i|\kappa).
\label{f-def}
\ee
The cosmic ray direction measurement uncertainty is relatively small ($\sim
1^\circ$) compared to the scale over which the area varies, so we can
approximate $f_{k,i}$ as
\ba
f_{k,i}(\kappa)
  \approx A_i\cos(\theta_i)
  \int  \ell_i(\omega_i)\rho_k(\omega_i|\kappa) d\omega_i,
\ea
where $\theta_i$ denotes the zenith angle of UHECR $i$ (reported by PAO-10)
and $A_i = A(t_i)$ is the area of the observatory at the arrival time of
UHECR $i$.  The integral can be computed analytically;
\ba
\int d\omega_i \ell_i(\omega_i) \rho_k(\omega_i|\kappa) =
\begin{cases}
\frac{\kappa_c\kappa}{4\pi\sinh(\kappa_c)\sinh(\kappa)}
  \frac{\sinh(|\kappa_c n_i+\kappa\hdrxn_k|)}{|\kappa_c n_i+\kappa\hdrxn_k|}
  & \mbox{if $k\geq 1$},\\
\frac{1}{4\pi} & \mbox{if $k=0$}.
\end{cases}
\label{f-approx}
\ea
The total event rate for cosmic rays with the properties (direction, energy,
and arrival time) of detected ray $i$ combines the contributions from each
potential source, i.e.,
$r(t_i) = \sum_k F_k f_{k,i}(\kappa)$.

To calculate $N_{\rm exp}$, we must account
for the observatory's exposure map.  The effective exposure given to cosmic
rays from source $k$ throughout the time of the survey depends, not just on
the direction to the source, but also on the deflection distribution,
$\rho_k$ (and thus on $\kappa$), since rays from that source will not arrive
precisely from the source direction.  The exposure factor for source $k$ is
\be
\epsilon_k(\kappa) =
  \int d\tdrxn \rho_k(\tdrxn|\kappa) \epsilon(\tdrxn).
\label{eps-def}
\ee
Note that $\epsilon_k$ has units of area $\times$ time, and for the
isotropic background component ($k=0$), $\epsilon_0(\kappa)$ is a constant
equal to the sky-averaged exposure (in the notation of
the \ref{supp}, $\epsilon_0 = \alpha_T/4\pi$).
To find the total expected number of detected cosmic rays we sum over
sources: $N_{\rm exp} = \sum_{k \ge 0} F_k \epsilon_k(\kappa)$.

The prior probability mass function for the label of a {\em detected} cosmic
ray is not given by (\ref{label-pmf}); the terms must be weighted according to
the source exposures.  The result is
\be
P(\lambda_i=k|F_0,\Fvec,\kappa) =
  \frac{F_k\epsilon_k(\kappa)}{\sum_{j=0}^{N_A} F_j\epsilon_j(\kappa)}.
\label{label-eps-pmf}
\ee

We now have the ingredients needed to evaluate
equation~(\ref{simple-ppp-like}), generalized to include the cosmic ray
marks (directions and labels) and their uncertainties.  The resulting
likelihood function is
\be
\like(F_0,\Fvec,\kappa)
  = \exp\left(-\sum_k F_k\epsilon_k\right)
      \prod_i \left( \sum_k f_{k,i} F_k \right).
\label{like-poisson}
\ee
The product-of-sums factor resembles the likelihood for a finite mixture
model (FMM), if we identify the $f_{k,i}$ factors as the component
densities, and the $F_k$ factors as the mixing weights.  A common technique
for computing with mixture models is to rewrite the likelihood function as a
sum-of-products by introducing latent label parameters identifying which
component each datum may be assigned to (see, e.g., \cite{BG88-BayesFMM}).
Following this approach here, the likelihood function can be rewritten as a
sum over latent assignments of cosmic rays to sources,
\be
\like(F_0,\Fvec,\kappa)
  = \sum_{\lambda} \left(\prod_k F_k^{m_k(\lambda)}e^{-F_k\epsilon_k}\right)
    \prod_i f_{\lambda_i,i},
\label{like-assoc}
\ee
where $\lambda = \{\lambda_i\}$ and $\sum_\lambda$ denotes an
$N_C$-dimensional sum over all possible assignments of cosmic rays to
sources, and the multiplicity $m_k(\lambda)$ is the number of UHECRs
assigned to source $k$ according to $\lambda$.  We suppress the $\kappa$
dependence of $\epsilon_k(\kappa)$ and $f_{k,i}(\kappa)$ here and elsewhere
to simplify expressions.  Note that the $F_k$ dependence (for a given
$\lambda$) is of the same form as a gamma distribution.

Rewriting the previous expression with $(F_T,f)$ in place of $(F_0,F_A)$,
and using $F_k = w_kF_A = f w_k F_T$ (for $k\ge 1$), we can rewrite
$\like(F_0,\Fvec,\kappa)$ as
\ba
\label{eq:like}
\like(f,F_T,\kappa) 
  &=& \sum_\lambda (1-f)^{m_0(\lambda)} f^{N_C-m_0(\lambda)} F_T^{N_C}\\
  && \times e^{-F_T\left[(1-f)\epsilon_0+f\sum_{k\geq 1} w_k\epsilon_k \right]}
     \prod_{k\geq 1} w_k^{m_k(\lambda)} \prod_i f_{\lambda_i,i}.\nonumber
\ea

For computations it will be helpful to have the likelihood function
conditional on the label assignments,
\be
P(D|\lambda,F_0,\Fvec,\kappa)
  = \exp\left(-\sum_k F_k\epsilon_k\right)
    \left[ \sum_k F_k\epsilon_k \right]^{N_C}
    \prod_i \frac{f_{\lambda_i,i}}{\epsilon_{\lambda_i}}.
\label{eq:lik-lambda}
\ee
where $k$ runs over the host labels (from 0 to $N_A$), and $i$ runs over the
UHECR labels (from 1 to $N_C$).  We can recover the likelihood for $F_0$,
$\Fvec$, and $\kappa$ by multiplying by the prior for $\lambda$ from
equation~(\ref{label-eps-pmf}) and marginalizing, giving
equation~(\ref{like-assoc}).


%..............................................................................
\subsection{Estimating $\kappa$}

To estimate the deflection parameter, $\kappa$, we need the marginal
likelihood $\like_m(\kappa) = P(D|\kappa) = \int dF_T\int df
P(D,F_T,f|\kappa)$.  The integrand is the product of equation~(\ref{eq:like})
and the flux priors.  Using the exponential and beta priors described above,
we have that the marginal likelihood for $\kappa$ is
\ba
\like_m(\kappa)
  &=& \sum_\lambda \frac{\Gamma(N_C+1)\prod_{k\geq 1}w_k^{m_k(\lambda)}
      \prod_i f_{\lambda_i,i}}{s B(a,b)}\nonumber\\
  &&  \times\int_0^1
      \frac{f^{N_C-m_0(\lambda)+a-1}(1-f)^{m_0(\lambda)+b-1}}
           {\left[\frac{1}{s} + (1-f)\epsilon_o +
                f\sum_{k\geq 1}w_k\epsilon_k\right]^{N_C+1}}df.
\label{eq:marg}
\ea
Computing $\like_m(\kappa)$ requires summing over all possible values of
$\lambda$ which is intractable in practice.  In the \ref{supp}, we
describe how to use Chib's method \cite{MR1379473} to calculate this
marginal likelihood.

%..............................................................................
\subsection{Model Comparison}

% $\like_m = \int d\kappa \pi(\kappa)\like_m(\kappa)$

We use Bayes factors (ratios of marginal likelihoods) to compare models both
conditioned on $\kappa$ (using marginal likelihood functions
$\like_m(\kappa)$), and after marginalizing over $\kappa$ (using the log-flat
prior of equation~(\ref{k-prior}), and numerical quadrature over
$\kappa$).

We consider three models.  The null model, $M_0$, assumes that all the
UHECRs come from the isotropic background source population; recall that it
has no $\kappa$ dependence (see equation~(\ref{rho-iso})).  Model $M_1$
allows the UHECRs to come from any of the 17 AGN in the catalog or from the
isotropic background.  We also consider another model, $M_2$, in which the
UHECRs may come from the isotropic background or either of the two closest
AGN, Cen~A (NGC~5128) and NGC~4945; this model is motivated in part by
recent suggestions that most UHECRs may be heavy nuclei from a single nearby
source, as cited above.  
(We also briefly explore a similarly-motivated fourth model that assigns all
UHECRs to Cen~A; as noted below, this model is tenable only for
$\kappa\approx 0$.)  In order to compare models $M_1$ and $M_2$ (conditioned
on $\kappa$) to the null model, we compute the Bayes factors:
\be
\mbox{BF}_{10}(\kappa) = \frac{\like_{m,1}(\kappa)}{\like_{m,0}},
\qquad
\mbox{ BF}_{20}(\kappa) = \frac{\like_{m,2}(\kappa)}{\like_{m,0}}
\label{BF10+20}
\ee
where $\like_{m,0}$ is the marginal likelihood for the null
model (similar equations hold for models
that marginalize over $\kappa$).  The value of $\like_{m,0}$
can be found from equation~(\ref{like-assoc}), noting that for the
null model, there is only one term in the sum over $\lambda$
(with all $\lambda_i = 0$, since the only allowed value of $k$ is $k=0$).
Marginalizing this term over $F_T$ (equal to $F_0$ in this case) gives
\be
\like_{m,0} =
  \frac{1}{s}
  \left(\frac{s}{s\epsilon_0+1}\right)^{N_C+1}
  \Gamma(N_C+1) \times \prod_i f_{0,i}.
\ee

%..............................................................................
\subsection{Computational techniques}
\label{subsec:compn}

The principal obstacle to computing with this framework is the combinatorial
explosion in the number of possible associations as the sizes of the
candidate source population and the cosmic ray sample grow.  For small
amounts of magnetic deflection, the vast majority of candidate associations
are improbable (they associate well-separated objects with each other). But
there is evidence that UHECRs may be massive (and thus highly charged)
nuclei, which would undergo significant deflection. To probe the full
variety of astrophysically interesting models requires techniques that can
handle both the small- and large-deflection regimes, for catalog sizes
corresponding to current and forthcoming catalogs from PAO.

For parameter estimation within a particular model, we have developed a
Markov chain Monte Carlo (MCMC) algorithm that draws samples of the
parameters $f$, $F_T$, and $\lambda$ from their joint posterior
distribution.  The algorithm takes advantage of two features of the models
described above.  First, by introducing latent labels, $\lambda$, we could
write the likelihood function in a sum-of-products form,
equation~(\ref{like-assoc}), with factors that depend on the fluxes
$F_k$ in the manner of a gamma distribution.  Second, the forms of the
likelihood and priors are conjugate for $F_T$ and the labels, so we can find
closed-form expressions for their conditional distributions.  These features
enable us to use Gibbs sampling techniques well known in mixture modeling
for sampling the $F_T$ and $\lambda$ parameters.  We handle the $f$
parameter using a random walk Metropolis algorithm, so our overall algorithm
is a Metropolis-within-Gibbs algorithm.  The \ref{supp} provides details on
its implementation.

We treat the deflection parameter, $\kappa$, specially, considering
a logarithmically-spaced grid of values that we condition on.  We
did this so that we could explore the $\kappa$ dependence more
thoroughly than would be possible with posterior sampling of $\kappa$.
Of course, our Metropolis-within-Gibbs algorithm could be supplemented
with $\kappa$ proposals to enable sampling of the full posterior.

Finally, using Bayes factors to compare rival models requires computing
marginal likelihoods, which are not direct outputs of MCMC algorithms.
Using a simplified version of our model, and modest-sized simulated
datasets, we explored several approaches for marginal likelihood computation
in a regime where we could compute the correct result via direct summation
over all feasible associations.  We explored the harmonic mean estimator
(HME), Chib's method, and importance sampling algorithms.  The HME
performed poorly, often apparently converging to an incorrect result (such
behavior is not unexpected; see \cite{WS12-HMBad}).  Importance sampling
proved inefficient.  Chib's method was both accurate and efficient in these
trial calculations, and became our choice for the final implementation.
The \ref{supp} provides details.


\section{Results}
\label{sec:results}

Recall that the UHECR data reported by PAO-10 are divided into three periods.
The PAO team used an initially larger period~1 sample (including
lower-energy events) to optimize an energy threshold determining which
events to analyze in period~2; the reported period~1 events are only those
with energies above the optimized threshold.  The optimization maximized a
measure of anisotropy in the above-threshold Period~1 sample.  Without access
to the full Period~1 sample, we cannot evaluate the impact of this
optimization on our modeling of anisotropy in the reported Period~1 data
(nor can we usefully pursue a Bayesian treatment of a GZK energy cutoff
parameter).  Because of this complication, we have performed
analyses for various subsets of the data.  As our main results, we report
calculations using data from Periods 2 and 3 combined (``untuned data'') and
for Periods 1, 2, and 3 combined (``all data'').  We also report some
results for each period considered separately, and we use them to perform a
simple test of consistency of the results across periods, in an effort
to assess the impact of tuning on the suitability of the Period~1 data
for straightforward statistical analysis.

%..............................................................................
\subsection{Results conditioning on the deflection scale, $\kappa$}

We first consider models conditional on the value of the magnetic deflection
scale parameter, $\kappa$, calculating Bayes factors comparing models,
and estimates of the association fraction, $f$.

We report model comparison results as curves showing Bayes
factors as functions of $\kappa$.  These quantities are astrophysically
interesting but must be interpreted with caution.  The actual values of the
conditional Bayes factors can only be interpreted as Bayes factors for a
particular value of $\kappa$ deemed interesting {\em a priori}.  For example,
were one to assume that UHECRs are protons, adopt a particular Galactic
magnetic field model, and assume that intergalactic magnetic fields do not
produce significant deflection (which is plausible for protons from local
sources), one would be interested only in large values of $\kappa$ of order
several hundred (corresponding to small angular scales for deflection).  On
the other hand, if one presumed that UHECRs are predominantly heavy nuclei,
then deflection by Galactic fields could be very strong, corresponding to
$\kappa$ of order unity (deflection by intergalactic fields might also be
significant in this case).  Models hypothesizing that most UHECRs are
heavy nuclei produced by Cen~A would fall in this small-$\kappa$ regime.
By presenting results conditional on $\kappa$, various cases
such as these may be considered.  Also, the Bayes factor conditioned on
$\kappa$ is proportional to the marginal likelihood for $\kappa$, so the
same curves summarize the information in the data for estimating $\kappa$ if
it is considered unknown.  We plot the curves against a logarithmic $\kappa$
axis, so they may be interpreted (up to normalization) as posterior
probability density functions based on a log-flat $\kappa$ prior.

The Bayes factors comparing models $M_1$ and $M_2$ to $M_0$ for various values
of $\kappa\in[1,1000]$, and for various partitions of the data, are shown in
Figure~\ref{fig:BFplot}.  For cases using only the untuned data (periods 2, 3,
and $2+3$), we find that both BF$_{10}$ and BF$_{20}$ (see
equation~\ref{BF10+20}) are close to 1 for all values of $\kappa\in[1,1000]$
for the beta-(1,1) (uniform) prior for $f$.  The Bayes factors are only
a little higher in the case of beta-(1,5) prior, indicating the results
are robust to reasonable changes in the $f$ prior. These values imply
that the posterior odds for the association models $M_1$ and $M_2$
versus the null isotropic background model $M_0$ are nearly equal to the
prior odds, indicating the untuned data provide little evidence for or
against either association model versus the isotropic model.

\begin{figure}
\centerline{\includegraphics[angle=-90,width=\textwidth]{BF_kappa_17AGNs_1+2.eps}}
\caption{Bayes factors comparing the assocation model with 17 AGNs (top row) or
2 AGNs(bottom row) with the null isotropic background model, conditional
on $\kappa$, shown as a function of $\kappa$ (bottom axis) and the
corresponding deflection angle scale, $\sigma$ (top axis).  Results are
shown for various partions of the data (identified by line style,
identified in the legend), and for two choices of the prior on $f$: a
flat prior (left column), and a Beta$(1,5)$ prior (right column).}
\label{fig:BFplot}
\end{figure}

Considering the Period~1 data qualitatively changes the results.  The
solid (blue) curves in Figure~\ref{fig:BFplot} show the Bayes factor
vs.\ $\kappa$ results based solely on the Period~1 data; there is
strong evidence for association models conditioned on $\kappa$ values
of around 50 to 100.\footnote{A common convention for interpreting
Bayes factors is due to Kass and Raftery, who consider a Bayes factor between
3 and 20 to indicate ``positive'' evidence and between 20 and 150 to indicate
``strong'' evidence \cite{Kass:Raft:baye:1995}.}
Analyzing the data from all three periods jointly produces the long-dashed
(purple) curves.  Using a uniform prior for $f$, we find BF$_{10}$ attains a
maximum of 90 at $\kappa\approx 46$ while BF$_{20}$ attains a maximum of 262
at $\kappa\approx 38$.  Both BF$_{10}$ and BF$_{20}$ are larger than 30 for
all $\kappa\in[20,120]$.  Both of the association models are strongly
preferred over the null in this range of $\kappa$, while the comparison is
inconclusive for $\kappa$ outside this range.

The originally published data (in PAO-08) covered periods 1 and 2.  For
comparison with studies of that original catalog, Figure~\ref{fig:BFplot}
include curves showing the Bayes factor vs.\ $\kappa$ based on data
from periods 1 and 2.  This partition of the data produces the largest
Bayes factors, $\sim 1000$ for $\kappa \approx 50$.  The curves are
qualitatively consistent with accumulation of evidence from Period~1 and
Period~2.\footnote{Note that the Bayes
factor for the $1+2$ partition should not be expected to equal the
product of the Bayes factors based on the Period~1 and Period~2 partitions,
because the models are composite hypotheses, and the data from different
periods generally will favor different values of the model parameters.}
These results amplify what was found in the analysis using all of
the data:  the strongest evidence for association comes from the Period~1
data.  This is troubling because this data was used (along with unreported
lower-energy data) to tune the energy cut defining all of the samples, and
there is no way for independent investigators to account for the effects of
the tuning on the strength of the evidence in the Period~1 data.

We show marginal posterior distributions for $f$ in
Figure~\ref{fig:postf}, for both $M_1$ and $M_2$, using both the untuned
data, and using all data. For a given model, the posterior does not
change much when Period~1 data are included.  The posteriors indicate
evidence for small but nonzero values of $f$, of order a few percent to
20\%.  They strongly rule out values of $f>0.3$, indicating that most
UHECRs must be assigned to the isotropic background component in these
models.
This holds even for values of $\kappa$ as small as $\approx 10$,
corresponding to quite large magnetic deflection scales, as might be
experienced by iron nuclei in typical cosmic magnetic fields.  Of
course, when $\kappa=0$ the association models become indistinguishable
from an isotropic background model.  These results suggest that a model
assigning {\em all} UHECRs to a single nearby source, such as Cen~A, may be
tenable only with very large deflection scales.  In the
\ref{supp} we briefly explore such a model and confirm this
conclusion.

\begin{figure}
\centerline{\includegraphics[angle=-90,width=.9\textwidth]{postf.eps}}
\caption{Posterior distributions for $f$, conditioned on $\kappa$ = 10,
31.6, 100, 316 and 1000.}
\label{fig:postf}
\end{figure}

A recent approximate Bayesian analysis \cite{WMJ11-BayesUHECR}, based on a
discrete pixelization of the sky, attributed a similar fraction of the
sample of 27 Period~1 and Period~2 UHECRs to standard-candle AGN sources,
considering $\approx 900$ AGN within 100~Mpc from the VCV as candidate
sources.  We compare our approaches and results in the \ref{supp}.

The posterior mode is at larger values of $f$ for model $M_1$ (with 17 AGN)
than for $M_2$ (with the two closest AGN), suggesting that there is evidence
that AGN in the G10 catalog besides Cen~A and NGC~4945 are sources of
UHECRs.  Our multilevel model allows us to address source identification
explicitly, by providing a posterior distribution for possible association
assignments (values of $\lambda$).  In Table~\ref{lambdaTable} we show
marginal posterior probabilities for associations that have non-negligible
probabilities (i.e., $>0.1$), based on models $M_1$ and $M_2$ for two
representative values of $\kappa$ ($\kappa = 31.62$, corresponding to a
$15.5^\circ$ deflection scale, is a favored value for analyses including
Period~1 data as shown below; $\kappa=1000$, corresponding to a $2.7^\circ$
deflection scale, may be appropriate if UHECRs are predominantly protons). 
Rows are labeled by cosmic ray number, $i$, and columns by AGN number, $k$;
the tabulated values are $P(\lambda_i=k|\cdots)$.  Cosmic rays 17 and 20 (in
Period~2) are associated with Cen~A (AGN 13) with modest to high probability
in all cases.  No other assignments are robust (notably, Period~3 has no
robust assignments, despite containing more than three times the number of
cosmic rays as Period~2).  If UHECRs experience only small deflections, then
besides the two Cen~A associations, it is highly probable that cosmic ray 8
(in Period~1) is associated with NGC~4945.  For the larger deflection scale,
nearly a quarter of the cosmic rays have candidate associations with
probability $>0.1$, although none of those associations have probability
$>0.5$.
\mnote{Added last sent.}
The larger favored value of $f$ for $M_1$ thus reflects the 17~AGN
model finding enough plausible associations (besides those with Cen~A and
NGC~4945) that it is likely that some of them are genuine, even though
it cannot specify which.

We can also calculate posterior probabilities for multiplet assignments.  In
general, the probability for a multiplet assigning a set of cosmic rays to a
particular candidate source will not be the product of the probabilities for
assigning each ray to the source.  In Table~\ref{lambdaTable} we see that
CRs 17 and 20 are often commonly assigned to Cen~A.  As an example, for
$M_1$ with $\kappa=1000$, their separate probabilities for assignment to
Cen~A are 0.85 and 0.94, respectively.  The probability for a doublet
assignment of both of them to Cen~A in this model is 0.80, which happens to
be nearly equal to the product of their separate (marginal) assignment
probabilities.  Were we to marginalize over $\kappa$, the multiplet
probability would differ from the product, since the preferred value
of $\kappa$ differs slightly between these two CRs.

%\enote{Do we want to say any more about multiplet probabilities, e.g.,
%for low-$\kappa$ models?}

\begin{sidewaystable}
\begin{center}
\begin{tabular}{|c|c|c|c|c|c|c|c|c|c|c|c|c|c|c|}

  \hline
  & \multicolumn{10}{|c|}{17 AGN + isotropic} & \multicolumn{4}{|c|}{2 AGN + isotropic} \\
  \cline{2-15}
  & \multicolumn{5}{|c|}{$\kappa$=31.62} & \multicolumn{5}{|c|}{$\kappa$=1000}& \multicolumn{2}{|c|}{$\kappa$=31.62} & \multicolumn{2}{|c|}{$\kappa$=1000} \\
   \cline{2-15}
  CR$\backslash$ AGN &
        2  &   7  &  11  &  13  &  17  &   2   &   11   &   13   &   16  &   17  &   11  &   13  &   11   &   13  \\
  \hline
  2 &   -  &   -  & 0.24 & 0.46 &   -  &   -   &   -    &    -   &   -    &   -    & 0.26  & 0.51  &   -    &   -   \\
  3 &   -  &   -  & 0.42 & 0.20 &   -  &   -   &   -    &    -   &   -    &   -    & 0.47  & 0.22  &   -    &   -   \\
  4 &   -  &   -  &   -  &    - & 0.17 &   -   &   -    &    -   &   -    &   -    &   -   &    -  &   -    &   -   \\
  5 &   -  &   -  & 0.18 & 0.28 &   -  &   -   &   -    &    -   &   -    &   -    & 0.22  & 0.35  &   -    &   -   \\
  6 & 0.11 &   -  &   -  &   -  &   -  &   -   &   -    &    -   &   -    &   -    &   -   &   -   &   -    &   -   \\
  8 &   -  &   -  & 0.43 & 0.36 &   -  &   -   & 0.89   &    -   &   -    &   -    & 0.47  & 0.38  & 0.90   &   -   \\
 13 &   -  &   -  &   -  &   -  & 0.17 &   -   &   -    &   -    &   -    & 0.11   &   -   &   -   &   -    &   -   \\
 14 &   -  &   -  & 0.47 & 0.27 &   -  &   -   &   -    &   -    &   -    &   -    & 0.51  & 0.29  &   -    &   -   \\
 \hline
 17 &   -  &   -  & 0.38 & 0.41 &   -  &   -   &   -    & 0.85   &   -    &   -    & 0.41  & 0.44  &   -    & 0.86  \\
 18 &   -  &   -  &   -  & 0.15 &   -  &   -   &   -    &   -    &   -    &   -    &   -   & 0.20  &   -    &   -   \\
 20 &   -  &   -  & 0.36 & 0.43 &   -  &   -   &   -    & 0.94   &   -    &   -    & 0.39  & 0.46  &   -    & 0.95  \\
 23 &   -  &   -  & 0.32 & 0.26 &   -  &   -   &   -    &   -    &   -    &   -    & 0.37  & 0.30  &   -    &   -   \\
 26 &   -  & 0.17 & 0.10 & 0.19 &   -  &   -   &   -    &   -    &   -    &   -    & 0.15  & 0.27  &   -    &   -   \\
 \hline
 33 &   -  &   -  & 0.40 & 0.11 &   -  &   -   &   -    &   -    &   -    &   -    & 0.46  & 0.12  &   -    &   -   \\
 34 &   -  &   -  & 0.47 & 0.27 &   -  &   -   &   -    &   -    &   -    &   -    & 0.51  & 0.30  &   -    &   -   \\
 36 &   -  &   -  & 0.21 & 0.35 &   -  &   -   &   -    &   -    & 0.48   &   -    & 0.24  & 0.42  &   -    &   -   \\
 47 &   -  &   -  & 0.14 & 0.42 &   -  &   -   &   -    &   -    &   -    &   -    & 0.15  & 0.48  &   -    &   -   \\
 54 &   -  &   -  & 0.19 & 0.46 &   -  &   -   &   -    &   -    &   -    &   -    & 0.21  & 0.52  &   -    &   -   \\
 55 & 0.15 &   -  &   -  &   -  &   -  & 0.34  &   -    &   -    &   -    &   -    &   -   &   -   &   -    &   -   \\
 57 &   -  & 0.41 &   -  &   -  &   -  &   -   &   -    &   -    &   -    &   -    &   -   &   -   &   -    &   -   \\
 67 &   -  &   -  & 0.32 & 0.30 &   -  &   -   &   -    &   -    &   -    &   -    & 0.37  & 0.34  &   -    &   -   \\
 \hline

\end{tabular}
\end{center}
\caption{The posterior probability that each cosmic ray is assigned to each
AGN given $\kappa = 31.62$ and 1000, using cosmic rays from periods 1+2+3.
Only assignments with probabilities greater than 0.1 are shown.  The AGN
identifiers are:  2:~NGC~0613; 7:~NGC~3621; 11:~NGC~4945;
13:~NGC~5128  (Cen~A); 17:~NGC~6300.}
\label{lambdaTable}
\end{sidewaystable}


%..............................................................................
\subsection{Results with $\kappa$ as a free parameter}

Joint marginal posterior distributions for $\log_{10}(\kappa)$ and $f$ are
shown in Figure~\ref{fig:jointkappaf}, for both association models, and for
untuned data and all data samples.  For the all-data cases, the joint
posterior distribution is unimodal and attains its maximum at ($\kappa$=32 ,
$f$=0.13) and ($\kappa$=32 , $f$=0.09) for the association model with 17
AGNs and 2 AGNs, respectively.  For untuned data, the joint posteriors are
bimodal with one of the mode at the value of $\kappa$ slightly less than in
the case of all 3 periods, and the other mode at $\kappa\approx 1000$,
similar to the plot of Bayes factors in Figure~\ref{fig:BFplot}.  The
results from the two samples are more similar than this description may
indicate; they have significant peaks in the same region, but the likelihood
function is relatively flat for the largest and smallest values of $\kappa$
(this is also apparent in Figure~\ref{fig:BFplot}).

In all cases, the preferred values of $\kappa$ correspond to deflection
scales $\approx 10^\circ$.  As noted above, models of proton propagation in
cosmic magnetic fields predict deflections of a few degrees.
The posterior distributions for $\kappa$ are comfortably consistent with
such predictions, but they do favor the larger scales that would be
experienced by heavier nuclei.  These scales are consistent with the
suggestive evidence from PAO that UHECRs may be comprised of heavier nuclei
than lower-energy cosmic rays.

\begin{figure}
\centerline{\includegraphics[width=\textwidth]{kappa_f-log_kappa.eps}}
\caption{Marginal joint posterior distributions for the magnetic deflection
concentration parameter, $\kappa$, and the association fraction, $f$,
considering UHECR data from different periods, and candidate host catalogs of 2
or 17 nearby AGN.  Contours bound HPD credible regions of probability
0.25 (blue), 0.5 (green), 0.75 (red), 0.95 (brown), and 0.99 (gray).}
\label{fig:jointkappaf}
\end{figure}

Values for Bayes factors accounting for $\kappa$ uncertainty are listed in
Table~\ref{tab:BFtab}, for both association models, and for both individual
and combined data samples (these values are based on the default flat
prior for $f$).  We find strong evidence for both association models when
considering all the cosmic ray data.  If we exclude the tuned data of period
1, then we see positive evidence for association if we consider only period
2 but positive evidence for the null model if we consider only period 3.  If
we pool the untuned data, the data are equivocal.  Together, these results
raise concerns about consistency of the data and adequacy of the models; we
address this further below.  These results do not change qualitatively when
we use the alternative prior for $f$ described in \S~3.2.

\begin{table}
\begin{tabular}{|c|c|c |c| c| c| c| c|}
\hline
& & \multicolumn{6}{|c|}{Data Periods Used}\\
\cline{3-8}
 Priors for $f$ & Model & 1 &  2 &  3 &  1\&2 & 2\&3 &  1\&2\&3\\
\hline
beta(1,1) & 17 AGNs & 31 & 6.5 & 0.15 = 1/6.7 & 370 & 0.99 & 26\\
 & 2 AGNs  & 15 & 9.9 & 0.11 = 1/9.1 & 440 & 1.1 & 51\\
\hline
beta(1,5) & 17 AGNs & 39 & 15 & 0.52 = 1/1.9 & 710 & 3.4 & 79\\
 & 2 AGNs  & 32 & 28 & 0.42 = 1/2.4 & 1100 & 4.1 & 180\\
\hline
\end{tabular}
\caption{Overall Bayes factors comparing association models with 17 AGNs or
2 AGNs to the null isotropic background model, for two different priors for
$f$}\label{tab:BFtab}
\label{BFTable}
\end{table}

%> 1/.11
%[1] 9.09091
%> 1/.15
%[1] 6.666667

Marginal posterior distributions for $f$ and for $F_T$ are shown in
Figure~\ref{fig:posterior}.  For the untuned data, the posterior mode of
$f$ is 0.051 for $M_1$ (17 AGNs) and 0.047 for $M_2$ (2 AGNs);
the 95\% highest density credible intervals for $f$ are $[0, 0.23]$
and $[0.002, 0.145]$, respectively.  Using all of the data, the
distributions shift to somewhat larger values of $f$; the
posterior mode of $f$ is 0.11 for $M_1$ and 0.08 for $M_2$, and $f=0$
has a significantly smaller density.  However, the uncertainties are
large enough that the $f$ estimates are consistent with each other.  The
posterior distributions for $F_T$ are very similar in all models.  The
peaks are a little higher and the widths of the peaks are smaller when
we consider the cosmic rays from periods 1--3, as expected, since we
have more data.  The posterior modes correspond to total fluxes of about
0.04 km$^{-2}$ yr$^{-1}$ in all cases.

\begin{figure}
\centerline{$
\begin{array}{cc}
\includegraphics[angle=-90,width=.5\textwidth]{posterior_f_all_margOverKappa.eps} &
\includegraphics[angle=-90,width=.5\textwidth]{posterior_FT_all_margOverKappa.eps}
\end{array}$}
\caption{Marginal posterior distributions for $f$ (the fraction of
UHECRs associated with AGNs in candidate catalogs), and $F_T$ (the total
flux), considering UHECR data from different periods, and models
associating UHECRs with either 2 or 17 nearby AGN.}
\label{fig:posterior}
\end{figure}


%\begin{figure}
%\centerline{$
%\begin{array}{cc}
%\includegraphics[width=1.75in,angle=-90]{posterior_F_margOverKappa_17AGNs.eps} &
%\includegraphics[width=1.75in,angle=-90]{posterior_F_margOverKappa_2AGNs.eps}
%\end{array}$}
%\caption{50\%, 68\% and 95\% credible regions from joint posteriors for $(F_0,F_A)$ for 17-AGN %model (left) and 2-AGN model (right)}
%\label{fig:jointF}
%\end{figure}


% This is in place of the model checking section, which appears here in
% the arXiv version:


\subsection{Model checking}

In the \ref{supp} we describe results of two types of
tests of our models, motivated by period-to-period variability of some of
the results reported above.

First, we performed simple change point analyses to see whether the
period-to-period variation of the Bayes factors for association vs.\ isotropy
indicates the population-level properties of the detected cosmic rays vary
from period to period.  We compared versions of $M_1$ and $M_2$ that allow
model parameters to change between periods to versions that keep the
parameters the same for all periods.  We find that there is no significant
evidence for variability of model parameters from period to period.  

%That is, presuming one of the models is adequate, the apparent discrepancy
%among the Bayes factors in Table~\ref{BFTable} reflects variability that may
%be expected for these modest sample sizes.

Second, we performed predictive checks to see whether the period-to-period
Bayes factor variations are surprising in the context of either the null or
association models, essentially using the Bayes factors as goodness-of-fit
test statistics.  
We simulated data from the null (isotropic) model and compared the Bayes
factors based on the observed data with those found in the simulations; we
did the same for a representative association model.
We find that Bayes factors favoring association as large as that
found with the Period~1 PAO data are unlikely for
isotropic models.
This implies the distribution of directions in the Period~1 sample is
anisotropic, but the calculation does not address whether this may be due to
tuning or to genuine anisotropy.
For association models, the large Bayes factors for periods 1 and 2,
and the small Bayes factor in Period~3, are not individually surprising.
But it is very surprising to see a combination of large Bayes factors
for the two small subsamples, and a small Bayes factor for the large
subsample.  The full dataset thus is not comfortably fit by either
isotropic or association models.  We discuss this further below.


\section{Summary and Discussion}
\label{sec:summary}

We have described a new multilevel Bayesian framework for modeling the
arrival times, directions, and energies of UHECRs, including statistical
assessment of directional coincidences with candidate sources.
Our framework explicitly models cosmic ray emission, propagation (including
deflection of trajectories by cosmic magnetic fields), and detection.  This
approach cleanly distinguishes astrophysical and experimental processes
underlying the data.  It handles uncertain parameters in these processes via
marginalization, which accounts for uncertainties while allowing use of all
of the data (in contrast to hypothesis testing approaches that optimize over
parameters, requiring holding out a subset of the data for tuning).
We demonstrated the framework by implementing calculations with simple but
astrophysically interesting models for the 69 UHECRs with energies above
55~EeV detected by PAO and reported in PAO-10.  Here we first summarize
our findings based on these models, and then describe directions for
future work.

%Its Bayesian underpinning enables accounting for
%a priori uncertainty in model parameters via averaging (marginalization).
%This stands in contrast to previously-used hypothesis testing approaches
%that handle free parameters defining a test procedure by optimizing using
%a subset of the data, with subsequent analysis omitting the tuning data.

%..............................................................................
\subsection{Astrophysical results}

We modeled UHECRs as coming from either nearby AGN (in a volume-limited
sample including all 17 AGN within 15~Mpc) or an isotropic background
population of sources; AGN are considered to be standard candles in
our models.  We thoroughly explored three models.  In $M_0$ all CRs come
from the isotropic background; in $M_1$ all CRs come from either a
background or one of the 17 closest AGN; in $M_2$ all CRs come from either
a background source or one of the two closest AGN (Cen~A and NGC~5128,
neighboring AGN at a distance of 5~Mpc).  The data were reported in three
periods.  Data from Period~1 were used to tune the energy threshold defining
the published samples in all periods by maximizing an index of anisotropy in
Period~1.  Out of concern that this tuning compromises the data in Period~1
for our analysis, we analyzed the full dataset and various subsamples,
including an ``untuned'' sample omitting Period~1 data.

Using {\em all} of the data, Bayes factors indicate there is strong evidence
favoring either $M_1$ or $M_2$ against $M_0$ but do not discriminate between
$M_1$ and $M_2$.  The most probable models associate about 5\% to 15\% of
UHECRs with nearby AGN, and strongly rule out associating more than
$\approx 25$\% of UHECRs with nearby AGN.  Most of the high-probability
associations in the 17~AGN model are with the two closest AGN.

However, if we use only the {\em untuned} data, the Bayes factors are
equivocal (although the most probable association models resemble those
found using all data).  If we subdivide the untuned data, we find positive
evidence for association using the Period~2 sample, but weak evidence {\em
against} association using the much larger Period~3 sample.  Together, these
results suggest that the statistical character of the data may differ from
period to period, due to tuning of the Period~1 data or other causes.

One way to explore this is to ask whether the data from the various periods
are better explained using models with differing parameter values rather
than a shared set of values.  We investigated this via a change-point
analysis that considered the times bounding the periods as candidate change
points.  The results are consistent with the hypothesis that the parameters
do {\em not} vary between periods, justifying using the combined data for
these models.  This suggests the variation of the Bayes factors across
periods is a consequence of the modest sample sizes.  However, the
change point analysis does not address the possibility that none of the
models is adequate, with model misspecification being the cause of the
apparently discrepant Bayes factors.

We used simulated data from both the isotropic model and
high-probability association models to perform predictive checks of our
models, using the Bayes factors based on subsets of the data as test
statistics.  The simulations show that the between-period variation seen
here is not unexpected for the currently available sample sizes when data are
simulated from association models.  The simulations also indicate that large
Bayes factors favoring association are unlikely for {\em untuned} samples of
the size of the Period~1 sample in the context of isotropic models.  These
predictive checks indicate that the Period~1 data are consistent with being
a fair sample from an association model but not from an isotropic model.
Whether the effects of tuning could explain the apparent inconsistency of
the Period~1 data with an isotropic origin remains an open question that is
not easy to address without access to the untuned data.

\mnote{Added Cen~A summary}
Overall, the results indicate the data are consistent with standard
candle models that assign $\approx 2$\% to 20\% of UHECRs to nearby AGN
(within 15~Mpc), with the remainder assigned to sources drawn from an
isotropic distribution.
Relatively small magnetic deflection angular scales of $\approx 3^\circ$
to $20^\circ$ are favored; models that assign a large fraction of UHECRs
to a single nearby source (e.g., Cen~A) are ruled out unless very large
deflection scales are specified a priori, and even then they are disfavored.

Despite the consistency of these results, we hesitate to offer these models
as astrophysically plausible explanations of the PAO UHECR data, both
because of how important the problematic period~1 sample is in the analysis,
and because of astrophysical limitations of the models considered here and
elsewhere.  In particular, the high-probability models assign the
vast majority of UHECRs to sources in an isotropic distribution. But the
observation by PAO of a GZK-like cutoff in the energy spectrum of UHECRs
argues strongly that UHECRs originate from within $\sim 100$~Mpc, where the
distribution of both visible matter (galaxies) and dark matter is
significantly {\em an}isotropic.  If most or all UHECRs are protons, so that
magnetic deflection is not very strong, an isotropic distribution of
UHECR arrival directions is implausible. It then may be the case that some
of the strength of the evidence for association with nearby AGN is due to
the ``straw man'' nature of the isotropic alternative.  On the other hand, if
most UHECRs are heavy nuclei, then strong magnetic deflection could
isotropize the arrival directions. The highest probability association
models have relatively small angular deflection scales, but it could be that
the few UHECRs that these models associate with the nearest AGN happen to be
protons or very light nuclei.  Future models could account for this
by allowing a mixture of $\kappa$ values among cosmic rays, as noted
in \S~\ref{sec:dflxn}.

In addition, the standard candle cosmic ray intensity model adopted here and
in other studies is astrophysically implausible and very likely strongly
constrains inferences.  The strongest visible clustering of measured UHECR
directions is toward the two closest AGN, just 5~Mpc away and only a few
degrees apart on the sky.  Most of the high-probability associations
identified in our models are to these AGN.  They are so close that they
imply standard-candle cosmic ray intensities that would produce a negligible
flux of cosmic rays from the vast majority of other AGN within 100~Mpc.  Put
another way, a standard-candle model assigning just one UHECR to an AGN near
the 100~Mpc GZK limit would imply a cosmic ray flux from nearby AGN so huge
that this scenario is ruled out simply by visible inspection of sky maps. 
Models with more flexible luminosity functions would likely allow assignment
of many more UHECRs to sources spanning a range of distances.

%..............................................................................
\subsection{Future directions}

All of these considerations indicate a more thorough exploration of
UHECR production and propagation models is needed.  We thus consider
the analyses here to be a demonstration of the utility and feasibility
of analyzing such models within a multilevel Bayesian framework, and
not a definitive astrophysical analysis of the data.
We are pursuing more complex models elsewhere, expanding on the present
analysis in four directions.

First, we are considering larger, statistically well-characterized catalogs
of potential hosts, e.g., the recently-compiled catalog of X-ray selected
AGN detected by the Burst and Transient (BAT) instrument on the {\em Swift}
satellite, a catalog considered by PAO-10.

Second, we are building more realistic background distributions, for example
by using the locations of nearby galaxy clusters, or the entire nearby
galaxy distribution, to build smooth background densities (e.g., via kernel
density estimation, or fitting of mixture or multipole models).

Third, we are considering richer luminosity function models, including models
assigning a distribution of cosmic ray intensities to all candidate sources,
and models that place some sources in ``on''
states and the others ``off.''  The latter models are motivated both by the
possibility of beaming of cosmic rays, and by evidence for AGN intermittency
in jet substructure, and could enable assignment of significant numbers of
UHECRs to both distant and nearby sources.

Finally, more complicated deflection models are possible.  
For example, we have developed a class of ``radiant'' models that produce
correlated deflections (as seen in some astrophysical simulations).  For a
radiant model, each source has a single guide direction associated with it,
drawn from a Fisher distribution centered at the source direction, with
concentration $\kappa_g$; the guide direction serves as a proxy for the shared
magnetic deflection history of cosmic rays from that source.  Each cosmic ray
associated with that source then has its arrival direction drawn from an
independent Fisher distribution centered about the guide direction, with
concentration potentially depending on cosmic ray energy and source distance;
this distribution describes the effect of the deflection history unique to a
particular cosmic ray.  The resulting directions for a multiplet will cluster
along a ray pointing toward the source.  The resulting joint distribution for
the directions in a multiplet (with the guide direction marginalized) is
exchangeable but not independent.

For the current, modest-sized UHECR catalog, the complexity of
some of these generalizations  is probably not warranted.  But PAO is
expected to operate for many years, and the sample is continually growing in
size.  Making the most of existing and future data will require, not only
more realistic models, but also more complete disclosure of the data.
In particular, a fully Bayesian treatment---including modeling of the energy
dependence in the UHECR flux and deflection scale---requires data
uncorrupted by tuning cuts.  Further, the most accurate analysis should use
event-specific direction and energy uncertainties (likelihood summaries),
rather than the typical error scales currently reported.  We hope 
our framework helps motivate more complete releases of future PAO data.


% These are in the Supp:
%\appendix
%\section{Auger observatory exposure}
\label{app:expo}

PAO can reliably detect and measure UHECRs arriving from directions within
$60^\circ$ from the observatory zenith.  Due to Earth's rotation, the zenith
traces a circular path on the sky, and the PAO field of view changes with time.
The observatory's geodetic latitude is $-35.5^\circ$, so the field of view
always includes the SCP and directions within a $5.5^\circ$ cone about it.
Outside of that cone, the time spent within the field of view decreases with
increasing latitude, vanishing for northern latitudes above $24.5^\circ$.  This
boundary corresponds to the thick gray curve shown in the sky maps.

In addition, at a given instant, the projected area of the observatory varies
with direction.  Since the observatory detects air showers, the effective area
of the detector toward a particular source direction is the projected area of
the layer of atmosphere above PAO toward that direction.  Due to Earth's
rotation, this projected area is a function of time; we denote it by
$A_p(t,\tdrxn)$ for direction $\tdrxn$ (a unit vector toward a fixed direction
on the sky) at time $t$.  We account for the zenith angle criterion by
setting $A_p=0$ for directions outside of the instantaneous field of view.

As described in \S~\ref{sec:data}, the exposure map, $\expo(\tdrxn)$,
is defined by
\begin{equation}
\expo(\tdrxn) \coloneqq \int_T  A_p(\tdrxn,t) dt.
\label{expo-def}
\end{equation}
The projected area can be written as $A_p(\tdrxn,t) = A(t) \mu(\tdrxn,t)$,
where $A(t)$ is the effective planar area of the detection volume, and
$\mu(t,\tdrxn)$ is a projection factor.  $A(t)$ varies slowly with time as
the observatory grows.  The projection factor varies much more quickly, due
to Earth's rotation. The PAO team has shown that for UHECRs, to a good
approximation a simple geometric projection varying periodically with a
period of 1~sidereal day gives a very accurate description of the PAO
exposure \cite{2001APh....14..271S}. As a result, the time integral in
equation~(\ref{expo-def}) can be approximated as $\int A(t) m(\tdrxn) dt$,
where $m(\tdrxn)$ is the geometric projection factor averaged over 1
sidereal day. The average projection factor is constant with respect to
right ascension (the equatorial sky coordinate corresponding to geodetic
longitude) due to rotational averaging.  It varies strongly with declination
(the equatorial sky coordinate corresponding to geodetic latitude).
Figure~\ref{fig:pjxn} shows the average projection as a function of
declination.

\begin{figure}
\centerline{\includegraphics[width=.8\textwidth]{avg_pjxn_factor.eps}}
\caption{Average projection factor, $m(\tdrxn)$, describing the declination
dependence of the PAO exposure map.}
\label{fig:pjxn}
\end{figure}

With these approximations, to evaluate $\expo(\tdrxn)$, we need the
time integral of the observatory area, $A(t)$.  By convention, this
quantity is reported indirectly by describing the observatory's
sensitivity to an isotropic distribution of sources (lower-energy cosmic
rays have an isotropic distribution).  For such a distribution, the expected
number of rays would be proportional to the {\em total exposure}, the
integral of the exposure map over the whole sky:
\begin{equation}
\alpha_T \coloneqq \int \expo(\tdrxn) d\tdrxn = \int A(t) dt \int m(\tdrxn) d\tdrxn.
\label{ET-def}
\end{equation}
The total exposure has units of area $\times$ time $\times$ solid angle;
(it has also been called ``aperture,'' as ``exposure'' is more traditionally
used for quantities with units of area $\times$ time).  The PAO team
reports $\alpha_T$ for each observing period in PAO-10.

To calculate the exposure map from $\alpha_T$ and $m(\tdrxn)$, define the
integrated projection factor by $M \coloneqq \int d\tdrxn m(\tdrxn)$ (with
units of solid angle).  Then the exposure is
\begin{equation}
\expo(\tdrxn) = \frac{\alpha_T}{M} m(\tdrxn).
\label{expo-mET}
\end{equation}

%  % \section{Likelihood factorization}
\label{app:like}

Consider estimating the probability density function for a sample of $N$
points $\{x_i\}$ in a Euclidean space using a mixture model built from the
parameterized kernel density $f(x;\theta)$ with parameters $\theta$
(typically including at least location and scale parameters).  With the
number of components in the mixture fixed as $K$, the likelihood function
for the set of kernel parameters $\{\theta_k\}$ and (normalized) mixing
weights $\{w_k\}$ is
\be
\like(\{w_k\},\{\theta_k\}) = 
  \prod_{i=1}^N \left[ \sum_{k=1}^K w_k f(x_i;\theta_k) \right].
\label{like-mix}
\ee
The factor associated with a particular datum, $\sum_k w_k f(x_i;\theta_k)$,
may be interpreted two ways. We may consider it to be the value of a single,
complex density function that happens to representable as a weighted sum.
Alternatively, we may consider it to represent a choice of one of the $K$
components, with probability $w_k$, as the source for the datum, which is
then drawn from the component density $f(x_i; \theta_k)$.  We can make
the latter interpretation more explicit by introducing latent labels,
$\{\lambda_i\}$, with $\lambda_i = k$ denoting assignment of datum $i$
to component $k$.  To keep track of the probabilities for particular
assignments in equation~(\ref{like-mix}), we use the labels to
distinguish the values of $w_k$ associated with the various datum
factors, by replacing $w_k$ in the $i$th factor with $w_{\lambda_i}$
(and summing over values of $\lambda_i$ rather than $k$).
Abbreviating $f(x_i;\theta_k)$ by $f_{k,i}$, we can rewrite
the likelihood as follows:
\begin{align}
\like(\{w_k\},\{\theta_k\})
  &= \prod_{i=1}^N \left[ \sum_{\lambda_i=1}^{K} w_{\lambda_i}
          {f}_{\lambda_i, i} \right]\\
  &= \left(\sum_{\lambda_1=1}^{K} w_{\lambda_1}
          {f}_{\lambda_1, 1}\right) \times\cdots\times
     \left(\sum_{\lambda_N=1}^{K} w_{\lambda_N}
          {f}_{\lambda_N, N}\right)\nonumber\\
  &= \sum_{\lambda_1=1}^{K} \cdots \sum_{\lambda_N=1}^{K}
     \prod_{i=1}^N w_{\lambda_i} {f}_{\lambda_i, i}\nonumber\\
  &= \sum_{\lambda_1\ldots\lambda_N}
     \left( \prod_k w_k^{m_k(\lambda)}\right)
    \prod_i {f}_{\lambda_i, i},\nonumber
\label{product-sum}
\end{align}
where for the last line we have collected factors of a particular weight by
introducing a multiplicity function, $m_k(\lambda)$, counting the number of
times component index $k$ appears in the list of $N$ labels $\lambda_i$.
This dual representation is well known in the literature on FMMs
(see, e.g., \cite{BG88-BayesFMM}).

Turning now to the cosmic ray likelihood function in
equation~(\ref{like-poisson}), the product factor has the algebraic form of
a FMM likelihood, with the source fluxes, $F_k$, playing the role of
mixing weights (but now no longer normalized).  Equation~(\ref{like-assoc})
then follows by the same manipulations as shown above, with the
addition of splitting the exponentiated sum in equation~(\ref{like-poisson})
into separate $e^{-F_k\epsilon_k}$ factors, grouped with with their
associated $F_k^{m_k(\lambda)}$ factors.  The resulting likelihood
function essentially corresponds to a Poissonized FMM.

%\section{Computational Methods}
\label{app:compn}


%..............................................................................
\subsection{Algorithm for Markov chain Monte Carlo}
\label{sec:MCMC}

\mnote{Clarifications, mostly for arXiv readers}
To draw posterior samples, we perform Metropolis-within-Gibbs sampling on
parameters $f,F_T$ and $\lambda$, using Gibbs sampling for $F_T$ and
$\lambda$, and Metropolis sampling for $f$.  The Gibbs sampling steps
alternate between sampling from the full conditional distribution for $F_T$
(i.e., the distribution given the data and all other parameters), and that for
$\lambda$.  The full conditionals may be derived from (\ref{eq:like}) and the
$(F_T,f)$ priors.  The conditional for the total flux is
\be
F_T|f,\lambda,D \sim 
  \text{Gamma}\left(N_C+1,
    \frac{1}{\frac{1}{s}+(1-f)\epsilon_0+f\sum_{k\geq 1}w_k\epsilon_k}\right),
\ee
where $\text{Gamma}(\alpha,s)$ denotes the gamma distribution with shape
parameter $\alpha$ and scale parameter $s$.  (Note that this distribution
happens to be independent of $\lambda$.)  The conditional for the
cosmic ray labels is a multinomial distribution with probabilities
\be
P(\lambda_i|F_T,f,D)
  \propto \frac{f_{\lambda_i,i}}{\epsilon_{\lambda_i}}\times h_{\lambda_i},
    \text{ where } h_{j} =
\begin{cases}(1-f)\epsilon_0 & \text{if $j=0$,}\\
  fw_j\epsilon_j &\text{if $j\geq 1$.}
\end{cases}.
\ee
Finally, the conditional for the associated fraction is
\ba \quad
P(f|\lambda,F_T,D)
  &\propto& \exp\left\{-F_T\left[(1-f)\epsilon_0+f\sum_{k\geq1}\epsilon_k w_k\right]\right\}\nonumber \\
  & & \times (1-f)^{m_0(\lambda)+b-1}f^{N_C-m_0(\lambda)+a-1}.
\ea
$F_T$ and $\lambda$ are sampled directly from the gamma and multinomial
distributions.  $f$ is sampled using a random walk Metropolis algorithm with
Gaussian proposals centered around the current value of $f$.
The variance of the Gaussian proposal density was tuned so that the
acceptance rate was about 25$\%$.

%..............................................................................
\subsection{Marginal Likelihood and Bayes Factor Computation}
\label{sec:Chib}

Following Chib (1995), we can write the marginal likelihood for $\kappa$ as
\be
\like_m(\kappa) = 
  \frac{P\left(D|F_T^*,f^*,\lambda^*\right)
        P\left(\lambda^*|F_T^*,f^*\right)P\left(F_T^*\right)
        P\left(f^*\right)}
       {P\left(F_T^*,f^*,\lambda^*|D\right)} \qquad ||\kappa,
\ee
where the double solidus indicates all the probabilities additionally
condition on $\kappa$.
Here $F_T^*, f^*, \lambda^*$ are in principle arbitrary, but in practice
should correspond to a point with high posterior density.  All the
terms in the numerator can be computed analytically, using the priors and
the likelihood from equation~(\ref{eq:lik-lambda}).  The denominator can be
expressed as
\be
P\left(F_T^*,f^*,\lambda^*|D\right) =
  P(f^*|F_T^*,\lambda^*,D)P(F_T^*|\lambda^*,D)P(\lambda^*|D)
   \qquad ||\kappa.
\ee
The first term on the right hand side is simply the full condition of $f^*$
evaluated at $F_T^*$ and $\lambda^*$.  Note that the normalizing constant can
be computed using numerical integration.  The remaining two terms need to be
estimated using MCMC and can be done as follows:
\be
P(F_T^*|\lambda^*,\kappa,D) \approx
  \frac{1}{G} \sum_{g=1}^G P(F_T^*|f'^{(g)},\lambda^*,\kappa,D),
\ee
\be
P(\lambda^*|\kappa,D) \approx
  \frac{1}{G} \sum_{g=1}^G P(\lambda^*|f^{(g)},F_T^{(g)},\kappa,D).
\ee
Here $G$ denotes the number of iterations.  $f^{(g)}$ and $F_T^{(g)}$ denote
the sample from the MCMC in iteration $g$.  $f'^{(g)}$ is a sample from a new
MCMC run using the full conditionals given earlier with $\lambda$ fixed at
$\lambda^*$ in iteration $g$.

For each $\kappa$ of interest, we first ran 5,000 iterations of Metropolis
within Gibbs to obtain the high posterior density values, $F_T^*, f^*,
\lambda^*$.  Then, for each $\kappa$, 3 additional chains of Gibbs sampling
were run, each with 10,000 iterations.  For subsequent calculations and
plots, the chains were thinned, so that the lag-one autocorrelation is at
most 0.15 for all parameters $F_T, f, \lambda$.

We diagnosed convergence by visually inspecting the 
trace plots from different chains, and by computing the Gelman-Rubin potential 
scale reduction statistic for chains of samples of continuous parameters
such as $f$ and $F_T$.  For the discrete $\lambda$ parameters, we computed
the fraction of time that each cosmic ray is assigned to each source
during every 10 iterations. The diagnostics were done on these fractions
similarly to the continuous parameters.

To validate our algorithms (including our convergence criteria) we developed
an enumerative algorithm that can directly calculate several posterior
quantities for simplified models of small catalogs in the small-deflection
regime, via a guided traversal of a tree of possible associations that has
been thresholded to eliminate associations with negligible probability.  We
compared results from this deterministic algorithm with our MCMC results. We
also used simulated datasets to verify that marginal distributions produce
credible intervals of probability $P$ that have prior-averaged coverage
equal to $P$, a simplified version of the validation tests proposed by Cook,
Gelman, and Rubin \cite{CGR06-Validn}.


%\section{Cen~A single-source model}
\label{app:CenA}

Recent PAO results on the chemical composition of UHECRs, cited above,
suggest that UHECRs may be predominantly heavy nuclei, which would
suffer large magnetic deflections.  Some investigators have suggested
that UHECRs are all heavy nuclei from a single source---the nearest AGN,
Cen~A---with the apparent approximate isotropy of arrival directions
a consequence of strong deflection \cite{B+09-CenA,GBdS10-CenA,BdS12-CenA}.

As a simple test of this idea, we studied a model corresponding to our
buckshot AGN association model, but with a single candidate source, Cen~A, and
with the association fraction $f=1$.  Figure~\ref{fig:BF-CenA} shows the Bayes
factor comparing such a model to the isotropic background model, conditional
on $\kappa$, for various partitions of the data.  For $\kappa=0$, the Cen~A
model makes the same predictions as the background model, and the Bayes factor
is unity.  As $\kappa$ increases, the Bayes factor never grows significantly
larger than unity, so the Cen~A model is not supported by any partition of the
data, for any $\kappa$.  For partitions of the data including the 42 Period~3
events, the Bayes factor opposes the Cen~A model, strongly ruling out models
with $\kappa\simgreat 0.5$, i.e., with deflection angular scales $<90^\circ$.

\begin{figure}
\centerline{\includegraphics[angle=-90,width=\textwidth]{BF-CenAvsIsotropic-SmallKappa.eps}}
\caption{Bayes factors comparing a model Cen~A to be the source of
all UHECRs against the null isotropic background model, conditional
on $\kappa$, shown as a function of $\kappa$.  Results are
shown for various partions of the data (identified by line style,
identified in the legend).  Right panel expands the low-$\kappa$ region.}
\label{fig:BF-CenA}
\end{figure}

The $90^\circ$ scale is consistent with the expected angular scale for
deflection by a regular magnetic field in equation~(\ref{dflxn-reg}) as long
as $Z$ is large ($\simgreat 15$); but note that this equation is valid only in
the small-deflection limit.  The $90^\circ$ scale is consistent with turbulent
magnetic deflection only for the heaviest nuclei (e.g., for iron $Z=26$) and
for large magnetic field and length scales.  For these astrophysical
parameters, the Cen~A model remains disfavored, although not overwhelmingly. 
We thus consider the PAO-10 data to argue against the hypothesis that all
UHECRs originate from Cen~A.

%%..............................................................................
\section{Comparison with prior Bayesian work}
\label{app:WMJ11}

Watson, Mortlock, and Jaffe \cite{WMJ11-BayesUHECR} (WMJ11) have
independently developed an approximate Bayesian approach for assessing
evidence for association of the PAO-08 data (with data for the 27 UHECRs in
periods 1 and 2) with nearby AGN.  They consider the best-fit CR directions
as points drawn from a Poisson intensity on the celestial sphere constructed
as a mixture of distributions located at AGN directions, with weights
reflecting the AGN distances (corresponding to a standard-candle
source intensity), and an isotropic background component.  The source
components have the form of a ``two-dimensional Gaussian on the sphere,''
with the probability density for a best-fit measured direction $\hat n$
arising from an AGN at $\hdrxn$ given by
\be
p(\hat n|\hdrxn)
  = \frac{1}{2\pi\left(1 - e^{-2/\sigma^2}\right)}
    \exp\left(-\frac{1 - \hat n\cdot\hdrxn}{\sigma^2}\right),
\label{sphere-gauss}
\ee
with $\sigma$ dubbed the ``smearing angle.'' This is a Fisher distribution
(as in equation~\ref{rho-def}), with concentration parameter $\kappa =
1/\sigma^2$.  They focus on a model with  $\sigma = 3^\circ$ ($\approx
0.052$~rad) [sic], intended to reflect a combination of $\approx 1^\circ$
measurement uncertainties and few-degree magnetic deflections.  They also
provide a few results for $\sigma = 6^\circ$ and $10^\circ$ [sic].\footnote{We
note that the parameter $\sigma$ is not an angle and does not have angular
units (degrees or radians).  We presume these dimensionally inconsistent
equations imply solving the nonlinear equation relating $\sigma$ (or $\kappa$)
to the stated angular scale; e.g., for a 68.3\% confidence or credible region
with angular radius $\theta$, in the small-angle limit $\sigma \approx
0.66\theta$ (e.g., $\sigma \approx 0.035$ for $3^\circ$ uncertainties;
see equation~(\ref{kappa-theta})).}
They calculate an approximate likelihood function by finely pixelizing the
celestial sphere, calculating the number of cosmic ray direction
measurements expected in each pixel, and multiplying Poisson counting
probabilities for the bins (with one count for bins containing a best-fit
cosmic ray direction, and zero counts for the remaining bins).

WMJ11 adopt a similar candidate host population as was used in the PAO-07 and
PAO-08 analyses (the nearby AGN in the 12th VCV compilation; WMJ11 consider
$\approx 900$ AGN within 100~Mpc) on the presumption that it is almost
complete for the nearest AGN.  They conclude that there is ``strong evidence
of a UHECR signal from the known VCV AGNs,'' such that at least some UHECRs
come from AGN in the VCV catalog (or from sources within a few degrees of the
AGN).  For a $3^\circ$ smearing angle, the marginal posterior density for the
fraction associated with AGN (our $f$ parameter) has a mode of 0.15, and 68\%
highest density credible interval of $[0.08, 0.25]$.  For $6^\circ$ and
$10^\circ$ smearing angles the estimated association fraction is larger, but
the marginal distribution is also somewhat broader.  They do not compare their
association model to an alternative, and thus do not compute Bayes factors.

Our analysis framework and our results differ in significant ways from those
of WMJ11 (focusing on our results for the data from Periods~1 and 2). 
Methodologically, our approach is based on explicit modeling of associations
(via marginal likelihood factors associating a particular cosmic ray with a
particular AGN or the background), rather than considering best-fit cosmic
ray directions to be samples from a point process.  As noted in
\S~\ref{sec:dtxn}, a factor in the likelihood function in our approach
is analogous to that underlying FMMs, but this
mixture-like factor arises as a consequence of some of our modeling
assumptions; it is not a starting point, and it does not hold for all
astrophysically interesting association models that our framework
accommodates.  For example, in \S~\ref{sec:summary} we describe a more
realistic family of magnetic deflection models (``radiant'' models with
exchangeable rather than IID deflections for cosmic rays comprising a
multiplet) whose likelihood function does not have the simple FMM form.  In
other common astronomical coincidence assessment problems, such as
establishing associations of sources detected in different electromagnetic
wavebands, only singlet associations are meaningful; such models have no FMM
representation but may be accommodated by our framework.  Further
discussion of this is in \cite{Loredo12-Coinc}.

Another point of departure in methodology is that we distinguish measurement
error from magnetic deflection.  In the buckshot model adopted here, both
the measurement error and deflection distributions are Fisher distributions
(note that the composition of these distributions is not a Fisher
distribution).  In radiant models, for example, the deflection distribution
is more complicated.  Explicit, separate treatment of these physically
distinct effects enables handling heteroskedastic measurement errors
for the UHECR directions (PAO-10 reports only a typical measurement
error, but future catalogs will hopefully report the heteroskedastic
uncertainties found in detailed air shower fits).  Heteroskedastic
uncertainties further thwart a simple FMM representation for the likelihood,
again emphasizing the need for a framework built on explicit modeling
of individual associations.

Our approach also provides explicit estimates of probabilities for possible
associations.  In our MCMC algorithm, these can be found by calculating
frequencies for different values of the $\lambda$ labels; we report
such results in \S~\ref{sec:results}.  WMJ11 report a weight for
candidate associations, but note it is not a rigorous probability.

WMJ11 analyze the data from Periods~1 and 2 jointly, without commenting on the
possible effects of tuning on the implications of the Period~1 data.

Turning to astrophysical differences, we adopt the G10
\cite{2010MNRAS.406..597G} volume-complete catalog of nearby AGN as a
candidate host catalog, rather than the 12th VCV catalog used by WMJ11 and
others.  Notably, 6 of the 17 AGN in the G10 catalog are not in the 12th VCV
catalog (one of them appears in the more recent 13th VCV catalog).  We chose
the G10 catalog both for its completeness, and because use of a small
catalog was convenient for an initial study, as it enabled more extensive
analyses of real and simulated data than would be possible with AGN from
VCV. Figure~\ref{fig:f1000} shows our marginal posterior density for $f$ for
$\kappa = 1000$ (corresponding to a deflection scale $\approx 2.7^\circ$)
for model $M_1$, for different subsets of the PAO-10 data. The mode based on
the combined data from Periods~1 and 2 is at $f \approx 0.1$, about a
two-thirds of the WMJ11 value of $\approx 0.15$, even though their catalog
contains more than 50 times as many potential counterparts.  Our common
assumption of a standard candle intensity distribution is probably the main
reason that the results are not more discrepant. In particular, although
WMJ11 include AGN at distances to 100~Mpc in their catalog, the standard
candle assumption forces the analysis to assign negligible detectable rates
to all but the closest few AGN.  The similarity of our estimates despite the
disparity between our AGN catalog sizes highlights how restrictive the
standard candle assumption is.

\begin{figure}
\centerline{\includegraphics[angle=-90,width=.9\textwidth]{margf_kappa1000_17AGNs.eps}}
\caption{Posterior distributions for $f$ for model $M_1$, conditioned on
$\kappa$ = 1000, for various subsets of the PAO-10 data.}
\label{fig:f1000}
\end{figure}

We explore a far greater range of magnetic deflection angular scales
than did WMJ11.  Their discussion of deflection scales implicitly
presumes UHECRs are light nuclei.  Recent cosmic ray data and
theoretical models motivate serious consideration of the possibility
that many or most UHECRs are heavy nuclei, as noted above.

Finally, we differ qualitatively in our conclusions about the strength of
evidence for association of UHECRs with nearby AGN, particularly after
examining period-to-period differences, and considering the impact of
period~3 data (unavailable to WMJ11).  We quantify the support for
association hypotheses by calculating Bayes factors explicitly comparing
association and null models, both conditional on $\kappa$ (in
Figure~\ref{fig:BFplot}) and marginalized over a broad $\kappa$
range (in Table~\ref{BFTable}).

WMJ11 do not calculate Bayes factors comparing their association and null
models.  Their claim of strong evidence for association appears to be based on
the small marginal posterior density for values of the association fraction
near zero.  But this fails to distinguish parameter estimation from
model assessment.  As long as the likelihood for models with $f=0$ does not
vanish, parameter estimation (with a continuous prior on $f$) simply does not
address how the $f=0$ hypothesis compares to alternatives.  To do this
requires assigning a finite prior probability for $f=0$, which we do here by
considering it as a separate model and calculating Bayes factors.  The Bayes
factor comparing nested models depends on the size of the parameter space of
the larger model in a way that accounts for ``fine tuning'' of the additional
model parameters:  the larger model will have parameter values producing
better fits than the smaller model, but if the values of the additional
parameters are close enough to the default values corresponding to the smaller
model, the marginal likelihood for the larger model will be {\em smaller} than
that for the smaller model (the well-known ``Ockham's razor'' behavior of
Bayes factors).  Focusing on low-dimensional marginal distributions, such as
the posterior density for $f$, can give an exaggerated impression of the
strength of evidence for the larger model because it suppresses the large
volume of parameter space associated with its additional parameters.  Here,
association models have not only the $f$ parameter, but also many latent label
parameters (i.e., many association hypotheses that cannot be ruled out a
priori).  Calculation of the Bayes factor takes all of this into account.
Using the Period~1 and Period~2 data available to WMJ11 does in fact produce
large Bayes factors favoring association.  But partitions of the data
excluding Period~1 data produce much smaller Bayes factors, even though the
$p(f)$ distributions found with these partitions assign very small density to
$f=0$. The Bayes factor calculations indicate that the complexity of
association models may not be justified by existing data.

Perhaps most importantly from an astrophysical perspective, we performed
more extensive checking of our models, calling into question the adequacy of
our shared isotropic background and standard candle assumptions.  We discuss
this further in \S~\ref{sec:summary}.


%\section{Model checking}
\label{app:checking}

In this Appendix we describe results of two types of tests of our models.
Both are motivated by the evident variability of some of the parameter
estimation and model comparison results presented in \S~\ref{sec:results}
with the choice of observing period or periods to include in the analysis.

Our first tests address whether the variability indicates that the
properties of the detected cosmic rays change from period to period,
presuming that one of our models can adequately describe the data within
each period.  We implement a simple Bayesian change-point analysis that shows
there is no significant evidence for variability of model parameters from
period to period.  That is, presuming one of the models is adequate, the
apparent discrepancy among the Bayes factors in Table~\ref{BFTable} reflects
variability that may be expected for these modest sample sizes.

Alternatively, we may consider the possibility that none of the considered
models accurately describes the data, in which case the variability could be
an indication of incompatibility of the data with {\em all} of the models.
To address this, we perform predictive checks using the Bayes factors for
subsamples of the data as test statistics:  we ask whether the
period-to-period Bayes factor variations we find using the observed data
are surprising compared with what one finds from simulations under various
models.  We first compare the observed Bayes factors with predictions from the
null model, and then from representative association models.  These tests
are meant to explore broad compatibility of predictions and
observations; we make no attempt to formally assign significances to the
comparison such as $p$-values.

%..............................................................................
\subsection{Change-point analysis}

As seen above, including the data from Period~1 can change Bayes factors
dramatically (but not estimates of $F_T$, $f$, or $\kappa$); also, Bayes
factors differ markedly even between the {\em untuned} samples, Periods~2
and 3.  Based on astrophysical considerations, the properties of incident
UHECRs should not vary over the time scales under consideration.  Even if CRs
are generated in brief bursts, these bursts would have observed durations of
hundreds or thousands of years because the cosmic rays would take paths of
varying lengths to Earth.  Therefore, any evidence indicating that the
observed properties of cosmic rays are changing over a period of a few years
would indicate a problem with the data, e.g., statistical inhomogeneity due to
the special treatment of data in period~1, or instability of the observatory's
apparatus or data reduction pipeline.

As a simple test for variability in the ensemble properties of cosmic rays
from period to period, we compared versions of $M_1$ and $M_2$ that allow
model parameters to change between periods to versions that keep the
parameters the same for all periods.  That is, we
explore change-point models, with the change point locations at period
boundaries.  Specifically, we compute 3 quantities:
\ba
B_{(1)(23)} &=& \frac{\like_1 \like_{23}}{\like_{123}}\nonumber\\
B_{(2)(3)} &=& \frac{\like_2 \like_3}{\like_{23}}\nonumber\\
B_{(1)(2)(3)} &=& \frac{\like_1\like_2 \like_3}{\like_{123}}
\ea
where $\like_{i_1,\ldots,i_q}$ is the marginal likelihood computed using
the Chib estimate based on data from periods $i_1,\ldots,i_q$.
For example, $B_{(1)(23)}$ compares a model allowing parameters to differ
between period~1 and the later periods, to a model with common parameter
values across all periods.  Note that $B_{(2)(3)}$ considers only untuned
data.  We compute $B_{(1)(23)}, B_{(2)(3)}$ and $B_{(1)(2)(3)}$ for both the
17 AGN and 2 AGN association models. Figure~\ref{fig:changepoint} shows
these Bayes factors as functions of $\kappa$.

\begin{figure}
\centerline{$
\begin{array}{cc}
\includegraphics[angle=-90,width=.5\textwidth]{BF_changepoint_17AGNs.eps} &
\includegraphics[angle=-90,width=.5\textwidth]{BF_changepoint_2AGNs.eps}
\end{array}$}
\caption{Bayes factors for change-point models.}
\label{fig:changepoint}
\end{figure}

In the case of 17 AGN, for change-point models considering all of the data,
the Bayes factors stay within [1/3,3] indicating no preference for one model
over the other.
The same is true for the 2~AGN model, except for
$\kappa > 300$, where there is a modest preference for models with
consistent parameter values across all periods.

We also considered change point models that partition the data between
(joint) association models and the null model, aiming to assess the
possibility that the data are consistent with isotropy in some intervals,
but with association (possibly spurious) in others.  We again found
Bayes factors to be equivocal.  

%..............................................................................
\subsection{Null model predictive checks}

If the null model is in operation, a measure of surprise for a particular data
set would be a high Bayes factor favoring an association model.  Of particular
concern here are the large Bayes factor values we find for the Period~1 data,
which may not be representative because of tuning.  We ask:  under the null
isotropic background model and in the absence of tuning, how likely it is to
see Bayes factors as high as 100 for some values of $\kappa$ in Period~1 (as
shown in Figure~\ref{fig:BFplot})?

To address this, we generate 200 datasets, each of which has 14 CRs (the
size of the Period~1 sample).  The CR
directions are generated uniformly over the sky, and are accepted with
probability proportional to the exposure map for that direction.  We
calculated Bayes factors for the 17~AGN association model with $\kappa = 10,
31.62, 100, 316.2$ and 1000.
Figure~\ref{fig:unifCumBF} shows cumulative histograms (as survival
functions) of the resulting Bayes factors, for each value of $\kappa$.  Most
of the datasets have Bayes factors less than 1. The smallest Bayes factor we
found is 0.04. Out of these 200 datasets, we found only 3 datasets with
$B_{10} \geq 10$.  Each of these datasets has $B_{10} \geq 10$ only at one
value of $\kappa$ that we computed. The Bayes factors are 29 ($\kappa=10$),
44 ($\kappa = 31.62$) and 13 ($\kappa = 316.2$).  We conclude that the Bayes
factor greater than 100 seen in Period~1 is unlikely to be due to chance if
UHECRs are fairly sampled from isotropically distributed directions.  This
implies the distribution of directions in the Period~1 sample is
anisotropic, but the calculation does not address whether this may be due to
tuning or to genuine anisotropy.

\begin{figure}
\centerline{\includegraphics[angle=-90,width=\textwidth]{BF_cumplot_genUnif_14CRs_logscale.eps}}
\caption{The number of simulated datasets having Bayes factor $\geq B$
vs. $B$, based on 200 simulated datasets with 14 CRs generated under the
null isotropic background model.  Axes are logarithmic.}
\label{fig:unifCumBF}
\end{figure}

%..............................................................................
\subsection{Association model predictive checks}

Now we address compatibility of the discrepant Bayes factors with the
predictions of the association models.  We ask: under the association model
with 17 AGN, how likely is it to simultaneously see large Bayes factors in
periods 1 or 2 (which have small numbers of detected CRs) and a small Bayes
factor in Period~3 (which has a larger number of detected CRs)?

To address this, we generate 100 datasets, each with 69 CRs, partitioned
into samples of size 14, 13 and 42 for periods 1, 2 and 3, respectively.  We
simulate from a model with association fraction $f=0.1$, near the modes
found in our analyses of the PAO data.  We first simulate an incident flux
of cosmic rays by assigning a ray to the AGN population with probability
equal to $f$, or to the isotropic background with probability $1-f$.
AGN-generated CRs get assigned to one of the 17 AGN with probabilities
proportional to the inverse square distances of the AGN. The arrival
directions of these CRs are drawn from a Fisher distribution centered at the
source AGN, with $\kappa = 50$.  Each generated CR direction is accepted
with probability proportional to the exposure map for that direction; a
dataset is generated when 69 candidate events are accepted.  Subsets
of each simulated dataset, with sizes corresponding to those of the
PAO subsamples, were analyzed with models conditioning on various
values of $\kappa$.  Figure~\ref{fig:assocCumBF} shows cumulative
histograms of the resulting Bayes factors.

Out of 100 simulated datasets, we found 17 that have some of the Bayes
factors $\geq 30$.  Of these, 9 datasets have $B_{10}\geq 30$ in either
Period~1 or 2, and a low $B_{10}$ value in Period~3, while the other 8
datasets have $B_{10}\geq 30$ in Period~3 and low $B_{10}$ values in periods
1 and 2.  Out of these 17 datasets, 10 of them have some values of
$B_{10}\geq 100$; two have Bayes factors over 1000 (in either Period~1
or Period~2).  
These results indicate that Bayes factors as high as 100 (as seen in Period~1)
may be expected a few percent of the time under the association model, and
also that large between-period variations in the Bayes factors should be
expected.  However, since Bayes factors $\gta 10$ are uncommon, and since the
distribution is independent from one interval to the next, it is unlikely to
see significantly large Bayes factors in {\em both} periods 1 and 2.  We
crudely estimate the probability for seeing Bayes factors whose product
is $> 1000$ as less than $\sim 10^{-3}$.  Thus the pattern of Bayes factors is
surprising, again indicating that the tuning of the Period~1 data is
problematic.

\begin{figure}
\centerline{\includegraphics[angle=-90,width=\textwidth]{BFCumplot-AssocnSimn.eps}}
\caption{The number of simulated datasets having Bayes factor $\geq B$
vs. $B$, based on 100 simulated datasets with CRs generated under the
17-AGN association model with $f=0.1$ and $\kappa=50$.  Panels show
results for datasets with sizes corresponding to the PAO samples for
periods 1--3, as labeled; colors of curves indicate the $\kappa$ value
used for analyzing the simulated data.  Axes are logarithmic.}
\label{fig:assocCumBF}
\end{figure}



\section*{Acknowledgements}
This research was supported by the NSF via grants AST-0908439 and DMS-0805975.
We are grateful to Paul Sommers for helpful conversations about the
PAO instrumentation and data reduction and analysis processes.

% Descriptions of online supplements go here:

\begin{supplement}[id=supp]
  \sname{Supplement}
  \stitle{Technical Appendices}
  \slink[url]{http://lib.stat.cmu.edu/aoas/???/???}
  \sdescription{The online supplement contains five technical appendices
  with detailed material on the following topics:
\begin{itemize}
  \item[A.] Auger observatory exposure;
  \item[B.] Computational methods;
  \item[C.] Cen~A single-source model;
  \item[D.] Comparison with prior Bayesian work;
  \item[E.] Model checking.
\end{itemize}}
\end{supplement}


\bibliographystyle{plain}
\newcommand{\apj}{Astrophysical Journal}
\newcommand{\apjl}{Astrophysical Journal Letters}
\newcommand{\mnras}{Monthly Notices of the Royal Astronomical Society}
\newcommand{\jcap}{Journal of Cosmology and Astroparticle Physics}

\bibliography{UHECR}

% AOS,AOAS: If there are supplements please fill:
%\begin{supplement}[id=suppA]
%  \sname{Supplement A}
%  \stitle{Title}
%  \slink[url]{http://lib.stat.cmu.edu/aoas/???/???}
%  \sdescription{Some text}
%\end{supplement}

\end{document}

Chib, S. (1995) Marginal likelihood from the Gibbs Output, JASA, 90,
1313—1321.

Newton, M. A., and Raftery, A. E. (1994) Approximate Bayesian inference by the
weighted likelihood bootstrap, JRSS-B, 56, 3-48.

Cook, Gelman, Rubin 2006

