\section{Model checking}\label{sec:checking}

In this section we describe results of two types of tests of our models.
Both are motivated by the evident variability of some of the parameter
estimation and model comparison results presented above with the choice of
observing period or periods to include in the analysis.

Our first tests address whether the variability indicates that the
properties of the detected cosmic rays change from period to period,
presuming that one of our models can adequately describe the data within
each period. We implement a simple Bayesian change-point analysis that shows
there is no significant evidence for variability of model parameters from
period to period.  That is, presuming one of the models is adequate, the
apparent discrepancy among the Bayes factors in Table~\ref{BFTable} reflects
variability that may be expected for these modest sample sizes.

Alternatively, we may consider the possibility that none of the considered
models accurately describes the data, in which case the variability could be
an indication of incompatibility of the data with {\em all} of the models.
To address this, we perform predictive checks using the Bayes factors for
subsamples of the data as test statistics:  we ask whether the
period-to-period Bayes factor variations we find using the observed data
are surprising compared with what one finds from simulations under various
models.  We first compare the observed Bayes factors with predictions from the
null model, and then from representative association models.  These tests
are meant to explore broad compatibility of predictions and
observations; we make no attempt to formally assign significances to the
comparison such as $p$-values.

%..............................................................................
\subsection{Change-point analysis}
%\label{sec:difference_between_periods}

As seen above, including the data from Period~1 can change Bayes factors
dramatically (but not estimates of $F_T$, $f$, or $\kappa$); also, Bayes
factors differ markedly even between the {\em untuned} samples, Periods~2
and 3.  Based on astrophysical considerations, the properties of incident
UHECRs should not vary over the time scales under consideration.  Even if CRs
are generated in brief bursts, these bursts would have observed durations of
hundreds or thousands of years because the cosmic rays would take paths of
varying lengths to Earth.  Therefore, any evidence indicating that the
observed properties of cosmic rays are changing over a period of a few years
would indicate a problem with the data, e.g., statistical inhomogeneity due to
the special treatment of data in period~1, or instability of the observatory's
apparatus or data reduction pipeline.

As a simple test for variability in the ensemble properties of cosmic rays
from period to period, we compared versions of $M_1$ and $M_2$ that allow
model parameters to change between periods to versions that keep the
parameters the same for all periods.  We describe implementation and results
of this change-point analysis in In Supplementary Appendix~\ref{app:chgpt}.  
It shows that the period-to-period variation of Bayes factors for
association vs.\ isotropy is consistent with no variation of cosmic ray
properties across period boundaries.

%..............................................................................
\subsection{Null model predictive checks}

If the null model is in operation, a measure of surprise for a particular data
set would be a high Bayes factor favoring an association model.  Of particular
concern here are the large Bayes factor values we find for the Period~1 data,
which may not be representative because of tuning.  We ask:  under the null
isotropic background model and in the absence of tuning, how likely it is to
see Bayes factors as high as 100 for some values of $\kappa$ in Period~1 (as
shown in Figure~\ref{fig:BFplot})?

To address this, we generate 200 datasets, each of which has 14 CRs (the
size of the Period~1 sample).  The CR
directions are generated uniformly over the sky, and are accepted with
probability proportional to the exposure map for that direction.  We
calculated Bayes factors for the 17~AGN association model with $\kappa = 10,
31.62, 100, 316.2$ and 1000.
Figure~\ref{fig:unifCumBF} shows cumulative histograms (as survival
functions) of the resulting Bayes factors, for each value of $\kappa$.  Most
of the datasets have Bayes factors less than 1. The smallest Bayes factor we
found is 0.04. Out of these 200 datasets, we found only 3 datasets with
$B_{10} \geq 10$.  Each of these datasets has $B_{10} \geq 10$ only at one
value of $\kappa$ that we computed. The Bayes factors are 29 ($\kappa=10$),
44 ($\kappa = 31.62$) and 13 ($\kappa = 316.2$).  We conclude that the Bayes
factor greater than 100 seen in Period~1 is unlikely to be due to chance if
UHECRs are fairly sampled from isotropically distributed directions.  This
implies the distribution of directions in the Period~1 sample is
anisotropic, but the calculation does not address whether this may be due to
tuning or to genuine anisotropy.

\begin{figure}
\centerline{\includegraphics[angle=-90,width=\textwidth]{BF_cumplot_genUnif_14CRs_logscale.eps}}
\caption{The number of simulated datasets having Bayes factor $\geq B$
vs. $B$, based on 200 simulated datasets with 14 CRs generated under the
null isotropic background model.  Axes are logarithmic.}
\label{fig:unifCumBF}
\end{figure}

%..............................................................................
\subsection{Association model predictive checks}

Now we address compatibility of the discrepant Bayes factors with the
predictions of the association models.  We ask: under the association model
with 17 AGNs, how likely is it to simultaneously see large Bayes factors in
periods 1 or 2 (which have small numbers of detected CRs) and a small Bayes
factor in Period~3 (which has a larger number of detected CRs)?

To address this, we generate 100 datasets, each with 69 CRs, partitioned
into samples of size 14, 13 and 42 for periods 1, 2 and 3, respectively.  We
simulate from a model with association fraction $f=0.1$, near the modes
found in our analyses of the PAO data.  We first simulate an incident flux
of cosmic rays by assigning a ray to the AGN population with probability
equal to $f$, or to the isotropic background with probability $1-f$.
AGN-generated CRs get assigned to one of the 17 AGN with probabilities
proportional to the inverse square distances of the AGN. The arrival
directions of these CRs are drawn from a Fisher distribution centered at the
source AGN, with $\kappa = 50$.  Each generated CR direction is accepted
with probability proportional to the exposure map for that direction; a
dataset is generated when 69 candidate events are accepted.  Subsets
of each simulated dataset, with sizes corresponding to those of the
PAO subsamples, were analyzed with models conditioning on various
values of $\kappa$.  Figure~\ref{fig:assocCumBF} shows cumulative
histograms of the resulting Bayes factors.

Out of 100 simulated datasets, we found 17 that have some of the Bayes
factors $\geq 30$.  Of these, 9 datasets have $B_{10}\geq 30$ in either
Period~1 or 2, and a low $B_{10}$ value in Period~3, while the other 8
datasets have $B_{10}\geq 30$ in Period~3 and low $B_{10}$ values in periods
1 and 2.  Out of these 17 datasets, 10 of them have some values of
$B_{10}\geq 100$; two have Bayes factors over 1000 (in either Period~1
or Period~2).  
\mnote{Modified discussion}
These results indicate that Bayes factors as high as 100 (as seen in
Period~1) are not unlikely under the association model, and also that
large between-period variations in the Bayes factors should be expected.
However, since Bayes factors $\gg 1$ are uncommon, and since the
distribution is independent from one interval to the next, it is
unlikely to see significantly large Bayes factors in {\em both} periods
1 and 2.  We crudely estimate that the probability for seeing Bayes
factors whose product is $> 1000$ is less than $\sim 10^{-3}$.  Thus the
pattern of Bayes factors is surprising, again indicating that the tuning
of the Period~1 data is problematic.

\begin{figure}
\centerline{\includegraphics[angle=-90,width=\textwidth]{BFCumplot-AssocnSimn.eps}}
\caption{The number of simulated datasets having Bayes factor $\geq B$
vs. $B$, based on 100 simulated datasets with CRs generated under the
17-AGN association model with $f=0.1$ and $\kappa=50$.  Panels show
results for datasets with sizes corresponding to the PAO samples for
periods 1--3, as labeled; colors of curves indicate the $\kappa$ value
used for analyzing the simulated data.  Axes are logarithmic.}
\label{fig:assocCumBF}
\end{figure}
