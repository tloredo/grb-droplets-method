\section{Change-point analysis}
\label{app:chgpt}

We can use the marginal likelihoods for the various periods to investigate
whether parameter values for any of our models vary between time periods. 
We compare models that allow the parameters to change between periods to
models that keep the parameters the same for all periods.  That is, we
explore change-point models, with the change point locations at period
boundaries.  Specifically, we compute 3 quantities:
\ba
B_{(1)(23)} &=& \frac{\like_1 \like_{23}}{\like_{123}}\nonumber\\
B_{(2)(3)} &=& \frac{\like_2 \like_3}{\like_{23}}\nonumber\\
B_{(1)(2)(3)} &=& \frac{\like_1\like_2 \like_3}{\like_{123}}
\ea
where $\like_{i_1,\ldots,i_q}$ is the marginal likelihood computed using
the Chib estimate based on data from periods $i_1,\ldots,i_q$.
For example, $B_{(1)(23)}$ compares a model allowing parameters to differ
between period~1 and the later periods, to a model with common parameter
values across all periods.  Note that $B_{(2)(3)}$ considers only untuned
data.  We compute $B_{(1)(23)}, B_{(2)(3)}$ and $B_{(1)(2)(3)}$ for both the
17 AGN and 2 AGN association models. Figure~\ref{fig:changepoint} shows
these Bayes factors as functions of $\kappa$.

\begin{figure}
\centerline{$
\begin{array}{cc}
\includegraphics[angle=-90,width=.5\textwidth]{BF_changepoint_17AGNs.eps} &
\includegraphics[angle=-90,width=.5\textwidth]{BF_changepoint_2AGNs.eps}
\end{array}$}
\caption{Bayes factors for change-point models.}
\label{fig:changepoint}
\end{figure}

In the case of 17 AGN, for change-point models considering all of the data,
the Bayes factors stay within [1/3,3] indicating no preference for one model
over the other.
%
%\footnote{Note that since $\kappa$
%is allowed to vary independently between intervals, the change point
%models include models that allow some intervals to have a very small
%value of $\kappa$, producing a nearly isotropic distribution of arrival
%directions, which might approximately mimic the distribution of directions
%from unidentified sources out to $\sim 100$~Mpc, the distance scale
%implied by the GZK cutoff.}
%
The same is true for the 2~AGN model, except for
$\kappa > 300$, where there is a modest preference for models with
consistent parameter values across all periods.

We also considered change point models that partition the data between
(joint) association models and the null model, aiming to assess the
possibility that the data are consistent with isotropy in some intervals,
but with association (possibly spurious) in others.  We again found
Bayes factors to be equivocal.  
