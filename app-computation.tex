\section{Computational Methods}
\label{app:compn}


%..............................................................................
\subsection{Algorithm for Markov Chain Monte Carlo}
\label{sec:MCMC}

To draw posterior samples, we perform Metropolis-within-Gibbs sampling on
parameters $f,F_T$ and $\lambda$.  The full conditionals may be derived from 
(\ref{eq:lik}) and the $(F_T,f)$ priors.  The conditional for the total flux is
\be
F_T|f,\lambda \sim 
  \text{Gamma}\left(N_C+1,
    \frac{1}{\frac{1}{s}+(1-f)\epsilon_0+f\sum_{k\geq 1}w_k\epsilon_k}\right),
\ee
where $\text{Gamma}(\alpha,s)$ denotes the gamma distribution with shape
parameter $\alpha$ and scale parameter $s$.  The conditional for the
cosmic ray labels is a multinomial distribution with probabilities
\be
P(\lambda_i|F_T,f,D)
  \propto \frac{f_{\lambda_i,i}}{\epsilon_{\lambda_i}}\times h_{\lambda_i},
    \text{ where } h_{j} =
\begin{cases}(1-f)\epsilon_0 & \text{if $j=0$,}\\
  fw_j\epsilon_j &\text{if $j\geq 1$.}
\end{cases}.
\ee
Finally, the conditional for the associated fraction is
\ba \quad
\left[f|\lambda,F_T,D\right]
  &\propto& \exp\left\{-F_T\left[(1-f)\epsilon_0+f\sum_{k\geq1}\epsilon_k w_k\right]\right\}\nonumber \\
  & & \times (1-f)^{m_0(\lambda)+b-1}f^{N_C-m_0(\lambda)+a-1}.
\ea
$F_T$ and $\lambda$ are sampled directly from the gamma and multinomial
distributions.  $f$ is sampled using a random walk Metropolis algorithm with
Gaussian proposals centered around the current value of $f$.
The variance of the Gaussian proposal density was tuned so that the
acceptance rate was about 25$\%$.

%..............................................................................
\subsection{Marginal Likelihood and Bayes Factor Computation}
\label{sec:Chib}

Following Chib (1995), we can write the marginal likelihood for $\kappa$ as
\be
\like_m(\kappa) = 
  \frac{P\left(D|F_T^*,f^*,\lambda^*\right)
        P\left(\lambda^*|F_T^*,f^*\right)P\left(F_T^*\right)
        P\left(f^*\right)}
       {P\left(F_T^*,f^*,\lambda^*|D\right)} \qquad ||\kappa,
\ee
where the double solidus indicates all the probabilities additionally
condition on $\kappa$.
Here $F_T^*, f^*, \lambda^*$ are in principle arbitrary, but in practice
should correspond to a point with high posterior density.  All the
terms in the numerator can be computed analytically, using the priors and
the likelihood from equation~(\ref{eq:lik-lambda}).  The denominator can be
expressed as
\be
P\left(F_T^*,f^*,\lambda^*|D\right) =
  P(f^*|F_T^*,\lambda^*,D)P(F_T^*|\lambda^*,D)P(\lambda^*|D)
   \qquad ||\kappa.
\ee
The first term on the right hand side is simply the full condition of $f^*$
evaluated at $F_T^*$ and $\lambda^*$.  Note that the normalizing constant can
be computed using numerical integration.  The remaining two terms need to be
estimated using MCMC and can be done as follows:
\be
P(F_T^*|\lambda^*,\kappa,D) \approx
  \frac{1}{G} \sum_{g=1}^G P(F_T^*|f'^{(g)},\lambda^*,\kappa,D),
\ee
\be
P(\lambda^*|\kappa,D) \approx
  \frac{1}{G} \sum_{g=1}^G P(\lambda^*|f^{(g)},F_T^{(g)},\kappa,D).
\ee
Here $G$ denotes the number of iterations.  $f^{(g)}$ and $F_T^{(g)}$ denote
the sample from the MCMC in iteration $g$.  $f'^{(g)}$ is a sample from a new
MCMC run using the full conditionals given earlier with $\lambda$ fixed at
$\lambda^*$ in iteration $g$.

For each $\kappa$ of interest, we first ran 5,000 iterations of Metropolis
within Gibbs to obtain the high posterior density values, $F_T^*, f^*,
\lambda^*$.  Then, for each $\kappa$, 3 additional chains of Gibbs sampling
were run, each with 10,000 iterations.  For subsequent calculations and
plots, the chains were thinned, so that the lag-one autocorrelation is at
most 0.15 for all parameters $F_T, f, \lambda$.

We diagnosed convergence by visually inspecting the 
trace plots from different chains, and by computing the Gelman-Rubin potential 
scale reduction statistic for chains of samples of continuous parameters
such as $f$ and $F_T$.  For the discrete $\lambda$ parameters, we computed
the fraction of time that each cosmic ray is assigned to each source
during every 10 iterations. The diagnostics were done on these fractions
similarly to the continuous parameters.

To validate our algorithms (including our convergence criteria) we developed
an enumerative algorithm that can directly calculate several posterior
quantities for simplified models of small catalogs in the small-deflection
regime, via a guided traversal of a tree of possible associations that has
been thresholded to eliminate associations with negligible probability.  We
compared results from this deterministic algorithm with our MCMC results. We
also used simulated datasets to verify that marginal distributions produce
credible intervals of probability $P$ that have prior-averaged coverage
equal to $P$, a simplified version of the validation tests proposed by Cook,
Gelman, and Rubin \cite{CGR06-Validn}.

