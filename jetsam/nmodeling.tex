\section{Modeling the cosmic ray data}

% We adopt a Bayesian
% approach for directional coincidence assessment based on multilevel modeling,
% where upper levels in a model describe properties of potentially associated
% source populations and radiation propagation, and lower levels describe
% measurement errors and survey selection effects.

The basic statistical problem is to quantify evidence for associating some
number (possibly zero) of cosmic rays with each member of a candidate
source population.  The key observable is the cosmic ray direction; a set of
rays with directions near a putative host comprises a multiplet potentially
associated with that host.  This gives the problem the flavor of model-based
clustering (of points on the celestial sphere rather than in a Euclidean space),
but with some novel features:
\begin{itemize}
\item The model must account for measurement error in cosmic ray properties.

\item Observatories provide an incomplete and distorted sample of cosmic rays,
so the model must account for random truncation and nonuniform thinning.

\item The most realistic astrophysical models imply a joint distribution for
the properties of the cosmic rays assigned to a particular source (the
``multiplet distribution'') that is exchangeable rather than a product of
independent distributions (as is the case in standard clustering).

\item The binomial point process model underlying standard generative
clustering approaches is not appropriate because the number of cosmic rays
is informative about the intensity scale of the cosmic ray sources.
\end{itemize}

We analyze the data in a hierarchical Bayesian framework with
multiple levels that account for these complexities.  At the top level
(Level~1), for a given candidate source population, we posit a distribution
for source intensities (a ``luminosity function'' in astronomical lingo);
we also include as a ``zeroth'' source an isotropic background component
with uncertain intensity.  At Level~2 is a marked Poisson point process
model for latent cosmic ray properties; the incident cosmic ray arrival
times have a homogeneous intensity measure in time, and the marks include
latent ``guide'' directions for the cosmic rays (identical to the
source direction in the simplest models), the cosmic ray energies,
and latent categorical labels identifying the source of each ray.  Level~3
models magnetic deflection of the rays, scattering their directions from the
guide directions.  Finally, Level~4 is a measurement model with directional
uncertainties and accounting for truncation and thinning.

Figure~\ref{fig:levels} schematically depicts the structure of our
framework, including identification of the various random variables
appearing in the calculations described below.  The variables will
be defined as they appear in the detailed development below; the
figure serves as visual reference to the notation.  The figure is
not a graph; rather, the models we explore put probability distributions
over a space of graphs, each graph corresponding to a possible set of
associations of the cosmic rays with particular sources.

\begin{figure}
\centerline{\includegraphics[clip=true,width=\textwidth]{CRCoinc-Levels.eps}}
\caption{Schematic depiction of the levels in our cosmic ray association
models, identifying random variables appearing in each level, including
parameters of interest (bold red labels), latent variables representing cosmic ray
properties that are not directly observable (slant type labels), and
observables (bold blue labels).}
\label{fig:levels}
\end{figure}

We compare models with different source populations, including a ``null''
model assigning all cosmic rays to unresolved sources.  We aim to
(1)~Ascertain which cosmic rays may be associated with specific sources with
high probability; (2)~Estimate luminosity function parameters for
populations of astrophysical sources; (3)~Estimate the proportion of all
detected cosmic rays generated by each population.  We also
phenomenologically model the physical process of cosmic ray deflection by
magnetic fields.  We aim to (4)~Estimate parameters describing the effects
of cosmic magnetic fields; (5)~Investigate whether cosmic rays from a single
source are scattered independently (which we call a ``buckshot model'') or
share part of their scattering history (an exchangeable ``radiant model'').
Aims (2) and (5) are not attempted here but will be investigated in the future.

%..............................................................................
\subsection{Cosmic ray production}

% Note that the cosmic rays from a particular source arrive from
% different directions, so we cannot think of the flux $F_k$ in regard to
% a detector with unit area normal to the source direction; there is
% no one such direction.

We do not anticipate the UHECR flux passing through a volume element at the
Earth to vary in time over accesible time scales, so we can model the
arrival rate into a small volume of space from any particular direction as a
homogeneous Poisson point process in time.  Let $F_k$ denote the UHECR flux
from source $k$.  $F_k$ is the expected number of UHECRs per unit time from
source $k$ that would enter a fully exposed spherical detector of unit
cross-sectional area.  A cosmic ray source model must specify the directions
and fluxes of candidate sources.  In our framework, a candidate source
catalog specifies source directions for a fixed number of sources, $N_A$
($N_A = 17$ for the G10 AGN catalog).  In addition, we presume some cosmic
rays may come from uncatalogued sources, so we introduce a background
component, labeled by $k=0$, considered to be a population of isotropically
distributed sources,   We presume the background sources to be numerous
and to each have relatively low cosmic ray fluxes, so that at most a
single cosmic ray should be detected from any given background source
(i.e., we do not consider clustering of cosmic rays from directions not
corresponding to a specified source).  In this limit, the background
component may be parameterized with $F_0$ denoting the total flux from the
entire background population.

%\enote{In regard to the last comment:  I think we are implicitly presuming
%only a single CR from each background source.  This may be a slightly
%inconsistent simplification that should be noted.  E.g., for the radiant
%model, if a single unidentified background source produces a multiplet, in
%the radiant model this could lead to a ``ray'' of background sources with
%correlated directions but no apparent source.  For the buckshot model, there
%could be clusters with no apparent center.  This could be handled by
%introducing separate parameters for the total background flux, and the
%background source number density. Let's not go there for now!}

A model must specify a distribution for $\{F_k\} = \{F_0, \Fvec\}$; in
astronomical lingo, this corresponds to specifying a ``luminosity function''
for the background and source populations.  As a simple starting point, we
here focus on a model with an independent prior distribution for $F_0$, and
a ``standard candle'' model for the candidate host fluxes, $\Fvec$.  We
assume all sources have the same intensity, $I_0$, so the flux from a source
can be written as $F_k = I_0/D_k^2$ (the inverse-square law), with $D_k$ the
(known) distance to source $k$ (there could also be distance- and
energy-dependent attenuation due to cosmic ray-photon interactions, but the
sources we consider here are close enough that such attenuation should be
negligible).  The total flux from the sources is $F_A = \sum_k F_k$, and we
adopt $F_A$ as the source intensity parameter rather than $I_0$.  Thus
$F_k = w_k F_A$, with the weights $w_k$ given by
\be
w_k = \frac{1/D_k^2}{\sum_j 1/D_j^2},
\label{wt-def}
\ee
for $k=1$ to $N_A$.

%..............................................................................
\subsection{Prior Specification}

We must specify a prior distribution for $F_0$ and $F_A$.
Earlier observations constrained the total UHECR flux, $F_T = F_0 + F_A$.
Also, the null model will only have a single flux parameter
specifying the total flux from an isotropic distribution of source
directions.  This motivates adopting an alternative parameterization that
switches from $(F_0, F_A)$ to $(F_T,f)$, where $f = F_A/(F_0 + F_A)$ is the
fraction of the total flux attributed to the candidate host population.
In this parameterization, we can specify a common total flux prior
for all models.  This is astrophysically sensible since we have
results from prior experiments to set a scale for the total flux.
It is also statistically desirable; Bayes factors tend to be robust
to specification of priors for parameters common to models being compared.


We must specify a prior distribution for $F_0$ and $F_A$, or equivalently
$F_T$ and $f$.  We adopt
We adopt conjugate exponential priors with scales $s_0$ and $s_A$ for
$F_0$ and $F_A$, respectively.  We set the scales based on constraints
from earlier measurements.


In model M$_1$ and M$_2$, we assume that both $F_0$ and $F_A$ have
 an exponential prior with the same scale $s$, that is $s_0=s_a=s$. To keep the prior expected total fluxes the same for the null
model, we use the exponential prior with scale $2s$ for $F_0$ under M$_0$. Using the data from the
two previously operated observatories, AGASA and HiRes, we choose $s\approx 0.063 \mbox{km}^{-1}\mbox{year}^{-1}$

\subsection{Algorithm for Markov Monte Carlo}\label{sec:MCMC}

To draw posterior inference, we perform Gibbs sampling on parameters $F_0,F_A,\lambda$ with the full conditionals,
derived from  (\ref{eq:lik}) and the exponential priors, given as
\be
P(F_A|F_0,\lambda,D) \sim \mbox{gamma}\left(\mbox{shape}= 1+\sum_{k\geq 1}m_k(\lambda),\mbox{scale}=\frac{1}{\frac{1}{s} + \sum_{k\geq 1}w_k\epsilon_k}\right)
\ee
\be \quad
P(F_0|F_A,\lambda,D) \sim \mbox{gamma}\left(\mbox{shape}=1+m_0(\lambda), \mbox{scale}=\frac{1}{\frac{1}{s}+\epsilon_0}\right), \ \textrm{and}
\ee
\be
P(\lambda_i=k|F_A,F_0,D) \propto f_{k,i}F_k.
\ee



Given the fluxes, we model cosmic ray arrival times with a superposition of
homogeneous Poisson point processes from each component.  Besides its
arrival time, each cosmic ray has a label associated with it, identifying
its source component.  Let $\tlabel_i$ be the (latent) label for UHECR
$i$ ($\tlabel_i = 0$ for the background, or $k \ge 1$ for AGN $k$).  Since
a superposition of Poisson processes is a Poisson process, we may consider
the arrival times to come from a total event rate process, and the labels
to come from a categorical mark distribution with probability mass function
\be
P(\lambda_i=k|\Fvec) = \frac{F_k}{\sum_j F_j}.
\label{label-pmf}
\ee
In the absence of magnetic deflection, the labels could be replaced by
source directions (with background source directions assigned
isotropically), and the process could be considered to be Poisson in time
with a directional mark distribution.  But magnetic deflection requires a
more complex setup.

Our full framework also assigns energies as marks for each cosmic ray, drawn
from a distribution describing the cosmic ray spectrum.  The energies would
be important in an analysis that seeks to infer the cutoff energy
distinguishing local UHECRs rays from lower-energy cosmic rays (i.e., the
GZK cutoff).  Although the PAO-10 catalog includes event energies, the PAO
team has already made an energy cut, and in the absence of lower-energy
data, we cannot usefully infer a cutoff.  Thus in the analysis presented
here, we ignore the energy mark distribution.

%..............................................................................
\subsection{Cosmic ray deflection}

After leaving a source, UHECRs will have their paths deflected as they
traverse galactic and intergalactic magnetic fields.  The Galactic field is
partially measured and is known to have a complex structure; the magnetic
fields of other galaxies are at best crudely measured, and the much smaller
fields in intergalactic space are only weakly constrained (in fact, cosmic
rays might provide useful additional constraints).  A number of
investigators have modeled cosmic ray propagation in the Galaxy, or in
intergalactic space, using physical models based on existing field
measurements \cite{2010A&A...523A..49N,PhysRevD.82.043002,2010ApJ...719..459J}.
For our analysis we adopt simple
phenomenological models, inspired by the results of physical modeling.

In the simplest ``buckshot'' model, each cosmic ray from a particular source
experiences a deflection that is conditionally independent of the deflection
of other rays from that source, given a parameter, $\kappa$, describing the
distribution of deflections.  Let $\tlabel_i$ be an integer-valued latent
label for UHECR $i$, specifying its source ($\tlabel_i = 0$ for the background,
or $k \ge 1$ for AGN $k$).  We adopt a Fisher distribution for the
deflections, with $\kappa$ as the concentration parameter.  The probability
density for observing a cosmic ray from direction $\tdrxn$ if it is assigned
to source $k$ with direction $\hdrxn_k$ is
\be
\rho_k(\tdrxn|\kappa)
  = \frac{\kappa}{4\pi\sinh(\kappa)}\exp(\kappa\tdrxn\cdot\hdrxn_k).
\label{rho-def}
\ee
With this deflection distribution, when a cosmic ray is generated from an
isotropic background population, its deflected direction still has an
isotropic distribution.  Accordingly,
\be
\rho_0(\tdrxn|\kappa) =  \frac{1}{4\pi}.
\label{rho-iso}
\ee


%..............................................................................
\subsection{Cosmic ray detection and measurement}

Even though the arrival rate of UHECRs into a unit volume is constant in
time in our model, the expected number per unit time detected from a given
direction will vary as the rotation of the Earth changes the observatory's
projected area toward that direction, as noted above.  As a result, the
Poisson rate for detectable cosmic rays is inhomogeneous in time for
each source.

Recall that the likelihood function for an inhomogeneous Poisson
point process in time with rate $r(t)$ has the form
\be
\exp(-N_{\rm exp}) \prod_i r(t_i) \delta t,
\label{simple-ppp-like}
\ee
where the events are detected at times $t_i$ in detection intervals of size
$\delta t$, and $N_{\rm exp}$ is the total expected number in the observing
interval (the integral of the rate over the entire observing interval).  The
likelihood function for the cosmic ray data has a similar form, but with
adjustments due to the mark distribution and measurement errors.

If the label and arrival direction for detected cosmic ray $i$ were known, the
factor in the likelihood function associated with that cosmic ray would be
$F_k A_p(t_i, \tdrxn_i) \delta t$, where $k=\lambda_i$ and $\delta t$ is the
detection time interval (an ignorable constant factor).
In reality, the direction is uncertain; the PAO analysis pipeline produces
a likelihood function for the direction to the cosmic ray, $\ell_i(\tdrxn_i)$;
see equation~(\ref{ell-def}).
Introducing the uncertain direction as a nuisance parameter, with a
prior given by $\rho_k(\tdrxn_i|\kappa)$, the likelihood factor for
cosmic ray $i$ with known label may be calculated by marginalizing; it is
\be
f_{k,i}(\kappa) =
  \int d\omega_i \ell_i\left(\omega_i\right) A_p(t_i,\omega_i)
  \rho_k(\omega_i|\kappa).
\label{f-def}
\ee
The cosmic ray direction uncertainty is relatively small ($\sim 1^\circ$)
compared to the scale over which the area varies, so we can approximate
$f_{k,i}$ as
\ba
f_{k,i}(\kappa)
  \approx A_i\cos(\theta_i)
  \int  \ell_i(\omega_i)\rho_k(\omega_i|\kappa) d\omega_i,
\ea
where $\theta_i$ denotes the zenith angle of UHECR $i$ (reported by PAO) and
$A_i$ is the area of the observatory at the arrival time of UHECR $i$.  The
integral can be computed analytically;
\ba
\int d\omega_i \ell_i(\omega_i) \rho_k(\omega_i|\kappa) =
\begin{cases}
\frac{\kappa_c\kappa}{4\pi\sinh(\kappa_c)\sinh(\kappa)}
  \frac{\sinh(|\kappa_c d_i+\kappa\hdrxn_k|)}{|\kappa_c n_i+\kappa\hdrxn_k|}
  & \mbox{if $k\geq 1$},\\
\frac{1}{4\pi} & \mbox{if $k=0$}
\end{cases}
\label{f-approx}
\ea
The full likelihood factor associated with cosmic ray $i$ combines the
contributions from each source, i.e., sums over $k$.

To calculate the expected number factor in the likelihood, we must account
for the observatory's exposure map. The effective exposure given to cosmic
rays from source $k$ throughout the time of the survey depends, not just on
the direction to the source, but also on the deflection distribution,
$\rho_k$ (and thus on $\kappa$), since rays from that source will not arrive
precisely from the source direction. The exposure factor for source $k$ is
\be
\epsilon_k(\kappa) =
  \int d\tdrxn \rho_k(\tdrxn|\kappa) \int_T A(t,\omega) dt.
\label{eps-def}
\ee
An implication of the source-dependent exposure is that the probability mass
function for the label of a {\em detected} cosmic ray is not given by
(\ref{label-pmf}); the terms must be weighted according to the source
exposures.  The result is
\be
P(\lambda_i=k|\boldsymbol{F},\kappa) =
  \frac{F_k\epsilon_k(\kappa)}{\sum_j F_j\epsilon_j(\kappa)}.
\label{label-eps-pmf}
\ee

With these ingredients, we can now write down the marked Poisson point
process likelihood function for the detected cosmic rays:
\ba
\like(\Fvec,\kappa)
  &=& \exp\left(-\sum_k F_k\epsilon_k\right)
      \prod_i \left( \sum_k f_{k,i} F_k \right)\nonumber\\
  &=& \sum_{\lambda} \left(\prod_k F_k^{m_k(\lambda)}e^{-F_k\epsilon_k}\right)
    \prod_i f_{\lambda_i,i}
\label{like-def}
\ea
where the multiplicity $m_k(\lambda)$ is the number of UHECRs assigned to
source $k$ accordng to $\lambda$, and we suppress the $\kappa$ dependence of
$\epsilon_k(\kappa)$ here and elsewhere to simplify expressions.  The last
line follows from reversing the order of products and sums, familiar from
mixture modeling of density functions, but applied here to Poisson intensity
functions.

For computations it will be helpful to have the likelihood function
conditional on the label assignments,
\be
P(D|\lambda,\Fvec,\kappa)
  = \exp\left(-\sum F_k\epsilon_k\right) (F_k\epsilon_k )^{N_c}
    \prod_i \frac{f_{\lambda_i,i}}{\epsilon_{\lambda_i}}.
\ee
where $N_c$ is the number of UHECRs.  We can recover the likelihood for
$\Fvec$ and $\kappa$ by multiplying by the prior for
$\lambda$ and marginalizing, giving
\be
P\left(D|\boldsymbol{F},\kappa\right) = \sum_{\lambda}\left(\prod_k F_k^{m_k(\lambda)}e^{-F_k\epsilon_k}\right) \prod_i f_{\lambda_i,i}\label{eq:DR01}
\ee
which is identical to (\ref{like-def}).

Plugging in $F_k = w_kF_A$, we can rewrite $\like(\Fvec,\kappa)$ as
\ba
P\left(D|F_0,F_A,\kappa\right) &=& \sum_{\lambda} F_0^{m_0(\lambda)}e^{-F_0\epsilon_0} F_A^{N_c-m_0(\lambda)}e^{-F_A\sum w_k\epsilon_k}\nonumber\\
& & \times\prod_{k\geq 1}w_k ^{m_k(\lambda)} \prod_i f_{\lambda_i,i}
\label{eq:lik}
\ea

%..............................................................................
\subsection{Model Comparison}

Adopting the exponential priors with scales $s_0$ and $s_A$ for $F_0$ and
$F_A$, respectively, we have that the marginal likelihood for $\kappa$ is
\ba  \label{eq:marg}
\like_m(\kappa) = P(D|\kappa) &=& \sum_\lambda \left\{\frac{1}{s_0}\left(\frac{1}{\epsilon_0+\frac{1}{s_0}}\right)^{m_0(\lambda)+1}\Gamma(m_0(\lambda)+1)\right.\nonumber\\
& & \times \frac{1}{s_A}\left(\frac{1}{\sum_{k\geq 1}w_k\epsilon_k+\frac{1}{s_A}}\right)^{N_c-m_0(\lambda)+1}\nonumber\\
& &\left.\times \Gamma\left(N_c-m_0(\lambda)+1\right)\prod_{k\geq1}w_k^{m_k(\lambda)}\prod_i f_{\lambda_i,i}\right\}.
\ea
Even though $\like_m(\kappa)$ is available in closed form, it requires
summing over all possible values of $\lambda$ which is intractable in
practice. In Section \ref{sec:MCMC}, we present the Chib estimate
\cite{MR1379473} for this marginal likelihood.  We use Bayes factors to compare
different association models.  The ``null" model, M$_0$, assumes that all
the UHECRs come from the isotropic background source. Model M$_1$ allows the
UHECRs to come from any of the 17 AGNs in the catalog or from the isotropic
background. We are also interested in another model M$_2$ in which the
UHECRs comes from the isotropic background or either of the two closest
AGNs, Centaurus A (NGC 5128) and NGC 4945. In order to compare models M$_1$
and M$_2$ to the null model, we compute the Bayes factors:
\be
\mbox{BF}_{10}(\kappa) = \frac{\like_{m,1}}{\like_{m,0}},
     \mbox{ BF}_{20} = \frac{\like_{m,2}}{\like_{m,0}}
\ee
where $\like_1$ and $\like_2$ are computed using equation (\ref{eq:marg}), and
\be
\like_{m,0} = \frac{1}{s_0}\left(\frac{1}{\epsilon_0+\frac{1}{s_0}}\right)^{N_c+1}
     \Gamma(N_c+1) \times \prod_i f_{0,i}.
\ee
These expressions compare models conditioned on $\kappa$.  We also
compare models  marginalized over $\kappa$, as discussed
below. 