\section{Summary and Discussion}

We modeled CRs as coming from either nearby AGNs or an isotropic ``null'' source.  Three models were considered.  In $M_0$ all CRs come from the null source; in $M_1$ all CRs come from either the null source or one of the 17 closest AGNs; in $M_2$ all CRs come from either the null source or  one of the two closest AGNs.  The data come from three periods.  The data from the first period were used to ``tune'' the energy threshold to maximize an index of anisotropy.

Using all three periods, Bayes factors provide strong evidence for either $M_1$ or $M_2$ against $M_0$ but little evidence for or against $M_1$ versus $M_2$.  The Bayes factors vary considerably between periods, so we investigated whether data from the three periods should be combined.  Bayes factors reported in Section \ref{sec:difference_between_periods} support the hypothesis that the parameters do not vary between periods, which would justify using the combined data.

A simulation study in Section \ref{sec:simulations} investigates the variation of Bayes factors under repeated sampling and shows that the between-period variation seen here is not unexpected but rather is due to the small sample sizes in the three periods, especially the first two.  We also could conclude that the large Bayes factors in period 1 is unlikely to be due to chance.  It could be due to tuning; the possible effects of tuning have not been investigated here but this would be a good area for future investigation.

We have assumed that all AGNs have the same luminosities so that that their fluxes are inversely proportional to their squared distances from earth.  Other luminosity functions could be investigated, for example, that some AGNs are emitting CRs and others are not, which would require a model with indicators for the unknown emitter AGNs.


More complicated deflection models are possible.  We are considering models
that allow $\kappa$ to depend on the cosmic ray energy (the deflection scale
is expected to decrease with increasing energy), as well as ``radiant''
models that produce correlated deflections (as seen in some astrophysical
simulations).  For a radiant model, each source has a single guide direction
associated with it, drawn from a Fisher distribution centered at the source
direction, with concentration $\kappa_g$; the guide direction serves a proxy
for the shared magnetic deflection history of cosmic rays from that source.
Each cosmic ray associated with that source then has its arrival direction
drawn from an independent Fisher distribution centered about the guide
direction, with concentration $\kappa(E)$; this distribution describes the
effect of the deflection history unique to a particular cosmic ray.  The
resulting directions for a multiplet will cluster along a ray pointing
toward the source.  The resulting joint distribution for the directions in a
multiplet (with the guide direction marginalized) is exchangeable but not
independent.  For the current, moderately sparse cosmic ray catalog, the
complexity of this model is probably not warranted.  Here we focused on the
buckshot model, but radiant models will be worth exploring as the catalog
size grows.\footnote{Note that the buckshot model corresponds to the
radiant model in the limit of infinite $\kappa_g$, i.e., with the
guide direction coincident with the AGN direction.}

A fuller investigation of the sources of UHECRs requires that the PAO disclose more data, e.g., untuned data from period 1
and data collected since period 3 ended in 2009.  



