
To explore the latter possibility, we have to step outside the model family
one way or another.  One path is to explicitly introduce further alternative
models (perhaps nonparametric) and perform formal inference in the broader
class.  Another is to abandon a formal Bayesian approach, and instead seek
some measure of compatibility of model predictions with the data, expressing
how surprising the observed data are in light of any of the considered
models, e.g., using test statistics with $p$-values, or cross-validation
checks.  Neither approach is definitive.  The former approach still
conditions on a family of models, albeit a broader one.  The latter approach
does not explicitly specify new alternatives; but how one chooses to measure
predictive compatibility will concentrate power in some implicitly specified
subspace of the space of all models (\enote{REF-LR05}, \S~14.6).


BF table without 1+2:

\begin{table}
\begin{tabular}{|c|c|c |c| c| c| c|}
\hline
& & \multicolumn{5}{|c|}{Bayes factors}\\
\cline{3-7}
 Priors for $f$ & Model &Period 1 & Period 2 & Period 3 & Periods 2\&3 & Periods 1\&2\&3\\
\hline
beta(1,1) & 17 AGNs & 30.53 & 6.51 & 0.15 = 1/6.67 & 0.99 & 25.90\\
 & 2 AGNs  & 14.78 & 9.89 & 0.11 = 1/9.09 & 1.06 & 50.67\\
\hline
beta(1,5) & 17 AGNs & 39.27 & 15.12 & 0.52 = 1/1.92 & 3.39 & 78.69\\
 & 2 AGNs  & 31.97 & 27.97 & 0.42 = 1/2.38 & 4.08 & 176.65\\
\hline
\end{tabular}
\caption{Overall Bayes factors comparing association models with 17 AGNs
or 2 AGNs to the null isotropic background model for two different priors for $f$}\label{tab:BFtab}
\label{BFTable}
\end{table}
