\section{Auger observatory exposure}
\label{app:expo}

PAO can reliably detect and measure UHECRs arriving from directions within
$60^\circ$ from the observatory zenith.  Due to Earth's rotation, the zenith
traces a circular path on the sky, and the PAO field of view changes with time.
The observatory's geodetic latitude is $-35.5^\circ$, so the field of view
always includes the SCP and directions within a $5.5^\circ$ cone about it.
Outside of that cone, the time spent within the field of view decreases with
increasing latitude, vanishing for northern latitudes above $24.5^\circ$.  This
boundary corresponds to the thick gray curve shown in the sky maps.

In addition, at a given instant, the projected area of the observatory varies
with direction.  Since the observatory detects air showers, the effective area
of the detector toward a particular source direction is the projected area of
the layer of atmosphere above PAO toward that direction.  Due to Earth's
rotation, this projected area is a function of time; we denote it by
$\Aperp(t,\tdrxn)$ for direction $\tdrxn$ (a unit vector toward a fixed direction
on the sky) at time $t$.  We account for the zenith angle criterion by
setting $\Aperp=0$ for directions outside of the instantaneous field of view.

As described in \S~\ref{sec:data}, the exposure map, $\expo(\tdrxn)$,
is defined by
\begin{equation}
\expo(\tdrxn) \coloneqq \int_T  \Aperp(\tdrxn,t) dt.
\label{expo-def}
\end{equation}
The projected area can be written as $\Aperp(\tdrxn,t) = A(t) \mu(\tdrxn,t)$,
where $A(t)$ is the effective planar area of the detection volume, and
$\mu(t,\tdrxn)$ is a projection factor.  $A(t)$ varies slowly with time as
the observatory grows.  The projection factor varies much more quickly, due
to Earth's rotation. The PAO team has shown that for UHECRs, to a good
approximation a simple geometric projection varying periodically with a
period of 1~sidereal day gives a very accurate description of the PAO
exposure \cite{2001APh....14..271S}. As a result, the time integral in
equation~(\ref{expo-def}) can be approximated as $\int A(t) m(\tdrxn) dt$,
where $m(\tdrxn)$ is the geometric projection factor averaged over 1
sidereal day. The average projection factor is constant with respect to
right ascension (the equatorial sky coordinate corresponding to geodetic
longitude) due to rotational averaging.  It varies strongly with declination
(the equatorial sky coordinate corresponding to geodetic latitude).
Figure~\ref{fig:pjxn} shows the average projection as a function of
declination.

\begin{figure}
\centerline{\includegraphics[width=.8\textwidth]{avg_pjxn_factor.eps}}
\caption{Average projection factor, $m(\tdrxn)$, describing the declination
dependence of the PAO exposure map.}
\label{fig:pjxn}
\end{figure}

With these approximations, to evaluate $\expo(\tdrxn)$, we need the
time integral of the observatory area, $A(t)$.  By convention, this
quantity is reported indirectly by describing the observatory's
sensitivity to an isotropic distribution of sources (lower-energy cosmic
rays have an isotropic distribution).  For such a distribution, the expected
number of rays would be proportional to the {\em total exposure}, the
integral of the exposure map over the whole sky:
\begin{equation}
\alpha_T \coloneqq \int \expo(\tdrxn) d\tdrxn = \int A(t) dt \int m(\tdrxn) d\tdrxn.
\label{ET-def}
\end{equation}
The total exposure has units of area $\times$ time $\times$ solid angle;
(it has also been called ``aperture,'' as ``exposure'' is more traditionally
used for quantities with units of area $\times$ time).  The PAO team
reports $\alpha_T$ for each observing period in PAO-10.

To calculate the exposure map from $\alpha_T$ and $m(\tdrxn)$, define the
integrated projection factor by $M \coloneqq \int d\tdrxn m(\tdrxn)$ (with
units of solid angle).  Then the exposure is
\begin{equation}
\expo(\tdrxn) = \frac{\alpha_T}{M} m(\tdrxn).
\label{expo-mET}
\end{equation}
