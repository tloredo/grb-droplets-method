\section{Results}
\label{sec:results}

Recall that the UHECR data reported by PAO-10 are divided into three periods.
The PAO team used an initially larger period~1 sample (including
lower-energy events) to optimize an energy threshold determining which
events to analyze in period~2; the reported period~1 events are only those
with energies above the optimized threshold.  The optimization maximized a
measure of anisotropy in the above-threshold Period~1 sample.  Without access
to the full Period~1 sample, we cannot evaluate the impact of this
optimization on our modeling of anisotropy in the reported Period~1 data
(nor can we usefully pursue a Bayesian treatment of a GZK energy cutoff
parameter).  Because of this complication, we have performed
analyses for various subsets of the data.  As our main results, we report
calculations using data from Periods 2 and 3 combined (``untuned data'') and
for Periods 1, 2, and 3 combined (``all data'').  We also report some
results for each period considered separately, and we use them to perform a
simple test of consistency of the results across periods, in an effort
to assess the impact of tuning on the suitability of the Period~1 data
for straightforward statistical analysis.

%..............................................................................
\subsection{Results conditioning on the deflection scale, $\kappa$}

We first consider models conditional on the value of the magnetic deflection
scale parameter, $\kappa$, calculating Bayes factors comparing models,
and estimates of the association fraction, $f$.

We report model comparison results as curves showing Bayes
factors as functions of $\kappa$.  These quantities are astrophysically
interesting but must be interpreted with caution.  The actual values of the
conditional Bayes factors can only be interpreted as Bayes factors for a
particular value of $\kappa$ deemed interesting {\em a priori}.  For example,
were one to assume that UHECRs are protons, adopt a particular Galactic
magnetic field model, and assume that intergalactic magnetic fields do not
produce significant deflection (which is plausible for protons from local
sources), one would be interested only in large values of $\kappa$ of order
several hundred (corresponding to small angular scales for deflection).  On
the other hand, if one presumed that UHECRs are predominantly heavy nuclei,
then deflection by Galactic fields could be very strong, corresponding to
$\kappa$ of order unity (deflection by intergalactic fields might also be
significant in this case).  Models hypothesizing that most UHECRs are
heavy nuclei produced by Cen~A would fall in this small-$\kappa$ regime.
By presenting results conditional on $\kappa$, various cases
such as these may be considered.  Also, the Bayes factor conditioned on
$\kappa$ is proportional to the marginal likelihood for $\kappa$, so the
same curves summarize the information in the data for estimating $\kappa$ if
it is considered unknown.  We plot the curves against a logarithmic $\kappa$
axis, so they may be interpreted (up to normalization) as posterior
probability density functions based on a log-flat $\kappa$ prior.

The Bayes factors comparing models $M_1$ and $M_2$ to $M_0$ for various values
of $\kappa\in[1,1000]$, and for various partitions of the data, are shown in
Figure~\ref{fig:BFplot}.  For cases using only the untuned data (periods 2, 3,
and $2+3$), we find that both BF$_{10}$ and BF$_{20}$ (see
equation~\ref{BF10+20}) are close to 1 for all values of $\kappa\in[1,1000]$
for the beta-(1,1) (uniform) prior for $f$.  The Bayes factors are only
a little higher in the case of beta-(1,5) prior, indicating the results
are robust to reasonable changes in the $f$ prior. These values imply
that the posterior odds for the association models $M_1$ and $M_2$
versus the null isotropic background model $M_0$ are nearly equal to the
prior odds, indicating the untuned data provide little evidence for or
against either association model versus the isotropic model.

\begin{figure}
\centerline{\includegraphics[angle=-90,width=\textwidth]{BF_kappa_17AGNs_1+2.eps}}
\caption{Bayes factors comparing the assocation model with 17 AGNs (top row) or
2 AGNs(bottom row) with the null isotropic background model, conditional
on $\kappa$, shown as a function of $\kappa$ (bottom axis) and the
corresponding deflection angle scale, $\sigma$ (top axis).  Results are
shown for various partions of the data (identified by line style,
identified in the legend), and for two choices of the prior on $f$: a
flat prior (left column), and a Beta$(1,1)$ prior (right column).}
\label{fig:BFplot}
\end{figure}

Considering the Period~1 data qualitatively changes the results.  The
solid (blue) curves in Figure~\ref{fig:BFplot} show the Bayes factor
vs.\ $\kappa$ results based solely on the Period~1 data; there is
strong evidence for association models conditioned on $\kappa$ values
of around 50 to 100.\footnote{A common convention for interpreting
Bayes factors is due to Kass and Raftery, who consider a Bayes factor between
3 and 20 to indicate ``positive'' evidence and between 20 and 150 to indicate
``strong'' evidence \cite{Kass:Raft:baye:1995}.}
Analyzing the data from all three periods jointly produces the long-dashed
(purple) curves.  Using a uniform prior for $f$, we find BF$_{10}$ attains a
maximum of 90 at $\kappa\approx 46$ while BF$_{20}$ attains a maximum of 262
at $\kappa\approx 38$.  Both BF$_{10}$ and BF$_{20}$ are larger than 30 for
all $\kappa\in[20,120]$.  Both of the association models are strongly
preferred over the null in this range of $\kappa$, while the comparison is
inconclusive for $\kappa$ outside this range.

The originally published data (in PAO-08) covered periods 1 and 2.  For
comparison with studies of that original catalog, Figure~\ref{fig:BFplot}
include curves showing the Bayes factor vs.\ $\kappa$ based on data
from periods 1 and 2.  This partition of the data produces the largest
Bayes factors, $\sim 1000$ for $\kappa \approx 50$.  The curves are
qualitatively consistent with accumulation of evidence from Period~1 and
Period~2.\footnote{Note that the Bayes
factor for the $1+2$ partition should not be expected to equal the
product of the Bayes factors based on the Period~1 and Period~2 partitions,
because the models are composite hypotheses, and the data from different
periods generally will favor different values of the model parameters.}
These results amplify what was found in the analysis using all of
the data:  the strongest evidence for association comes from the Period~1
data.  This is troubling because this data was used (along with unreported
lower-energy data) to tune the energy cut defining all of the samples, and
there is no way for independent investigators to account for the effects of
the tuning on the strength of the evidence in the Period~1 data.

We show marginal posterior distributions for $f$ in
Figure~\ref{fig:postf}, for both $M_1$ and $M_2$, using both the untuned
data, and using all data. For a given model, the posterior does not
change much when Period~1 data are included.  The posteriors indicate
evidence for small but nonzero values of $f$, of order a few percent to
20\%.  They strongly rule out values of $f>0.3$, indicating that most
UHECRs must be assigned to the isotropic background component in these
models.
This holds even for values of $\kappa$ as small as $\approx 10$,
corresponding to quite large magnetic deflection scales, as might be
experienced by iron nuclei in typical cosmic magnetic fields.  Of
course, when $\kappa=0$ the association models become indistinguishable
from an isotropic background model.  These results suggest that a model
assigning {\em all} UHECRs to a single nearby source, such as Cen~A, may be
tenable only with very large deflection scales.  In Supplementary
Appendix~\ref{app:CenA} we briefly explore such a model and confirm this
conclusion.

\begin{figure}
\centerline{\includegraphics[angle=-90,width=.9\textwidth]{postf.eps}}
\caption{Posterior distributions for $f$, conditioned on $\kappa$ = 10,
31.6, 100, 316 and 1000.}
\label{fig:postf}
\end{figure}

A recent approximate Bayesian analysis \cite{WMJ11-BayesUHECR}, based on a
discrete pixelization of the sky, attributed a similar fraction of the
sample of 27 Period~1 and Period~2 UHECRs to standard-candle AGN sources,
considering $\approx 900$ AGN within 100~Mpc from the VCV as candidate
sources.  We compare our approaches and results in Supplementary
Appendix~\ref{app:WMJ11}.

The posterior mode is at larger values of $f$ for model
$M_1$ (with 17 AGN) than for $M_2$ (with the two closest AGN), suggesting
that there is evidence that AGN in the G10 catalog besides Cen~A
and NGC~4945 are sources of UHECRs.  Our multilevel model allows us to
address source identification explicitly, by providing a posterior
distribution for possible association assignments (values of $\lambda$).
In Table~\ref{lambdaTable} we show marginal posterior probabilities for 
associations that have non-negligible probabilities under both $M_1$
and $M_2$, for two representative values of $\kappa$ ($\kappa = 31.62$,
corresponding to a $15.5^\circ$ deflection scale, is a favored value for
analyses including Period~1 data as shown below; $\kappa=1000$,
corresponding to a $2.7^\circ$ deflection scale, may be appropriate if
UHECRs are predominantly protons).  Rows are labeled by cosmic ray number,
$i$, and columns by AGN number, $k$; the tabulated values are
$P(\lambda_i=k|\cdots)$.  Cosmic rays 17 and 20 (in Period~2) are associated
with Cen~A (AGN 13) with modest to high probability in all cases.  No other
assignments are robust (notably, Period~3 has no robust assignments, despite
containing more than three times the number of cosmic rays as Period~2).  If
UHECRs experience only small deflections, then besides the two Cen~A
associations, it is highly probable that cosmic ray 8 (in Period~1) is
associated with NGC~4945.  For the larger deflection scale, although nearly a
quarter of the cosmic rays have candidate associations with probability
$>0.1$, none of those associations have probability $>0.5$.
\mnote{Added last sent.}
In summary, the larger favored value of $f$ for $M_1$ reflects the 17~AGN
model finding enough plausible associations (besides those with Cen~A and
NGC~4945) that it is likely that some of them are genuine.

We can also calculate posterior probabilities for multiplet assignments.  In
general, the probability for a multiplet assigning a set of cosmic rays to a
particular candidate source will not be the product of the probabilities for
assigning each ray to the source.  In Table~\ref{lambdaTable} we see that
CRs 17 and 20 are often commonly assigned to Cen~A.  As an example, for
$M_1$ with $\kappa=1000$, their separate probabilities for assignment to
Cen~A are 0.85 and 0.94, respectively.  The probability for a doublet
assignment of both of them to Cen~A in this model is 0.80, which happens to
be nearly equal to the product of their separate (marginal) assignment
probabilities.  Were we to marginalize over $\kappa$, the multiplet
probability would differ from the product, since the preferred value
of $\kappa$ differs slightly between these two CRs.

%\enote{Do we want to say any more about multiplet probabilities, e.g.,
%for low-$\kappa$ models?}

\begin{sidewaystable}
\begin{center}
\begin{tabular}{|c|c|c|c|c|c|c|c|c|c|c|c|c|c|c|}

  \hline
  & \multicolumn{10}{|c|}{17 AGN + isotropic} & \multicolumn{4}{|c|}{2 AGN + isotropic} \\
  \cline{2-15}
  & \multicolumn{5}{|c|}{$\kappa$=31.62} & \multicolumn{5}{|c|}{$\kappa$=1000}& \multicolumn{2}{|c|}{$\kappa$=31.62} & \multicolumn{2}{|c|}{$\kappa$=1000} \\
   \cline{2-15}
  CR$\backslash$ AGN &
        2  &   7  &  11  &  13  &  17  &   2   &   11   &   13   &   16  &   17  &   11  &   13  &   11   &   13  \\
  \hline
  2 &   -  &   -  & 0.24 & 0.46 &   -  &   -   &   -    &    -   &   -    &   -    & 0.26  & 0.51  &   -    &   -   \\
  3 &   -  &   -  & 0.42 & 0.20 &   -  &   -   &   -    &    -   &   -    &   -    & 0.47  & 0.22  &   -    &   -   \\
  4 &   -  &   -  &   -  &    - & 0.17 &   -   &   -    &    -   &   -    &   -    &   -   &    -  &   -    &   -   \\
  5 &   -  &   -  & 0.18 & 0.28 &   -  &   -   &   -    &    -   &   -    &   -    & 0.22  & 0.35  &   -    &   -   \\
  6 & 0.11 &   -  &   -  &   -  &   -  &   -   &   -    &    -   &   -    &   -    &   -   &   -   &   -    &   -   \\
  8 &   -  &   -  & 0.43 & 0.36 &   -  &   -   & 0.89   &    -   &   -    &   -    & 0.47  & 0.38  & 0.90   &   -   \\
 13 &   -  &   -  &   -  &   -  & 0.17 &   -   &   -    &   -    &   -    & 0.11   &   -   &   -   &   -    &   -   \\
 14 &   -  &   -  & 0.47 & 0.27 &   -  &   -   &   -    &   -    &   -    &   -    & 0.51  & 0.29  &   -    &   -   \\
 \hline
 17 &   -  &   -  & 0.38 & 0.41 &   -  &   -   &   -    & 0.85   &   -    &   -    & 0.41  & 0.44  &   -    & 0.86  \\
 18 &   -  &   -  &   -  & 0.15 &   -  &   -   &   -    &   -    &   -    &   -    &   -   & 0.20  &   -    &   -   \\
 20 &   -  &   -  & 0.36 & 0.43 &   -  &   -   &   -    & 0.94   &   -    &   -    & 0.39  & 0.46  &   -    & 0.95  \\
 23 &   -  &   -  & 0.32 & 0.26 &   -  &   -   &   -    &   -    &   -    &   -    & 0.37  & 0.30  &   -    &   -   \\
 26 &   -  & 0.17 & 0.10 & 0.19 &   -  &   -   &   -    &   -    &   -    &   -    & 0.15  & 0.27  &   -    &   -   \\
 \hline
 33 &   -  &   -  & 0.40 & 0.11 &   -  &   -   &   -    &   -    &   -    &   -    & 0.46  & 0.12  &   -    &   -   \\
 34 &   -  &   -  & 0.47 & 0.27 &   -  &   -   &   -    &   -    &   -    &   -    & 0.51  & 0.30  &   -    &   -   \\
 36 &   -  &   -  & 0.21 & 0.35 &   -  &   -   &   -    &   -    & 0.48   &   -    & 0.24  & 0.42  &   -    &   -   \\
 47 &   -  &   -  & 0.14 & 0.42 &   -  &   -   &   -    &   -    &   -    &   -    & 0.15  & 0.48  &   -    &   -   \\
 54 &   -  &   -  & 0.19 & 0.46 &   -  &   -   &   -    &   -    &   -    &   -    & 0.21  & 0.52  &   -    &   -   \\
 55 & 0.15 &   -  &   -  &   -  &   -  & 0.34  &   -    &   -    &   -    &   -    &   -   &   -   &   -    &   -   \\
 57 &   -  & 0.41 &   -  &   -  &   -  &   -   &   -    &   -    &   -    &   -    &   -   &   -   &   -    &   -   \\
 67 &   -  &   -  & 0.32 & 0.30 &   -  &   -   &   -    &   -    &   -    &   -    & 0.37  & 0.34  &   -    &   -   \\
 \hline

\end{tabular}
\end{center}
\caption{The posterior probability that each cosmic ray is assigned to each
AGN given $\kappa = 31.62$ and 1000, using cosmic rays from periods 1+2+3.
Only assignments with probabilities greater than 0.1 are shown.  The AGN
identifiers are:  2:~NGC~0613; 7:~NGC~3621; 11:~NGC~4945;
13:~NGC~5128  (Cen~A); 17:~NGC~6300.}
\label{lambdaTable}
\end{sidewaystable}


%..............................................................................
\subsection{Results with $\kappa$ as a free parameter}

Joint marginal posterior distributions for $\log_{10}(\kappa)$ and $f$ are
shown in Figure~\ref{fig:jointkappaf}, for both association models, and for
untuned data and all data samples.  For the all-data cases, the joint
posterior distribution is unimodal and attains its maximum at ($\kappa$=32 ,
$f$=0.13) and ($\kappa$=32 , $f$=0.09) for the association model with 17
AGNs and 2 AGNs, respectively.  For untuned data, the joint posteriors are
bimodal with one of the mode at the value of $\kappa$ slightly less than in
the case of all 3 periods, and the other mode at $\kappa\approx 1000$,
similar to the plot of Bayes factors in Figure~\ref{fig:BFplot}.  The
results from the two samples are more similar than this description may
indicate; they have significant peaks in the same region, but the likelihood
function is relatively flat for the largest and smallest values of $\kappa$
(this is also apparent in Figure~\ref{fig:BFplot}).

In all cases, the preferred values of $\kappa$ correspond to deflection
scales $\approx 10^\circ$.  As noted above, models of proton propagation in
cosmic magnetic fields predict deflections of a few degrees.
The posterior distributions for $\kappa$ are comfortably consistent with
such predictions, but they do favor the larger scales that would be
experienced by heavier nuclei.  These scales are consistent with the
suggestive evidence from PAO that UHECRs may be comprised of heavier nuclei
than lower-energy cosmic rays.

\begin{figure}
\centerline{\includegraphics[width=\textwidth]{kappa_f-log_kappa.eps}}
\caption{Marginal joint posterior distributions for the magnetic deflection
concentration parameter, $\kappa$, and the association fraction, $f$,
considering UHECR data from different periods, and candidate host catalogs of 2
or 17 nearby AGN.  Contours bound HPD credible regions of probability
0.25 (blue), 0.5 (green), 0.75 (red), 0.95 (brown), and 0.99 (gray).}
\label{fig:jointkappaf}
\end{figure}

Values for Bayes factors accounting for $\kappa$ uncertainty are listed in
Table~\ref{tab:BFtab}, for both association models, and for both individual
and combined data samples (these values are based on the default flat
prior for $f$).  We find strong evidence for both association models when
considering all the cosmic ray data.  If we exclude the tuned data of period
1, then we see positive evidence for association if we consider only period
2 but positive evidence for the null model if we consider only period 3.  If
we pool the untuned data, the data are equivocal.  Together, these results
raise concerns about consistency of the data and adequacy of the models; we
address this further below.  These results do not change qualitatively when
we use the alternative prior for $f$ described in \S~3.2.

\begin{table}
\begin{tabular}{|c|c|c |c| c| c| c| c|}
\hline
& & \multicolumn{6}{|c|}{Data Periods Used}\\
\cline{3-8}
 Priors for $f$ & Model & 1 &  2 &  3 &  1\&2 & 2\&3 &  1\&2\&3\\
\hline
beta(1,1) & 17 AGNs & 31 & 6.5 & 0.15 = 1/6.7 & 370 & 0.99 & 26\\
 & 2 AGNs  & 15 & 9.9 & 0.11 = 1/9.1 & 440 & 1.1 & 51\\
\hline
beta(1,5) & 17 AGNs & 39 & 15 & 0.52 = 1/1.9 & 710 & 3.4 & 79\\
 & 2 AGNs  & 32 & 28 & 0.42 = 1/2.4 & 1100 & 4.1 & 180\\
\hline
\end{tabular}
\caption{Overall Bayes factors comparing association models with 17 AGNs or
2 AGNs to the null isotropic background model, for two different priors for
$f$}\label{tab:BFtab}
\label{BFTable}
\end{table}

%> 1/.11
%[1] 9.09091
%> 1/.15
%[1] 6.666667

Marginal posterior distributions for $f$ and for $F_T$ are shown in
Figure~\ref{fig:posterior}.  For the untuned data, the posterior mode of
$f$ is 0.051 for $M_1$ (17 AGNs) and 0.047 for $M_2$ (2 AGNs);
the 95\% highest density credible intervals for $f$ are $[0, 0.23]$
and $[0.002, 0.145]$, respectively.  Using all of the data, the
distributions shift to somewhat larger values of $f$; the
posterior mode of $f$ is 0.11 for $M_1$ and 0.08 for $M_2$, and $f=0$
has a significantly smaller density.  However, the uncertainties are
large enough that the $f$ estimates are consistent with each other.  The
posterior distributions for $F_T$ are very similar in all models.  The
peaks are a little higher and the widths of the peaks are smaller when
we consider the cosmic rays from periods 1--3, as expected, since we
have more data.  The posterior modes correspond to total fluxes of about
0.04 km$^{-2}$ yr$^{-1}$ in all cases.

\begin{figure}
\centerline{$
\begin{array}{cc}
\includegraphics[angle=-90,width=.5\textwidth]{posterior_f_all_margOverKappa.eps} &
\includegraphics[angle=-90,width=.5\textwidth]{posterior_FT_all_margOverKappa.eps}
\end{array}$}
\caption{Marginal posterior distributions for $f$ (the fraction of
UHECRs associated with AGNs in candidate catalogs), and $F_T$ (the total
flux), considering UHECR data from different periods, and models
associating UHECRs with either 2 or 17 nearby AGN.}
\label{fig:posterior}
\end{figure}


%\begin{figure}
%\centerline{$
%\begin{array}{cc}
%\includegraphics[width=1.75in,angle=-90]{posterior_F_margOverKappa_17AGNs.eps} &
%\includegraphics[width=1.75in,angle=-90]{posterior_F_margOverKappa_2AGNs.eps}
%\end{array}$}
%\caption{50\%, 68\% and 95\% credible regions from joint posteriors for $(F_0,F_A)$ for 17-AGN %model (left) and 2-AGN model (right)}
%\label{fig:jointF}
%\end{figure}
