\section{Summary and Discussion}
\label{sec:summary}

We have described a new multilevel Bayesian framework for modeling the
arrival times, directions, and energies of UHECRs, including statistical
assessment of directional coincidences with candidate sources.
Our framework explicitly models cosmic ray emission, propagation (including
deflection of trajectories by cosmic magnetic fields), and detection.  This
approach cleanly distinguishes astrophysical and experimental processes
underlying the data.  It handles uncertain parameters in these processes via
marginalization, which accounts for uncertainties while allowing use of all
of the data (in contrast to hypothesis testing approaches that optimize over
parameters, requiring holding out a subset of the data for tuning).
We demonstrated the framework by implementing calculations with simple but
astrophysically interesting models for the 69 UHECRs with energies above
55~EeV detected by PAO and reported in PAO-10.  Here we first summarize
our findings based on these models, and then describe directions for
future work.

%Its Bayesian underpinning enables accounting for
%a priori uncertainty in model parameters via averaging (marginalization).
%This stands in contrast to previously-used hypothesis testing approaches
%that handle free parameters defining a test procedure by optimizing using
%a subset of the data, with subsequent analysis omitting the tuning data.

%..............................................................................
\subsection{Astrophysical results}

We modeled UHECRs as coming from either nearby AGN (in a volume-limited
sample including all 17 AGN within 15~Mpc) or an isotropic background
population of sources; AGN are considered to be standard candles in
our models.  We thoroughly explored three models.  In $M_0$ all CRs come
from the isotropic background; in $M_1$ all CRs come from either a
background or one of the 17 closest AGN; in $M_2$ all CRs come from either
a background source or one of the two closest AGN (Cen~A and NGC~5128,
neighboring AGN at a distance of 5~Mpc).  The data were reported in three
periods.  Data from Period~1 were used to tune the energy threshold defining
the published samples in all periods by maximizing an index of anisotropy in
Period~1.  Out of concern that this tuning compromises the data in Period~1
for our analysis, we analyzed the full dataset and various subsamples,
including an ``untuned'' sample omitting Period~1 data.

Using {\em all} of the data, Bayes factors indicate there is strong evidence
favoring either $M_1$ or $M_2$ against $M_0$ but do not discriminate between
$M_1$ and $M_2$.  The most probable models associate about 5\% to 15\% of
UHECRs with nearby AGN, and strongly rule out associating more than
$\approx 25$\% of UHECRs with nearby AGN.  Most of the high-probability
associations in the 17~AGN model are with the two closest AGN.

However, if we use only the {\em untuned} data, the Bayes factors are
equivocal (although the most probable association models resemble those
found using all data).  If we subdivide the untuned data, we find positive
evidence for association using the Period~2 sample, but weak evidence {\em
against} association using the much larger Period~3 sample.  Together, these
results suggest that the statistical character of the data may differ from
period to period, due to tuning of the Period~1 data or other causes.

One way to explore this is to ask whether the data from the various periods
are better explained using models with differing parameter values rather
than a shared set of values.  We investigated this via a change-point
analysis that considered the time points bounding the periods as candidate
change points.  The results are consistent with the hypothesis that the
parameters do {\em not} vary between periods, justifying using the combined
data for these models.  This suggests the variation of the Bayes factors
across periods is a consequence of the modest sample sizes.  However, the
change point analysis does not address the possibility that none of the
models is adequate, with model misspecification being the cause of the
apparently discrepant Bayes factors.

We used simulated data from both the isotropic model and
high-probability association models to perform predictive checks of our
models, using the Bayes factors based on subsets of the data as test
statistics.  Simulations based on the isotropic model indicate that large
Bayes factors favoring association are unlikely for {\em untuned} samples of
the size of the Period~1 sample.  Simulations based on representative
association models indicate that such Bayes factors are not surprising
for samples of the size of Period~1, considered in isolation.  But
the observed pattern of large Bayes factors for the subsamples in
periods 1 and 2, and a small Bayes factor for the much larger Period~3
subsample, is very surprising.  The full dataset thus is not fit
comfortably by either isotropic models or standard-candle association
models.
Whether the effects of tuning could explain the apparent inconsistencies
remains an open question that is not easy to address without access to the
untuned data.

\mnote{Clarified}
Restricting to the untuned data (periods 2 and 3), the pattern of Bayes
factors is consistent with both isotropic models and representative standard
candle association models.
The best-fitting association models assign a few percent of UHECRs to nearby
AGN; at most $\approx 20$\% may be associated with AGN, with the remainder
assigned to sources drawn from an isotropic distribution.
Magnetic deflection angular scales of $\approx 3^\circ$
to $30^\circ$ are favored.
Models that assign a large fraction of UHECRs to a single nearby source (e.g.,
Cen~A) are ruled out unless very large deflection scales are specified a
priori, and even then they are disfavored.

Even restricting to results based on the untuned data, we hesitate to offer
these models as astrophysically plausible explanations of the PAO UHECR
data, both because of how important the problematic Period~1 sample is in
the analysis, and because of astrophysical limitations of the models
considered here and elsewhere.  In particular, the high-probability models
assign the vast majority of UHECRs to sources in an isotropic distribution.
But the observation by PAO of a GZK-like cutoff in the energy spectrum of
UHECRs argues strongly that UHECRs originate from within $\sim 100$~Mpc,
where the distribution of both visible matter (galaxies) and dark matter is
significantly {\em an}isotropic.  If most or all UHECRs are protons, so that
magnetic deflection is not very strong, an isotropic distribution of UHECR
arrival directions is implausible.  It then may be the case that some of the
strength of the evidence for association with nearby AGN is due to the
``straw man'' nature of the isotropic alternative.  On the other hand, if
most UHECRs are heavy nuclei, then strong magnetic deflection could
isotropize the arrival directions.  The highest probability association
models have relatively small angular deflection scales, but it could be that
the few UHECRs that these models associate with the nearest AGN happen to be
protons or very light nuclei.  Future models could account for this by
allowing a mixture of $\kappa$ values among cosmic rays, as noted in
\S~\ref{sec:dflxn}.

\mnote{Changed ``implausible'' to ``overly simplistic''}
In addition, the standard candle cosmic ray intensity model adopted here and
in other studies is astrophysically overly simplistic and very likely strongly
constrains inferences.  The strongest visible clustering of measured UHECR
directions is toward the two closest AGN, just 5~Mpc away and only a few
degrees apart on the sky.  Most of the high-probability associations
identified in our models are to these AGN.  They are so close that they
imply standard-candle cosmic ray intensities that would produce a negligible
flux of cosmic rays from the vast majority of other AGN within 100~Mpc.  Put
another way, a standard-candle model assigning just one UHECR to an AGN near
the 100~Mpc GZK limit would imply a cosmic ray flux from nearby AGN so huge
that this scenario is ruled out simply by visible inspection of sky maps. 
Models with more flexible luminosity functions would likely allow assignment
of many more UHECRs to sources spanning a range of distances.

%..............................................................................
\subsection{Future directions}

All of these considerations indicate a more thorough exploration of
UHECR production and propagation models is needed.  We thus consider
the analyses here to be a demonstration of the utility and feasibility
of analyzing such models within a multilevel Bayesian framework, and
not a definitive astrophysical analysis of the data.
We are pursuing more complex models elsewhere, expanding on the present
analysis in four directions.

First, we are considering larger, statistically well-characterized catalogs
of potential hosts, e.g., the recently-compiled catalog of X-ray selected
AGN detected by the Burst and Transient (BAT) instrument on the {\em Swift}
satellite, a catalog considered by PAO-10.

Second, we are building more realistic background distributions, for example
by using the locations of nearby galaxy clusters, or the entire nearby
galaxy distribution, to build smooth background densities (e.g., via kernel
density estimation, or fitting of mixture or multipole models).

Third, we are considering richer luminosity function models, including models
assigning a distribution of cosmic ray intensities to all candidate sources,
and models that place some sources in ``on''
states and the others ``off.''  The latter models are motivated both by the
possibility of beaming of cosmic rays, and by evidence for AGN intermittency
in jet substructure, and could enable assignment of significant numbers of
UHECRs to both distant and nearby sources.

Finally, more complicated deflection models are possible.  
For example, we have developed a class of ``radiant'' models that produce
correlated deflections (as seen in some astrophysical simulations).  For a
radiant model, each source has a single guide direction associated with it,
drawn from a Fisher distribution centered at the source direction, with
concentration $\kappa_g$; the guide direction serves as a proxy for the shared
magnetic deflection history of cosmic rays from that source.  Each cosmic ray
associated with that source then has its arrival direction drawn from an
independent Fisher distribution centered about the guide direction, with
concentration potentially depending on cosmic ray energy and source distance;
this distribution describes the effect of the deflection history unique to a
particular cosmic ray.  The resulting directions for a multiplet will cluster
along a ray pointing toward the source.  The resulting joint distribution for
the directions in a multiplet (with the guide direction marginalized) is
exchangeable but not independent.

For the current, modest-sized UHECR catalog, the complexity of
some of these generalizations  is probably not warranted.  But PAO is
expected to operate for many years, and the sample is continually growing in
size.  Making the most of existing and future data will require, not only
more realistic models, but also more complete disclosure of the data.
In particular, a fully Bayesian treatment---including modeling of the energy
dependence in the UHECR flux and deflection scale---requires data
uncorrupted by tuning cuts.  Further, the most accurate analysis should use
event-specific direction and energy uncertainties (likelihood summaries),
rather than the typical error scales currently reported.  We hope 
our framework helps motivate more complete releases of future PAO data.
