%$Id: macros.tex,v 1.3 2009/07/06 14:57:36 rlw Exp rlw $
\newcommand{\met}{\thinspace{\text{m}}}\newcommand{\km}{\thinspace{\text{km}}}
\newcommand{\dy}{\thinspace{\text{d}}} \newcommand{\yr}{\thinspace{\text{yr}}}
\newcommand{\mm}{\thinspace{\text{mm}}}\newcommand{\mi}{\thinspace{\text{mi}}}
\newcommand{\s}{\thinspace{\text{s}}}%
\newcommand{\fm}{\thinspace{\text{fm}}}% "femta meter", same as fermi, 1e-15m
\newcommand{\xbar}{\overline X}%
\newcommand{\xbbar}{\overline{\overline X}}%
%\font\ss=cmss12
\newcommand{\OFP}{(\Omega,\cF,\P)}
\newcommand{\bbC}{{\mathbb{C}}}
\newcommand{\bbF}{{\mathbb{F}}}
\newcommand{\bbN}{{\mathbb{N}}}
\newcommand{\bbQ}{{\mathbb{Q}}}
\newcommand{\bbR}{{\mathbb{R}}}
\newcommand{\bbX}{{\mathbb{X}}}
\newcommand{\bbZ}{{\mathbb{Z}}}
\newcommand{\bx}{{\mathbf{x}}}
\newcommand{\by}{{\mathbf{y}}}
\newcommand{\cA}{{\mathcal{A}}}
\newcommand{\cB}{{\mathcal{B}} }
\newcommand{\cC}{{\mathcal{C}} }
\newcommand{\cE}{{\mathcal{E}}}
\newcommand{\cF}{{\mathcal{F}}}
\newcommand{\cG}{{\mathcal{G}}}
\newcommand{\cH}{{\mathcal{H}}}
\newcommand{\cL}{{\mathcal{L}}}
\newcommand{\cM}{{\mathcal{M}}}
\newcommand{\cP}{{\mathcal{P}}}
\newcommand{\cR}{{\mathcal{R}}}
\newcommand{\cS}{{\mathcal{S}}}
\newcommand{\cT}{{\mathcal{T}}}
\newcommand{\cX}{{\mathcal{X}}}
\newcommand{\cY}{{\mathcal{Y}}}
\newcommand{\cZ}{{\mathcal{Z}}}
\newcommand{\fF}{{\mathfrak F}}
\newcommand{\fG}{{\mathfrak G}} 
\newcommand{\fH}{{\mathfrak H}}
\newcommand{\Eone}{{\ensuremath{\mathrm{E}_1}}}
\renewcommand{\P}{{\mathsf{P}}} \newcommand{\E}{{\mathsf{E}}}
\newcommand{\Ev}{{\mathsf{Ev}}}%
\newcommand{\V}{{\mathsf{V}}}\newcommand{\Cov}{{\mathsf{Cov}}}
\newcommand{\Be}{\mathsf{Be}}\newcommand{\Bi}{\mathsf{Bi}}
\newcommand{\Ca}{\mathsf{Ca}}
\newcommand{\Ex}{\mathsf{Ex}}\newcommand{\Ga}{\mathsf{Ga}}
\newcommand{\Di}{\mathsf{Di}}\newcommand{\Ge}{\mathsf{Ge}}
\newcommand{\IG}{\mathsf{IG}}\newcommand{\We}{\mathsf{We}}
\newcommand{\HG}{\mathsf{HG}}\newcommand{\MN}{\mathsf{MN}}
\newcommand{\NB}{\mathsf{NB}}\newcommand{\No}{\mathsf{No}}
\newcommand{\LN}{\mathsf{LN}}\newcommand{\Pa}{\mathsf{Pa}}
\newcommand{\Po}{\mathsf{Po}}\newcommand{\Un}{\mathsf{Un}}
\newcommand{\Lv}{\mathsf{Lv}}\newcommand{\St}{\mathsf{St}}
\newcommand{\StA}{\mathsf{St_{A}}}\newcommand{\StM}{\mathsf{St_{M}}}
\newcommand{\eps}{\epsilon}\newcommand{\hide}[1]{}
\newcommand{\argmax}{\mathop{\mathrm{argmax}}}
\newcommand{\argmin}{\mathop{\mathrm{argmin}}}
\newcommand{\diag}{\mathop{\mathrm{diag}}}
\renewcommand{\th}{{\ensuremath^{\mbox{\tiny th}}}}
\newcommand{\nd}{{\ensuremath^{\mbox{\tiny nd}}}}
\newcommand{\st}{{\ensuremath^{\mbox{\tiny st}}}}
\newcommand{\ii}{{\ensuremath{\bar{\i}}}}% \newcommand{\ii}{{\hat i}}
\newcommand{\jj}{{\ensuremath{\bar{\j}}}}%
\newcommand{\R}{\texttt{R}}
\newcommand{\df}{\mathrel{\mathop{:=}}}
\newcommand{\fd}{\mathrel{\mathop{=:}}}
\newcommand{\ind}{\mathrel{\mathop{\sim}\limits^{\mathrm{ind}}}}
\newcommand{\iid}{\mathrel{\mathop{\sim}\limits^{\mathrm{iid}}}}
\newcommand{\mayeq}{\mathrel{\mathop{=}\limits^?}}
\newcommand{\HI}{{Hawai\!\`{}\!i}} % okina between the i's
%\newcommand{\pperp}{\mathrel{{\rlap{$~\perp$}\perp\,\,}}}
% Donald Arseneau's more sophisticated version:
\newcommand{\pperp}{\protect\mathpalette{\protect\independenT}{\perp}}
   \def\independenT#1#2{\mathrel{\rlap{$#1#2$}\mkern4mu{#1#2}}}
\newcommand{\nn}{\nonumber} % or \notag
\newbox\asbox
\setbox\asbox=\hbox{\vrule height 15pt depth3.5pt width0pt}
\def\astrut{\relax\ifmmode\copy\strutbox\else\unhcopy\strutbox\fi}
\def\Strut{\vrule width0pt height 16pt depth 4pt}%
\newcommand{\yn}{\ensuremath{\mathsf Y~\mathsf N\quad}}
\newcommand{\TF}{\ensuremath{\mathsf T~\mathsf F\quad}}
\newcommand{\YN}{\hbox{$\bigcirc$~Yes\quad$\bigcirc$~No}}
\newcommand{\STRUT}[2]{\vrule width0pt height #1pt depth #2pt}%
\newdimen\bsigdep
\def\bSig{{\setbox0\hbox{$\Sigma$}\bsigdep=1\dp0\advance\bsigdep by
 .2\ht0 \rlap{\kern.3\wd0\vrule height1.2\ht0 depth1\bsigdep}\box0}}%
\def\Sbar{\bSig}%
%\newdimen\digitwidth\setbox0=\hbox{{\mathrm0}\digitwidth=\wd0\def\0{\kern\digitwidth}}
\newcommand{\0}{\phantom{0}}
\newcommand{\ans}[1]{\hbox to #1truecm{
    \vrule height 20pt depth3.5pt width0pt
    \leaders\hrule height-2.5pt depth3pt\hfill}}
\newcommand{\ansbox}{\rlap{\hspace{-5mm}\framebox[30mm]{\STRUT{10}{5}}}}
%\font\tinyss=cmss8 at 8truept % for ^T etc
\newcommand{\tsf}[1]{\mathsf{\scriptscriptstyle{#1}}}
\newcommand{\STA}{STA\thinspace} \newcommand{\MTH}{MTH\thinspace}
\newcommand{\fxa}{\mbox{$f(x\mid\alpha)$}}
\newcommand{\fxp}{\mbox{$f(x\mid p)$}}
\newcommand{\fxt}{\mbox{$f(x\mid\theta)$}}
\newcommand{\fxat}{\mbox{$f(x\mid\alpha,\theta)$}}
\newcommand{\tp}{^{\tsf{T}}}
\newcommand{\as}{\emph{a.s.}}
\newcommand{\eg}{\emph{e.g.{}}}%
\newcommand{\etal}{\emph{et al}}
\newcommand{\etc}{\emph{etc}}
\newcommand{\ie}{\emph{i.e.{}}}
\newcommand{\pg}{\emph{p.}\thinspace}
\newcommand{\pp}{\emph{pp.}\thinspace}
%\newcommand{\half}{{\frac12}}
\newcommand{\half}{{\mathchoice{{\textstyle\frac12}} {{\textstyle\frac12}}
                {{\scriptscriptstyle\frac12}}{{\scriptscriptstyle\frac12}}}}
\newcommand{\one}{\mathbf{1}}
\newcommand{\bone}[1]{\one_{\{#1\}}}
\newcommand{\Sec}[1]{Section\thinspace(\ref{#1})}
\newcommand{\Thm}[1]{Theorem\thinspace\ref{#1}}
\newcommand{\Cor}[1]{Corollary\thinspace\ref{#1}}
\newcommand{\Eqn}[1]{Eqn\thinspace(\ref{#1})}

\newcommand{\Eqns}[2]{Eqns\thinspace(\ref{#1},\thinspace\ref{#2})}
\newcommand{\Eqss}[2]{Eqns\thinspace(\ref{#1}--\ref{#2})}
\newcommand{\Fig}[1]{Figure\thinspace(\ref{#1})}
\newcommand{\Figs}[2]{Figures\thinspace(\ref{#1}) and (\ref{#2})}
\newcommand{\Figab}[2]{Figure\thinspace(\ref{#1}#2)}
\newcommand{\Tab}[1]{Table\thinspace(\ref{#1})}
\newcommand{\jth}{{\ensuremath j^{\mbox{\tiny th}}}}%
\renewcommand{\ij}{_{ij}}
\providecommand{\ji}{_{ji}}
\newcommand{\bet}[1]{\left [#1\right ]} % bet [ ]
\newcommand{\cet}[1]{\left (#1\right )} % cet ( )
\newcommand{\set}[1]{\left\{#1\right\}} % set { }
\newcount\ola \newcount\olb \newcount\olc \newcount\old \newcount\ole
\newcount\och\newcount\level
%\DeclareMathOperator{\sgn}{sgn} would be better in preamble...
\newcommand{\sgn}{\mathop{\mathrm{sgn}}}
\newcommand{\supp}{\mathop{\mathrm{supp}}}
\newtheorem{cor}{Corollary}
\newtheorem{define}{Definition}
\newtheorem{lem}{Lemma}
%\newtheorem{prob}{Problem}
\newtheorem{prop}{Proposition}
\newtheorem{thm}{Theorem}
\newtheorem{examp}{Example}
%%%%%%%%%%%%%%%%%%%%%
\newcommand{\ed}{\emph{ed}}
\newcommand{\eds}{\emph{eds}}
\ifx\newcolumntype\undefined
  \typeout{No Array, no sweat.}
\else
  \typeout{Hey, we've got Array!}
  \newcolumntype{C}{>{$}c<{$}}
  \newcolumntype{L}{>{$}l<{$}}
  \newcolumntype{R}{>{$}r<{$}}
\fi
\def\OL#1{\par\noindent\hangindent=#1\parindent % Outline
  \kern1\hangindent\ignorespaces}%
\def\ol#1{%
    \level=#1
    \ifcase\level
    \ola=0 \olb=0 \olc=0 \old=0 \ole=0\or         % Level 0 (reset)
    \olb=0 \olc=0 \old=0 \ole=0 \advance\ola by 1 % Level 1
    \gdef\olev{\uppercase\expandafter{\romannumeral\ola}} \or
    \olc=0 \old=0 \ole=0 \advance\olb by 1        % Level 2
    \och=64 \advance\och by\olb
    \gdef\olev{\char\och}\or
    \old=0 \ole=0 \advance\olc by 1               % Level 3
    \och=48 \advance\och by\olc
    \gdef\olev{\char\och}\or
    \ole=0 \advance\old by 1                      % Level 4
    \och=96 \advance\och by\old
    \gdef\olev{\char\och}\or
    \advance\ole by 1                             % Level 5
    \gdef\olev{\romannumeral\ole} \or
    \message{Outline depth too deep: #1}\fi
    \ifnum\level>0 \OL\level\llap{\olev.\enspace}\ignorespaces\fi}%
\long\def\comment#1/*#2*/{\endcomment}%
\def\endcomment{\relax}%
\newcounter{probno}\newcounter{partno}[probno]
\newcommand{\prob}{\par\medskip\goodbreak\stepcounter{probno}\noindent\hbox
 to25mm {\textbf{Problem \arabic{probno}}.\hfil}}%
\newcommand{\cont}{\par\vfill\newpage\hbox
 to35mm {\textbf{Problem \arabic{probno}} (cont).\hfil}}%
%\newcommand{\newpart}{\par\stepcounter{partno}\alph{partno})\quad}
\newcommand{\newpart}[1][0]{\ifnum\value{partno}=0\medskip\else\vfill\fi
  \par\stepcounter{partno}\alph{partno})
  \ifnum#1=0\quad\else(#1)~\fi}
%\newcommand{\newpart}[1][0]{\par\stepcounter{partno}\alph{partno})
%  \ifnum#1=0\quad\else(#1)~\fi}
%\makeatletter
%\def\newpart{\@ifnextchar[\@with\@without}
%\def\@with#1{\vfill\relax\par\stepcounter{partno}\alph {partno} (#1)}
%\def\@without{\vfill\relax\par\stepcounter{partno}\alph {partno}}
%\makeatother
%%%%%%%%%%%%%%%%%%%%%
\newenvironment{packed_enum}{
\begin{enumerate}
  \setlength{\itemsep}{1pt}
  \setlength{\parskip}{0pt}
  \setlength{\parsep}{0pt}
}{\end{enumerate}}
%%%%%%%%%%%%%%%%%%%%%
\makeatletter % Find hours (count1) and minutes (count2) past midnight:
\count1\time \divide\count1 60 \count2=-\count1
\multiply\count2 60 \advance\count2 \time
\edef\now{\two@digits{\the\count1}:\two@digits{\the\count2}}
%\renewcommand\section{\@startsection     % Smaller and sans-serif
%    {section}{1}{\z@}{-3.5ex \@plus -1ex \@minus -.2ex}%
%    {2.3ex \@plus.2ex}{\normalfont\large\bfseries\sffamily}}
%\renewcommand\subsection{\@startsection
%    {subsection}{2}{\z@}{-3.25ex\@plus -1ex \@minus -.2ex}%
%    {1.5ex \@plus .2ex}{\normalfont\large\bfseries\sffamily}}
%\renewcommand\subsubsection{\@startsection
%    {subsubsection}{3}{\z@}{-3.25ex\@plus -1ex \@minus -.2ex}%
%    {1.5ex \@plus .2ex}{\normalfont\normalsize\bfseries\sffamily}}
%\def\@seccntformat#1{\csname the#1\endcsname.\quad} % Add . to sec num's
%\long\def\@makecaption#1#2{%                          Use . not : in cap'ns
%  \vskip\abovecaptionskip
%  \sbox\@tempboxa{#1. #2}%
%  \ifdim \wd\@tempboxa >\hsize
%    #1. #2\par
%  \else
%    \global \@minipagefalse
%    \hb@xt@\hsize{\hfil\box\@tempboxa\hfil}%
%  \fi
%  \vskip\belowcaptionskip}
\makeatother

\def\wbox#1#2#3{{\vcenter{\vbox{\hrule height.#3pt
    \hbox{\vrule width.#3pt height#1pt \kern#2pt \vrule width.#3pt}%
                            \hrule height.#3pt}}}}%
\def\Proof.{\medbreak\noindent{\bf Proof.\enspace}}
\def\qed{\nobreak{\hfill\penalty0\hbox to1truecm{}\nobreak
    \hfill$\wbox634$\par\bigskip}}%

\ifx\url\undefined
  \typeout{No url, no sweat.}
\else
  \typeout{Hey, we've got url!}
\makeatletter
\def\url@rlwstyle{%
  \@ifundefined{selectfont}{\def\UrlFont{\sf}}
    {\def\UrlFont{\small\ttfamily}}}
\makeatother
\urlstyle{rlw}
\fi
\newif\ifdraft
\def\DRAFT{\special{! userdict begin /bop-hook{
  gsave 200 30 translate 65 rotate /Times-Roman
  findfont 220 scalefont setfont 0 0 moveto 0.95
  setgray (DRAFT) show grestore }def end}\drafttrue}%
\newcommand{\uhoh}[1]{\textbf{[Uh oh: #1]}}
% From tug www.tug.org/TUGboat/tb22-4/tb72wilson.pdf
\newif\ifsame
\newcommand{\strcfstr}[2]{%
  \samefalse
  \begingroup
    \def\1{#1}\def\2{#2}%
    \ifx\1\2\endgroup \sametrue
    \else \endgroup
    \fi}
% Use this with 'jobname' for cond'l compilation for *names* of files
\def\rev$Revi#1: #2 ${#2}
\def\dat$Dat#1: #2 #3 ${#2}
\def\need#1{\vskip0pt plus#1in\penalty-250\vskip0pt plus-#1in}%
\def\SVNtz$#1 -0#200#3${\global\def\tz{\ifcase#2 GMT\or-1\or-2\or ADT\or
    EDT\or EST \or CST\or MST\or PST\or AKST\or -10\or HST\else -#2\fi}}
%=  AST\or CDT \or MDT\or PDT\or AKDT\else -#2\fi}}
\let\ciao=\endinput
